%-----------------------------------------------------------------------------%
\chapter*{ABSTRACT}
%-----------------------------------------------------------------------------%
\noindent
{\itshape Automatic text categorization is one of the solution to help readers find the data they wanted quickly. In this research, the method used for learning and inference is Simplified Labeled Latent Dirichlet Allocation Classifier (SLLDA-C) based on Labeled Latent Dirichlet Allocation method and the method used for feature selection is TF-IDF. The first step is the contents of each document that has been labeled will be pre-processing which includes case folding, filtering, tokenizing, stopword removing, and stemming. The labels used in this research are machine learning, natural language processing, and image processing. The next step is to perform feature selection against the contents with TF-IDF for reducing the number of words that will be processed when learning process takes place. The words that have been selected will be learning with SLLDA-C to produce classifier model. The experiment will be done with 180 training datas and 20 testing datas. Based on the experiment, using TF-IDF the best average F-Measure results obtained is 90\% with learning time for 27 seconds.
\\\\
\noindent \textbf{Keyword:} Text Categorization, SLLDA-C, Labeled Latent Dirichlet Allocation, TF-IDF, F-Measure}
\\
%-----------------------------------------------------------------------------%
\chapter{KESIMPULAN DAN SARAN}
%-----------------------------------------------------------------------------%

\vspace{4.5pt}

\section{Kesimpulan}
\indent
Di bawah ini merupakan kesimpulan yang didapat berdasarkan penelitian kategorisasi dokumen dengan menggunakan {\itshape Labeled Latent Dirichlet Allocation} dengan TF-IDF.

\begin{enumerate}[nolistsep,leftmargin=0.5cm]
\item
Penggunaan {\itshape bag of words} untuk {\itshape bigram} dan {\itshape trigram} menggunakan n TF-IDF tertinggi kurang sesuai untuk digunakan pada metode {\itshape Labeled Latent Dirichlet Allocation}, karena berdasarkan pengujian hasil akurasi yang dihasilkan kurang baik (di bawah 40\%).
\item
Penggunaan metode {\itshape Labeled Latent Dirichlet Allocation} dengan kombinasi fitur TF-IDF, {\itshape unigram}, dan {\itshape single label} menghasilkan nilai rata-rata {\itshape F-Measure} sebesar 80\% hingga 90\%, sedangkan untuk kombinasi fitur TF-IDF, {\itshape unigram}, dan {\itshape multi label} menghasilkan nilai rata-rata {\itshape F-Measure} sebesar 50\% hingga 60\%.
\item
Hasil akurasi menggunakan {\itshape single label} lebih baik dibandingkan menggunakan {\itshape multi label} karena pada kasus {\itshape single label}, label yang dihasilkan oleh sistem hanya berjumlah 1 sehingga ketika terdapat 1 label yang cocok, maka sistem akan mengganggap hasil tersebut benar. Sedangkan untuk kasus {\itshape multi label}, label yang dihasilkan oleh sistem berjumlah lebih dari 1 sehingga jika terdapat 1 label saja yang tidak cocok, maka sistem akan mengganggap hasilnya salah.
\item
Penggunaan n TF-IDF terbaik terjadi ketika metode {\itshape Labeled Latent Dirichlet Allocation} menggunakan 600 TF-IDF tertinggi, jenis{\itshape bag of words} yang digunakan adalah {\itshape unigram}, dan keluaran dari sistem berupa {\itshape single label}. Hasil akurasi yang dihasilkan oleh kombinasi fitur tersebut adalah sebesar 90\%.
\item
Waktu pembelajaran tercepat untuk kasus {\itshape unigram}, {\itshape bigram}, dan {\itshape trigram} dengan {\itshape single label} terbaik terjadi ketika {\itshape unigram} menggunakan 600 TF-IDF tertinggi. Waktu yang dibutuhkan oleh kombinasi fitur tersebut adalah 27 detik.
\item
Kasus terbaik untuk kategorisasi dokumen menggunakan {\itshape Labeled Latent Dirichlet Allocation} terjadi ketika tanpa menggunakan TF-IDF, jenis {\itshape bag of words} yang digunakan adalah {\itshape unigram} atau {\itshape bigram}, dan keluaran dari sistem berupa {\itshape single label}. Hasil akurasi yang dihasilkan oleh kombinasi fitur tersebut adalah sebesar 95\%.
\item
Penggunaan metode {\itshape Labeled Latent Dirichlet Allocation} tanpa TF-IDF lebih tinggi daripada menggunakan n TF-IDF karena ada satu kondisi dimana ketika menggunakan TF-IDF terdapat beberapa kata yang sebenarnya penting tetapi tidak terpilih. Hal ini dapat terjadi karena kata yang tidak terpilih tersebut lebih banyak kemunculannya di beberapa dokumen dibanding kata yang lainnya, sehingga nilai TF-IDF untuk kata tersebut akan bernilai kecil.

\end{enumerate}

\section{Saran}
\indent
Di bawah ini merupakan saran yang didapatkan berdasarkan penelitian yang perlu dipertimbangkan untuk pengembangan penelitian ke depannya.

\begin{enumerate}[nolistsep,leftmargin=0.5cm]
\item
Memperbanyak data {\itshape training} yang digunakan dengan label yang bervariasi, karena dengan banyaknya data, kata-kata yang akan diolah dalam proses pembelajaran pun akan bertambah banyak dan akurasi menggunakan metode {\itshape Labeled Latent Dirichlet Allocation} juga akan meningkat.

\end{enumerate}

\newpage

\begin{thebibliography}{12}

\bibitem{1}
{Harrington, P. (2012). Machine Learning in Action. Manning.}
\bibitem{2}
{Tsai, S. C., Jiang, J. Y., dan Lee, S. J. (2010). A Mixture Approach for Multi-Label Document Classification. halaman 387 – 391. IEEE.}
\bibitem{3}
{Feng, G. Z., Guo, J., Jing, B. Y., dan Sun, T. (2015). Feature Subset Selection Using Naive Bayes for Text Classification. halaman 1-8. ACM.}
\bibitem{4}
{Bai, Y. dan Wang, J. (2015). News Classifications with Labeled LDA. jilid 1, halaman 75-83. IEEE.}
\bibitem{5}
{Bird, S., Klein, E., dan Loper, E. (2009). Natural Language Processing with Python: Analyzing Text with the Natural Language Toolkit. O'Reilly Media, Inc.}
\bibitem{6}
{Moschitti, A. (2003). Natural Language Processing and Automated Text Categorization. halaman 1-133. CiteSeerX.}
\bibitem{7}
{Islam, S. M., Jubayer, F. E. M., dan Ahmed, S. I. (2017). A Support Vector Machine mixed with TF-IDF Algorithm to Categorize Bengali Document. halaman 191-196. IEEE.}
\bibitem{8}
{Fitzgerald, M. (2012). Introducing Regular Expressions: Unraveling Regular Expressions, Step-by-Step. O'Reilly Media, Inc.}
\bibitem{9}
{Jivani, M. A. G. (2011). A Comparative Study of Stemming Algorithms. halaman 1930-1938. IEEE.}
\bibitem{10}
{Blei, D. M., Ng, A. Y., dan Jordan, M. I. (2003). Latent Dirichlet Allocation. jilid 3, halaman 993-1022. ACM.}
\bibitem{11}
{Ramage, D., Hall, D., Nallapati, R., dan Manning, C. D. (2009). Labeled LDA: A supervised topic model for credit attribution in multi labeled-corpora. jilid 1, halaman 248-256. ACM.}
\bibitem{12}
{Griffiths, T. L. dan Steyvers, M. (2004). Finding scientific topics. jilid 101, halaman 5228-5235. National Academy of Sciences.}
\bibitem{13}
{Darling, W. M. (2011). A Theoretical and Practical Implementation Tutorial on Topic Modeling and Gibbs Sampling. halaman 1-10. Research Gate.}

\end{thebibliography}

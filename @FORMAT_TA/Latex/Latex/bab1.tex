%-----------------------------------------------------------------------------%
\chapter{PENDAHULUAN}
%-----------------------------------------------------------------------------%

\vspace{4.5pt}

\section{Latar Belakang} \label{sec:latar_belakang}
\indent
Seiring dengan pertumbuhan internet yang semakin cepat, ketersediaan dari informasi teks juga ikut meningkat pesat. Terkadang para pembaca membutuhkan waktu untuk menemukan data yang yang mereka inginkan. Salah satu solusi untuk menangani masalah tersebut adalah dengan menggunakan kategorisasi teks otomatis. Penelitian ini akan menciptakan sebuah sistem kategorisasi yang dapat mengelompokkan sejumlah data ke dalam beberapa kelompok yang sesuai. Data yang akan digunakan pada penelitian ini adalah data mengenai dokumen penelitian. Dari kumpulan data tersebut, data akan dikelompokkan ke dalam beberapa kelompok berdasarkan topik yang sesuai. Bagian dokumen penelitian yang akan digunakan sebagai masukan untuk sistem kategorisasi adalah bagian abstrak.

\indent
Beberapa metode yang umum digunakan untuk kategorisasi di antaranya adalah {\itshape Support Vector Machine} [1], K-NN [2], dan {\itshape Naive Bayes} [3]. {\itshape Support Vector Machine} memiliki kelebihan untuk mengatasi beberapa fitur yang tidak relevan, teks yang jarang muncul, dan dapat melakukan generalisasi {\itshape sample} dengan baik (menghindari {\itshape overfitting}) tetapi SVM memiliki kelemahan juga pada sulitnya pemilihan parameter SVM yang optimal [1]. Metode K-NN efektif jika sistem memiliki data {\itshape training} dalam jumlah besar. Dengan K-NN juga, data {\itshape training robust} terhadap data {\itshape noise}, tetapi K-NN memiliki kelemahan pada penentuan nilai K dan atribut yang digunakan (tidak efisien) serta biaya komputasinya tinggi [2]. {\itshape Naïve Bayes} memiliki kelebihan yaitu algoritma yang digunakan sederhana, tetapi {\itshape Naïve Bayes} memiliki kelemahan untuk menangani data yang jarang muncul dan penentuan parameter yang bersifat independen [3]. 

\indent
Metode yang telah disebutkan di atas hanya mengelompokkan dokumen ke dalam 1 kategori yang paling dominan saja ({\itshape single label}), tetapi ada keadaan dimana suatu dokumen dapat memiliki lebih dari 1 kategori ({\itshape multi label}). {\itshape Labeled Latent Dirichlet Allocation} merupakan salah satu metode yang memungkinkan sebuah dokumen untuk memiliki lebih dari 1 kategori. Untuk menggunakan metode {\itshape Labeled Latent Dirichlet Allocation}, ada beberapa hal yang perlu dilakukan adalah melakukan {\itshape pre-processing} terhadap kata-kata yang terdapat pada dokumen, menghitung nilai TF-IDF, dan melakukan pembelajaran atau menghasilkan kesimpulan dengan menggunakan {\itshape Gibbs Sampling}. Kelebihan dari metode ini adalah dapat menghasikan lebih dari 1 kategori untuk setiap dokumen, memecahkan masalah dari K-NN (tidak efisien dan waktu komputasi tinggi) dan {\itshape Naïve Bayes} (tidak fleksibel), serta dapat mengungguli kelebihan yang dimiliki SVM juga. 

\indent
Meskipun {\itshape Labeled Latent Dirichlet Allocation} dapat dikatakan sebagai salah satu metode klasifikasi yang baik saat ini, tingkat akurasi dan waktu komputasi dari metode {\itshape Labeled Latent Dirichlet Allocation} masih dapat ditingkatkan dan dikurangi lagi untuk dapat menghasilkan hasil yang lebih akurat dan cepat [4]. Berdasarkan hal tersebut, salah satu pendekatan yang dapat digunakan adalah menggunakan metode {\itshape feature selection}, dimana metode ini berfungsi untuk mendapatkan kata-kata yang penting saja. Metode {\itshape feature selection} yang digunakan adalah TF-IDF, sehingga metode yang akan digunakan dalam penelitian ini adalah penggabungan {\itshape Labeled Latent Dirichlet Allocation} dan TF-IDF. Hasil dari sistem kategorisasi ini adalah berupa kategori (label).

\section{Rumusan Masalah}
\indent
Berdasarkan masalah yang telah diuraikan di atas, maka dapat dirumuskan beberapa masalah yang di antaranya adalah :

\begin{enumerate}[nolistsep,leftmargin=0.5cm]
\item 
Bagaimana menerapkan {\itshape Labeled Latent Dirichlet Allocation} untuk melakukan pengelompokan dokumen penelitian?
\item
Bagaimana proses kategorisasi dokumen penelitian dengan {\itshape  Labeled Latent Dirichlet Allocation}?
\item 
Bagaimana meningkatkan tingkat akurasi serta menurunkan waktu komputasi proses kategorisasi dokumen penelitian menggunakan {\itshape Labeled Latent Dirichlet Allocation}?
\item 
Apa pengaruh TF-IDF terhadap hasil akhir {\itshape Labeled Latent Dirichlet Allocation}?
\end{enumerate}

\section{Tujuan Penelitian}
\indent 
Berdasarkan latar belakang dan rumusan masalah tersebut, maka tujuan utama dari penelitian ini adalah :

\begin{enumerate}[nolistsep,leftmargin=0.5cm]
\item 
Mengelompokan dokumen penelitian ke dalam beberapa kategori untuk dijadikan sebagai {\itshape corpus}.
\item 
Dapat menentukan beberapa kategori yang sesuai terhadap dokumen penelitian yang diuji.
\item 
Meneliti bahwa dengan adanya metode {\itshape feature selection} pada proses {\itshape Labeled Latent Dirichlet Allocation}, tingkat akurasi yang akan dihasilkan akan lebih akurat dan waktu komputasi yang dibutuhkan akan lebih cepat.
\end{enumerate}

\section{Batasan Masalah}
\indent 
Agar penelitian ini dapat sesuai dengan tujuan utama yang telah disebutkan di atas, maka digunakan beberapa batasan seperti di bawah ini :

\begin{enumerate}[nolistsep,leftmargin=0.5cm]
\item 
Dokumen penelitian yang digunakan adalah dokumen berbahasa Inggris.
\item 
{\itshape Stemmer} yang akan digunakan untuk melakukan proses {\itshape stemming} adalah {\itshape Snowball Stemmer}.
\item 
Dokumen penelitian yang digunakan bersumber dari kumpulan tugas akhir mahasiswa Institut Teknologi Harapan Bangsa jurusan Informatika, IEEE, ACM, dan {\itshape Research Gate}.
\item 
Pada proses kategorisasi sistem tidak dapat menangani frase, homonim, homograf, polisemi, singkatan, dan kata tidak baku.
\item 
Bagian dokumen penelitian yang akan digunakan untuk pengolahan data adalah bagian abstrak.
\item
Label yang digunakan pada dokumen penelitian adalah {\itshape Machine Learning}, {\itshape Natural Language Processing}, dan {\itshape Image Processing}.
\item
Sistem yang dibuat hanya berbasis {\itshape web} dengan sistem operasi {\itshape Windows}.
\end{enumerate}

\section{Metode Penelitian}
\indent Di bawah ini merupakan tahapan-tahapan yang akan dilakukan selama penelitian :

\begin{enumerate}[nolistsep,leftmargin=0.5cm]
\item
Studi literatur mengenai {\itshape text pre-processing}, {\itshape feature selection}, {\itshape text classfication}, dan algoritma {\itshape Labeled Latent Dirichlet Allocation} yang digunakan untuk kategorisasi dokumen.
\item
Analisis terhadap {\itshape Labeled Latent Dirichlet Allocation} serta algoritma-algoritma yang digunakan.
\item
Perancangan sistem kategorisasi meliputi perancangan {\itshape database}, fungsi-fungsi yang terdapat dalam sistem, algoritma yang digunakan, dan perancangan antarmuka.
\item
Implementasi algoritma {\itshape Labeled Latent Dirichlet Allocation} pada sistem kategorisasi berdasarkan rancangan yang telah dibuat.
\item
Pengujian terhadap hasil penelitian dari segi ketepatan saat digunakan.
\end{enumerate}

\section{Kontribusi Penelitian}
\indent
Pada penelitian ini, penulis akan melakukan kategorisasi dokumen penelitian dengan menggunakan metode {\itshape Labeled Latent Dirichlet Allocation}.  Penulis akan mencoba menggunakan bagian abstrak pada dokumen penelitian sebagai data yang akan diolah untuk pembelajaran maupun pengambilan keputusan. Selain itu juga, Penulis akan mencoba untuk mengimplementasikan metode {\itshape feature selection} pada penelitian ini. Metode {\itshape feature selection} yang akan penulis gunakan adalah TF-IDF. Penggunaan metode {\itshape feature selection} ini merupakan salah satu tujuan penulis untuk meneliti apakah dengan TF-IDF tingkat akurasi yang didapatkan akan semakin akurat serta apakah dengan TF-IDF juga waktu komputasi yang dibutuhkan mesin dapat lebih cepat dibanding tanpa menggunakan TF-IDF.

\section{Sistematika Penulisan}
\indent 
Penyusunan penelitian ini akan dibuat dalam beberapa tahapan seperti di bawah ini :

\noindent 
\textbf{BAB I \hspace{1cm} PENDAHULUAN}
\begin{addmargin}[2.35cm]{0em}
Bab ini merupakan bagian pendahuluan, yang menjelaskan tentang kenapa penulis ingin melakukan penelitian ini, rumusan masalah, tujuan utama dari penelitian, batasan masalah, metodologi penelitian, kontribusi penelitian, serta sistematika penulisan.
\end{addmargin}

\noindent 
\textbf{BAB II \hspace{1cm} LANDASAN TEORI}
\begin{addmargin}[2.35cm]{0em}
Bab ini merupakan bagian landasan teori, yang menjelaskan tentang studi literatur yang telah dilakukan lalu dituangkan sebagai landasan teori penelitian ini.
\end{addmargin}

\noindent 
\textbf{BAB III \hspace{1cm} ANALISIS DAN PERANCANGAN}
\begin{addmargin}[2.35cm]{0em}
Bab ini merupakan bagian analisis dan perancangan, yang menjelaskan tentang proses analisa hingga perancagan dari penelitan ini.
\end{addmargin}

\noindent 
\textbf{BAB IV \hspace{1cm} IMPLEMENTASI DAN PENGUJIAN}
\begin{addmargin}[2.35cm]{0em}
Bab ini merupakan bagian implementasi dan pengujian, yang menjelaskan implementasi sistem dan pengujian terhadap sistem yang telah dibangun.
\end{addmargin}

\noindent 
\textbf{BAB V \hspace{1cm}  KESIMPULAN DAN SARAN}
\begin{addmargin}[2.35cm]{0em}
Bab ini merupakan bagian penutup, yang menjelaskan kesimpulan dan saran dari penelitian yang telah dilakukan.
\end{addmargin}

\newpage
%-----------------------------------------------------------------------------%
\chapter*{ABSTRAK}
%-----------------------------------------------------------------------------%
\noindent
Kategorisasi teks secara otomatis adalah salah satu solusi untuk membantu para pembaca menemukan data yang mereka inginkan dengan cepat. Dalam penelitian ini, metode yang digunakan untuk pembelajaran dan pengambilan keputusan adalah {\itshape Simplified Labeled Latent Dirichlet Allocation Classifier} (SLLDA-C) yang didasarkan pada metode {\itshape Labeled Latent Dirichlet Allocation} dan metode yang digunakan untuk seleksi fitur adalah TF-IDF. Langkah pertama adalah isi setiap dokumen yang telah diberi label akan dilakukan {\itshape pre-processing} yang meliputi proses {\itshape case folding}, {\itshape filtering}, {\itshape tokenizing}, {\itshape stopword removing}, dan {\itshape stemming}. Label yang digunakan dalam penitilian ini di antaranya adalah {\itshape machine learning}, {\itshape natural language processing}, dan {\itshape image processing}. Langkah selanjutnya adalah melakukan seleksi fitur terhadap isi dokumen dengan TF-IDF untuk mengurangi jumlah kata yang akan diolah ketika proses klasifikasi berlangsung. Kata-kata yang telah terpilih akan dilakukan pembelajaran dengan SLLDA-C untuk menghasilkan suatu {\itshape classifier model}. Pengujian akan dilakukan dengan jumlah data belajar sebanyak 180 data dan data uji sebanyak 20 data. Berdasarkan pengujian, dengan menggunakan TF-IDF hasil rata-rata {\itshape F-Measure} terbaik yang didapat adalah sebesar 90\% dengan waktu pembelajaran yang dibutuhkan adalah selama 27 detik.
\\\\
\noindent \textbf{Kata Kunci:} Kategorisasi Teks, SLLDA-C, {\itshape Labeled Latent Dirichlet Allocation}, TF-IDF, {\itshape F-Measure}
\\
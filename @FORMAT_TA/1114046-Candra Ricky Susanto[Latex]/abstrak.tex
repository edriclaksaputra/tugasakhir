\chapter*{ABSTRAK}

\noindent Analisis sentimen pada media sosial telah menjadi salah satu topik penelitian yang paling ditargetkan pada \textit{Natural Languange Processing} (NLP) \cite{2}. Analisis sentimen ini bertujuan untuk menentukan nilai polaritas dari sebuah dokumen secara otomatis. Salah satu tantangan pada analisis sentimen adalah melakukan klasifikasi terhadap teks sarkasme \cite{3}. Dalam penelitian ini, dikembangkan sistem analisis sentimen yang dapat melakukan klasifikasi teks positif, teks negatif, teks netral, dan teks sarkasme. Metode klasifikasi yang digunakan adalah \textit{Support Vector Machine} (SVM). Beberapa fitur yang digunakan untuk memberikan informasi dari dokumen adalah \textit{number of interjection word}, \textit{question word }\cite{5}, \textit{unigram, sentiment score,  capitalization}, \textit{topic} \cite{4}, \textit{part of speech} dan \textit{punctuation based} \cite{3}. Pengujian dilakukan dengan 2 teknik klasifikasi, yaitu \textit{levelled method} dan \textit{direct method} \cite{5}. Berdasarkan pengujian yang dilakukan, hasil akurasi mencapai 72\% yang diperoleh menggunakan metode SVM dengan teknik klasifikasi \textit{direct method}.

\noindent\textbf{Kata Kunci}: \textit{Natural Language Processing}, \textit{Classification}, \textit{Support Vector Machine}, Analisis Sentimen
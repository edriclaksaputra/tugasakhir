\chapter*{ABSTRACT}
\noindent Sentiment analysis on social media has become one of the most targeted research topics in Natural Languange Processing (NLP) \cite{2}. This sentiment analysis aims to determine the polarity value of a document automatically. One of the challenges in the sentiment analysis is to classify sarcasm text \cite{3}. In this study, developed a system of sentiment analysis that can classify positive text, negative text, neutral text, and sarcasm text. The classification method used is Support Vector Machine (SVM). Some of the features used to provide information from documents are number of interjection word, question word \cite{5}, unigram, sentiment score, capitalization, topic \cite{4}, part of speech and punctuation based \cite{3}. Testing is done by 2 classification techniques, namely levelled method and direct method \cite{5}. Based on the tests performed, the accuracy result was 72\% obtained using the SVM method with the classification technique direct method.

\noindent\textbf{Keyword}: Natural Language Processing, Classification, Support Vector Machine, Sentiment Analysis
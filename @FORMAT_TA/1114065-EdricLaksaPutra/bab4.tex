%-----------------------------------------------------------------------------%
\chapter{IMPLEMENTASI DAN PERBANDINGAN}
%-----------------------------------------------------------------------------%

%
\vspace{4.5pt}
Pada bab ini akan menjelaskan tentang pengimplementasian dan pengujian terhadap analisis sentimen yang telah dibangun berdasarkan bab-bab sebelumnya.
\section{Lingkungan Aplikasi}
Dalam aplikasi terbagi menjadi dua bagian, yaitu lingkungan implementasi perangkat keras dan perangkat lunak. Di dalam penelitian ini, perangkat keras yang digunakan adalah:
\begin{enumerate}[leftmargin=*]
	\item Asus A45A
	\item Processor Intel i3-2370M CPU 2.40GHz
	\item RAM 6 GB.
\end{enumerate}

Spesifikasi perangkat lunak yang digunakan untuk pengembangan sistem adalah:
\begin{enumerate}[leftmargin=*]
	\item Sistem Operasi\quad\quad\quad\,: Windows 10 Enterprise 1709 64-bit.
	\item Tool Pengembangan\quad: Eclipse Java EE IDE for Web Developers.
	\item Versi\quad\quad\quad\quad\quad\quad\quad\,: Neon.3 Release (4.6.3).
\end{enumerate}

\section{Daftar \textit{Project}, \textit{Class} dan \textit{Method}}
Pada bagian ini akan dijelaskan mengenai \textit{Project}, \textit{class} dan \textit{method} yang digunakan dalam pengembangan sistem analisis.
\subsection{\textit{Project} Customer}
\textit{Project Customer} berisi susunan \textit{class} pembentuk \textit{webservice customer}. Dalam project ini terdapat 4 buah \textit{package} yang berfungsi untuk memisahkan \textit{class} sesuai dengan fungsinya masing-masing. Ke empat \textit{package} ini yaitu \textit{package} controller yang berisi method-method yang dijalankan pada \textit{web service}, \textit{package} model yang berisi entitas dan tabel, \textit{package} repository yang menghubungkan entitas di \textit{package} model dengan database, dan \textit{package} service yang menginisialisasi \textit{webservice} yang akan terbentuk. Pada bagian ini akan dijelaskan mengenai \textit{class} dan \textit{method} yang digunakan pada pengembangan \textit{ webservice customer}:
\subsubsection{\textit{Package} Controller}
\textit{Package} controller berisi method API yang dapat digunakan untuk berkomunikasi dengan \textit{service} customer. \textit{Class} yang terdapat dalam \textit{package} ini adalah \textit{class} CustomerController dan CustomergroupController yang berisi 5 buah fungsi API. Di bawah ini merupakan daftar \textit{class} untuk \textit{package} controller pada \textit{webservice} customer.
\begin{table}[H]
	\small
	\centering
	\caption{Daftar {\itshape Class} pada {\itshape Package} Controller}
	\begin{adjustbox}{width=1\textwidth}
		\begin{tabular}{| p {3 cm} | p {8 cm} | p {3 cm} |}
			\hline
			{\bfseries Package} & {\bfseries Class} & {\bfseries Jenis Class} \\
			\hline
			\multirow{2}{*}{Controller} & CustomerController & {\itshape Class} \\
			& CustomergroupController & {\itshape Class} \\
			\hline
		\end{tabular}
	\end{adjustbox}
\end{table}
\begin{table}[H]
	\caption{Daftar \textit{Method} pada \textit{Class} CustomerController}
	\centering
	\small
	\begin{adjustbox}{width=1\textwidth}	
	\begin{tabular}{|p{0.4cm}|p{3.2cm}|p{1.4cm}|p{1.7cm}|p{1.55cm}|p{3cm}|}
		\hline
		\multirow{2}{*}{\textbf{No}} & \multirow{2}{*}{\textit{\textbf{Method}}} & \multicolumn{2}{c|}{\textit{\textbf{Input}}} & \multirow{2}{*}{\textit{\textbf{Output}}} & 
		\multirow{2}{*}{\textbf{Keterangan}}\\
		\cline{3-4}
		& & \textbf{Tipe} & \textbf{Variabel} & & \\
		\hline
		1 & addCustomer & void & Customer & Customer, HttpStatus & \textit{Method} ini digunakan untuk membuat \textit{object} customer yang baru melalui \textit{webservice}\\
		\hline
		2 & updateCustomer & void & Customer & HttpStatus & \textit{Method} ini digunakan untuk mengubah attribut dari \textit{object} customer yang baru melalui \textit{webservice}\\
		\hline
		3 & getCustomer & void & id & Customer, HttpStatus & \textit{Method} ini digunakan untuk mengambil satu \textit{object} customer yang baru melalui \textit{webservice}\\
		\hline
	\end{tabular}
	\end{adjustbox}
\end{table}
\begin{table}[H]
	\centering
	\small
	\begin{adjustbox}{width=1\textwidth}	
		\begin{tabular}{|p{0.4cm}|p{2.5cm}|p{1cm}|p{1.1cm}|p{3.4cm}|p{3cm}|}
			\hline
			4 & getAllCustomer & void & - & $<$List$<$Customer$>$$>$, HttpStatus & \textit{Method} ini digunakan untuk mengambil semua \textit{list} \textit{object} customer yang baru melalui \textit{webservice}\\
			\hline
			5 & deleteCustomer & void & id &HttpStatus & \textit{Method} ini digunakan untuk menghapus satu \textit{object} customer yang baru melalui \textit{webservice}\\
			\hline
		\end{tabular}
	\end{adjustbox}
\end{table}
\begin{table}[H]
	\caption{Daftar \textit{Method} pada \textit{Class} CustomergroupController}
	\centering
	\small
	\begin{adjustbox}{width=1\textwidth}	
		\begin{tabular}{|p{0.4cm}|p{3.5cm}|p{1.4cm}|p{1.7cm}|p{1.55cm}|p{3cm}|}
			\hline
			\multirow{2}{*}{\textbf{No}} & \multirow{2}{*}{\textit{\textbf{Method}}} & \multicolumn{2}{c|}{\textit{\textbf{Input}}} & \multirow{2}{*}{\textit{\textbf{Output}}} & 
			\multirow{2}{*}{\textbf{Keterangan}}\\
			\cline{3-4}
			& & \textbf{Tipe} & \textbf{Variabel} & & \\
			\hline
			1 & addCustomerGroup & Customer & customer & customer, HttpStatus & \textit{Method} ini digunakan untuk membuat \textit{object} CustomerGroup yang baru melalui \textit{webservice}\\
			\hline
			2 & updateCustomerGroup & Customer & customer & HttpStatus & \textit{Method} ini digunakan untuk mengubah attribut dari \textit{object} CustomerGroup yang baru melalui \textit{webservice}\\
			\hline
			3 & getCustomerGroup & Customer & customer & customer, HttpStatus & \textit{Method} ini digunakan untuk mengambil satu \textit{object} CustomerGroup yang baru melalui \textit{webservice}\\
			\hline
			4 & getAllCustomerGroup & Customer & customer & customer, HttpStatus & \textit{Method} ini digunakan untuk mengambil semua \textit{list} \textit{object} CustomerGroup yang baru melalui \textit{webservice}\\
			\hline
		\end{tabular}
	\end{adjustbox}
\end{table}
\begin{table}[H]
	\centering
	\small
	\begin{adjustbox}{width=1\textwidth}	
		\begin{tabular}{|p{0.4cm}|p{3.5cm}|p{1.4cm}|p{1.7cm}|p{1.55cm}|p{3cm}|}
			\hline
			5 & deleteCustomerGroup & Customer & customer & customer, HttpStatus & \textit{Method} ini digunakan untuk menghapus satu \textit{object} CustomerGroup yang baru melalui \textit{webservice}\\
			\hline
		\end{tabular}
	\end{adjustbox}
\end{table}
\subsubsection{\textit{Package} Model Customer}
\textit{Package} customer model berisi entitas-entitas penyusun dari \textit{service} customer. Berikut ini merupakan daftar \textit{class} untuk \textit{package} model.
\begin{table}[H]
	\small
	\centering
	\caption{Daftar {\itshape Class} pada {\itshape Package} model}
	\begin{adjustbox}{width=1\textwidth}
		\begin{tabular}{| p {3 cm} | p {8 cm} | p {3 cm} |}
			\hline
			{\bfseries Package} & {\bfseries Class} & {\bfseries Jenis Class} \\
			\hline
			\multirow{13}{*}{Model} & Regency & {\itshape Class} \\
			& AddressType & {\itshape Class} \\
			& Contact & {\itshape Class} \\
			& ContactAddress & {\itshape Class} \\
			& ContactEducation & {\itshape Class} \\
			& Country & {\itshape Class} \\
			& Customer & {\itshape Class} \\
			& Customergroup & {\itshape Class} \\
			& HealthConsumer & {\itshape Class} \\
			& Insurance & {\itshape Class} \\
			& InsuranceBridgeConf & {\itshape Class} \\
			& Profession & {\itshape Class} \\
			& Province & {\itshape Class} \\
			\hline
		\end{tabular}
	\end{adjustbox}
\end{table}
\begin{table}[H]
	\caption{Daftar attribut pada \textit{Class} Customer}
	\centering
	\small
	\begin{adjustbox}{width=1\textwidth}	
		\begin{tabular}{|p{4cm} p{2.1cm} p{3cm} p{3.1cm}|}
			\hline
			\multicolumn{2}{|l}{\textbf{Variabel:}}&\multicolumn{2}{l|}{\textbf{Variabel:}}\\
			boolean&cust\_no&Date&registrationdate\\
			boolean&active&int&idPriceLevel\\
			double&creditlimit&double&disc\\
			Customergroup&customergroup&&\\
			\hline
		\end{tabular}
	\end{adjustbox}
\end{table}

\begin{table}[H]
	\caption{Daftar attribut pada \textit{Class} Customergroup}
	\centering
	\small
	\begin{adjustbox}{width=1\textwidth}	
		\begin{tabular}{|p{4cm} p{2.1cm} p{3cm} p{3.1cm}|}
			\hline
			\multicolumn{2}{|l}{\textbf{Variabel:}}&\multicolumn{2}{l|}{}\\
			long&systemid&&\\
			String&groupname&&\\
			String&memo&&\\
			\hline
		\end{tabular}
	\end{adjustbox}
\end{table}

\begin{table}[H]
	\caption{Daftar attribut pada \textit{Class} Contact}
	\centering
	\small
	\begin{adjustbox}{width=1\textwidth}	
		\begin{tabular}{|p{2cm} p{2.1cm} p{5cm} p{3.1cm}|}
			\hline
			\multicolumn{2}{|l}{\textbf{Variabel:}}&\multicolumn{2}{l|}{\textbf{Variabel:}}\\
			long&systemid&String&initial\\
			String&lastname&int&amountofchildren\\
			int&maritalstatus&List$<$ContactAddress$>$&arrAddress\\
			String&middlename&Collection$<$ContactEducation$>$&arrEdu\\
			Date&birthday&String&notes\\
			String&birthplace&String&officeext\\
			String&bloodtype&String&passportid\\
			double&bodyheight&byte[]&photo\\
			double&bodyweight&String&prefixtitle\\
			String&citizenid&String&sms\\
			String&citizentype&String&suffixtitle\\
			Country&citizenship&Date&sys\_createdate\\
			String&email&String&sys\_creator\\
			String&firstname&Date&sys\_lastupdate\\
			int&gender&String&sys\_lastupdater\\
			String&homephone&String&taxid\\
			String&messaging&String&web\\
			\hline
		\end{tabular}
	\end{adjustbox}
\end{table}
\begin{table}[H]
	\caption{Daftar attribut pada \textit{Class} Contact Address}
	\centering
	\small
	\begin{adjustbox}{width=1\textwidth}	
		\begin{tabular}{|p{4cm} p{2.1cm} p{3cm} p{3.1cm}|}
			\hline
			\multicolumn{2}{|l}{\textbf{Variabel:}}&\multicolumn{2}{l|}{\textbf{Variabel:}}\\
			Long&systemid&String&postcode\\
			AddressType&addresstype&Regency&regency\\
			String&street&boolean&asbillingaddress\\
			String&fax&Date&sys\_createdate\\
			Contact&owner&String&sys\_creator\\
			String&phone&Date&sys\_lastupdate\\
			String&sys\_lastupdater&&\\
			\hline
		\end{tabular}
	\end{adjustbox}
\end{table}
\begin{table}[H]
	\caption{Daftar attribut pada \textit{Class} Address Type}
	\centering
	\small
	\begin{adjustbox}{width=1\textwidth}	
		\begin{tabular}{|p{4cm} p{2.1cm} p{3cm} p{3.1cm}|}
			\hline
			\multicolumn{2}{|l}{\textbf{Variabel:}}&\multicolumn{2}{l|}{\textbf{}}\\
			Integer&systemid&&\\
			String&memo&&\\
			String&typename&&\\
			\hline
		\end{tabular}
	\end{adjustbox}
\end{table}
\begin{table}[H]
	\caption{Daftar attribut pada \textit{Class} Contact Education}
	\centering
	\small
	\begin{adjustbox}{width=1\textwidth}	
		\begin{tabular}{|p{3cm} p{3.1cm} p{3cm} p{3.1cm}|}
			\hline
			\multicolumn{2}{|l}{\textbf{Variabel:}}&\multicolumn{2}{l|}{\textbf{\textbf{Variabel:}}}\\
			Date&finishrolling&Contact&owner\\
			String&gpa&String&sponsors\\
			int&grade&Date&startrolling\\
			String&institutionaladdress&Date&sys\_createdate\\
			String&institutionname&int&sys\_creator\\
			String&major&Date&sys\_lastupdate\\
			String&notes&int&sys\_lastupdater\\
			\hline
		\end{tabular}
	\end{adjustbox}
\end{table}
\begin{table}[H]
	\caption{Daftar attribut pada \textit{Class} Country}
	\centering
	\small
	\begin{adjustbox}{width=1\textwidth}	
		\begin{tabular}{|p{4cm} p{2.1cm} p{3cm} p{3.1cm}|}
			\hline
			\multicolumn{2}{|l}{\textbf{Variabel:}}&\multicolumn{2}{l|}{\textbf{}}\\
			Long&systemid&&\\
			String&name&&\\
			\hline
		\end{tabular}
	\end{adjustbox}
\end{table}
\begin{table}[H]
	\caption{Daftar attribut pada \textit{Class} HealthConsumer}
	\centering
	\small
	\begin{adjustbox}{width=1\textwidth}	
		\begin{tabular}{|p{4cm} p{2.1cm} p{3cm} p{3.1cm}|}
			\hline
			\multicolumn{2}{|l}{\textbf{Variabel:}}&\multicolumn{2}{l|}{\textbf{}}\\
			String&regis\_no&&\\
			Profession&id\_prof&&\\
			Insurance&insurance&&\\
			String&insurance\_type&&\\
			String&insurance\_no&&\\
			String&insurance\_prg&&\\
			\hline
		\end{tabular}
	\end{adjustbox}
\end{table}
\begin{table}[H]
	\caption{Daftar attribut pada \textit{Class} Insurance}
	\centering
	\small
	\begin{adjustbox}{width=1\textwidth}	
		\begin{tabular}{|p{5cm} p{3.1cm} p{2cm} p{2.1cm}|}
			\hline
			\multicolumn{2}{|l}{\textbf{Variabel:}}&\multicolumn{2}{l|}{\textbf{}}\\
			List$<$InsuranceBridgeConf$>$&insuranceBrigdeConfList&&\\
			int&systemid&&\\
			String&insurance&&\\
			String&memo&&\\
			boolean&active&&\\
			boolean&sysbuiltin&&\\
			\hline
		\end{tabular}
	\end{adjustbox}
\end{table}
\begin{table}[H]
	\caption{Daftar attribut pada \textit{Class} InsuranceBridgeConf}
	\centering
	\small
	\begin{adjustbox}{width=1\textwidth}	
		\begin{tabular}{|p{5cm} p{3.1cm} p{2cm} p{2.1cm}|}
			\hline
			\multicolumn{2}{|l}{\textbf{Variabel:}}&\multicolumn{2}{l|}{\textbf{}}\\
			Long&systemid&&\\
			String&field&&\\
			String&def\_val&&\\
			String&memo&&\\
			\hline
		\end{tabular}
	\end{adjustbox}
\end{table}
\begin{table}[H]
	\caption{Daftar attribut pada \textit{Class} Profession}
	\centering
	\small
	\begin{adjustbox}{width=1\textwidth}	
		\begin{tabular}{|p{5cm} p{3.1cm} p{2cm} p{2.1cm}|}
			\hline
			\multicolumn{2}{|l}{\textbf{Variabel:}}&\multicolumn{2}{l|}{\textbf{}}\\
			Integer&systemid&&\\
			String&profname&&\\
			String&memo&&\\
			Calendar&sys\_createdate&&\\
			Calendar&last\_createdate&&\\
			\hline
		\end{tabular}
	\end{adjustbox}
\end{table}
\begin{table}[H]
	\caption{Daftar attribut pada \textit{Class} Province}
	\centering
	\small
	\begin{adjustbox}{width=1\textwidth}	
		\begin{tabular}{|p{5cm} p{3.1cm} p{2cm} p{2.1cm}|}
			\hline
			\multicolumn{2}{|l}{\textbf{Variabel:}}&\multicolumn{2}{l|}{\textbf{}}\\
			Long&systemid&&\\
			Country&countryCode&&\\
			String&name&&\\
			Set$<$Regency$>$&regencySet&&\\
			\hline
		\end{tabular}
	\end{adjustbox}
\end{table}
\begin{table}[H]
	\caption{Daftar attribut pada \textit{Class} Regency}
	\centering
	\small
	\begin{adjustbox}{width=1\textwidth}	
		\begin{tabular}{|p{5cm} p{3.1cm} p{2cm} p{2.1cm}|}
			\hline
			\multicolumn{2}{|l}{\textbf{Variabel:}}&\multicolumn{2}{l|}{\textbf{}}\\
			Long&systemid&&\\
			Province&provId&&\\
			String&name&&\\
			\hline
		\end{tabular}
	\end{adjustbox}
\end{table}
\subsubsection{\textit{Package} Repository}
\textit{Package} repository berisi \textit{class-class} yang menjadi representasi model dari tabel-tabel yang berada pada basis data. Berikut ini merupakan \textit{class-class} yang terdapat pada \textit{package} repository.
\begin{table}[H]
	\small
	\centering
	\caption{Daftar {\itshape Class} pada {\itshape Package} repository}
	\begin{adjustbox}{width=1\textwidth}
		\begin{tabular}{| p {3 cm} | p {8 cm} | p {3 cm} |}
			\hline
			{\bfseries Package} & {\bfseries Class} & {\bfseries Jenis Class} \\
			\hline
			\multirow{2}{*}{repository} & CustomerRepository & {\itshape Interface} \\
			& CustomergroupRepository & {\itshape Interface} \\
			\hline
		\end{tabular}
	\end{adjustbox}
\end{table}
\subsubsection{\textit{Package} Service}
\textit{Package} service berisi \textit{class-class} yang menginisialisasi \textit{web service} customer yang akan terbentuk dan digunakan pada \textit{package} controller. Berikut ini merupakan \textit{class-class} yang terdapat pada \textit{package} service.
\begin{table}[H]
	\small
	\centering
	\caption{Daftar {\itshape Class} pada {\itshape Package} serviec}
	\begin{adjustbox}{width=1\textwidth}
		\begin{tabular}{| p {3 cm} | p {8 cm} | p {3 cm} |}
			\hline
			{\bfseries Package} & {\bfseries Class} & {\bfseries Jenis Class} \\
			\hline
			\multirow{5}{*}{service} & CRUDService & {\itshape Interface} \\
			& CustomergroupService & {\itshape Interface} \\
			& CustomerService & {\itshape Interface} \\
			& DefaultCustomergroupService & {\itshape Class} \\
			& DefaultCustomerService & {\itshape Class} \\
			\hline
		\end{tabular}
	\end{adjustbox}
\end{table}
\subsection{\textit{Project} \textit{Human Resource}}
\textit{Project Human Resource} berisi susunan \textit{class} pembentuk \textit{webservice Human Resource}. Dalam project ini terdapat 4 buah \textit{package} yang berfungsi untuk memisahkan \textit{class} sesuai dengan fungsinya masing-masing. Ke empat \textit{package} ini yaitu \textit{package} controller yang berisi method-method yang dijalankan pada \textit{web service}, \textit{package} model yang berisi entitas dan tabel, \textit{package} repository yang menghubungkan entitas di \textit{package} model dengan database, dan \textit{package} service yang menginisialisasi \textit{webserviec} yang akan terbentuk. Pada bagian ini akan dijelaskan mengenai \textit{class} dan \textit{method} yang digunakan pada pengembangan \textit{ webservice customer}:
\subsubsection{\textit{Package} Controller}
\textit{Package} controller berisi method API yang dapat digunakan untuk berkomunikasi dengan \textit{service} Human Resource. \textit{Class} yang terdapat dalam \textit{package} ini adalah \textit{class} EmployeeController dan JobSpecialityController yang berisi 5 buah fungsi API. Di bawah ini merupakan daftar \textit{class} untuk \textit{package} controller pada \textit{webservice} \textit{Human Resource}.
\begin{table}[H]
	\small
	\centering
	\caption{Daftar {\itshape Class} pada {\itshape Package} Controller}
	\begin{adjustbox}{width=1\textwidth}
		\begin{tabular}{| p {3 cm} | p {8 cm} | p {3 cm} |}
			\hline
			{\bfseries Package} & {\bfseries Class} & {\bfseries Jenis Class} \\
			\hline
			\multirow{2}{*}{Controller} & EmployeeController & {\itshape Class} \\
			& JobspecialityController & {\itshape Class} \\
			\hline
		\end{tabular}
	\end{adjustbox}
\end{table}
\begin{table}[H]
	\caption{Daftar \textit{Method} pada \textit{Class} EmployeeController}
	\centering
	\small
	\begin{adjustbox}{width=1\textwidth}	
		\begin{tabular}{|p{0.4cm}|p{2.5cm}|p{0.9cm}|p{1.5cm}|p{3.4cm}|p{2.5cm}|}
			\hline
			\multirow{2}{*}{\textbf{No}} & \multirow{2}{*}{\textit{\textbf{Method}}} & \multicolumn{2}{c|}{\textit{\textbf{Input}}} & \multirow{2}{*}{\textit{\textbf{Output}}} & 
			\multirow{2}{*}{\textbf{Keterangan}}\\
			\cline{3-4}
			& & \textbf{Tipe} & \textbf{Variabel} & & \\
			\hline
			1 & addEmployee & void & Employee & Employee, HttpStatus & \textit{Method} ini digunakan untuk membuat \textit{object} employee yang baru melalui \textit{webservice}\\
			\hline
			2 & updateEmployee & void & Employee & HttpStatus & \textit{Method} ini digunakan untuk mengubah attribut dari \textit{object} employee yang baru melalui \textit{webservice}\\
			\hline
			3 & getEmployee & void & id & Employee, HttpStatus & \textit{Method} ini digunakan untuk mengambil satu \textit{object} employee yang baru melalui \textit{webservice}\\
			\hline
			4 & getAllEmployees & void & - & $<$List$<$Employee$>$$>$, HttpStatus & \textit{Method} ini digunakan untuk mengambil semua \textit{list} \textit{object} employee yang baru melalui \textit{webservice}\\
			\hline
		\end{tabular}
	\end{adjustbox}
\end{table}
\begin{table}[H]
	\centering
	\small
	\begin{adjustbox}{width=1\textwidth}	
		\begin{tabular}{|p{0.4cm}|p{2.5cm}|p{0.9cm}|p{1.5cm}|p{3.4cm}|p{2.5cm}|}
			\hline
			5 & deleteEmployee & void & id & HttpStatus & \textit{Method} ini digunakan untuk menghapus satu \textit{object} employee yang baru melalui \textit{webservice}\\
			\hline
		\end{tabular}
	\end{adjustbox}
\end{table}
\begin{table}[H]
	\caption{Daftar \textit{Method} pada \textit{Class} JobspecialityController}
	\centering
	\small
	\begin{adjustbox}{width=1\textwidth}	
		\begin{tabular}{|p{0.4cm}|p{3.2cm}|p{0.8cm}|p{1.9cm}|p{2.85cm}|p{2.4cm}|}
			\hline
			\multirow{2}{*}{\textbf{No}} & \multirow{2}{*}{\textit{\textbf{Method}}} & \multicolumn{2}{c|}{\textit{\textbf{Input}}} & \multirow{2}{*}{\textit{\textbf{Output}}} & 
			\multirow{2}{*}{\textbf{Keterangan}}\\
			\cline{3-4}
			& & \textbf{Tipe} & \textbf{Variabel} & & \\
			\hline
			1 & addJobSpeciality & void & Jobspeciality & Jobspeciality, HttpStatus & \textit{Method} ini digunakan untuk membuat \textit{object} Jobspeciality yang baru melalui \textit{webservice}\\
			\hline
			2 & updateJobSpeciality & void & Jobspeciality & HttpStatus & \textit{Method} ini digunakan untuk mengubah attribut dari \textit{object} Jobspeciality yang baru melalui \textit{webservice}\\
			\hline
			3 & getJobSpeciality & void & id & Jobspeciality, HttpStatus & \textit{Method} ini digunakan untuk mengambil satu \textit{object} Jobspeciality yang baru melalui \textit{webservice}\\
			\hline
		\end{tabular}
	\end{adjustbox}
\end{table}
\begin{table}[H]
	\centering
	\small
	\begin{adjustbox}{width=1\textwidth}	
		\begin{tabular}{|p{0.4cm}|p{3.2cm}|p{1.cm}|p{1.7cm}|p{2.75cm}|p{2.5cm}|}
			\hline
			4 & getAllJobsSpeciality & void & - & $<$List
			$<$Jobspeciality$>$$>$, HttpStatus & \textit{Method} ini digunakan untuk mengambil semua \textit{list} \textit{object} Jobspeciality yang baru melalui \textit{webservice}\\
			\hline
			5 & deleteJobSpeciality & void & id & Jobspeciality, HttpStatus & \textit{Method} ini digunakan untuk menghapus satu \textit{object} CustomerGroup yang baru melalui \textit{webservice}\\
			\hline
		\end{tabular}
	\end{adjustbox}
\end{table}
\subsubsection{\textit{Package} Model Human Resource}
\textit{Package} Human Resource model berisi entitas-entitas penyusun dari \textit{service} Human Resource. Berikut ini merupakan daftar \textit{class} untuk \textit{package} model.
\begin{table}[H]
	\small
	\centering
	\caption{Daftar {\itshape Class} pada {\itshape Package} model}
	\begin{adjustbox}{width=1\textwidth}
		\begin{tabular}{| p {3 cm} | p {8 cm} | p {3 cm} |}
			\hline
			{\bfseries Package} & {\bfseries Class} & {\bfseries Jenis Class} \\
			\hline
			\multirow{12}{*}{Model} & Regency & {\itshape Class} \\
			& AddressType & {\itshape Class} \\
			& Contact & {\itshape Class} \\
			& ContactAddress & {\itshape Class} \\
			& ContactEducation & {\itshape Class} \\
			& Country & {\itshape Class} \\
			& Department & {\itshape Class} \\
			& Employee & {\itshape Class} \\
			& Employment & {\itshape Class} \\
			& Jobspeciality & {\itshape Class} \\
			& Province & {\itshape Class} \\
			& RoleInDept & {\itshape Class} \\
			\hline
		\end{tabular}
	\end{adjustbox}
\end{table}
\begin{table}[H]
	\caption{Daftar attribut pada \textit{Class} Employment}
	\centering
	\small
	\begin{adjustbox}{width=1\textwidth}	
		\begin{tabular}{|p{4cm} p{2.1cm} p{3cm} p{3.1cm}|}
			\hline
			\multicolumn{2}{|l}{\textbf{Variabel:}}&\multicolumn{2}{l|}{\textbf{Variabel:}}\\
			Long&systemid&Employee&employee\\
			String&ein&Date&hiredate\\
			BigInteger&basesalary&RoleInDept&roleIndept\\
			Integer&contracttype&String& sourceinstitution\\
			int&sourcetype&Date& createdate\\
			Date&lastupdate&Date& terminatedate\\
			\hline
		\end{tabular}
	\end{adjustbox}
\end{table}
\begin{table}[H]
	\caption{Daftar attribut pada \textit{Class} Employee}
	\centering
	\small
	\begin{adjustbox}{width=1\textwidth}	
		\begin{tabular}{|p{4cm} p{2.1cm} p{3cm} p{3.1cm}|}
			\hline
			\multicolumn{2}{|l}{\textbf{Variabel:}}&\multicolumn{2}{l|}{}\\
			String&m\_ein&&\\
			Jobspeciality&jobspeciality&&\\
			Collection$<$Employment$>$&employmentCollection&&\\
			\hline
		\end{tabular}
	\end{adjustbox}
\end{table}
\begin{table}[H]
	\caption{Daftar attribut pada \textit{Class} Departement}
	\centering
	\small
	\begin{adjustbox}{width=1\textwidth}	
		\begin{tabular}{|p{4cm} p{2.1cm} p{3cm} p{3.1cm}|}
			\hline
			\multicolumn{2}{|l}{\textbf{Variabel:}}&\multicolumn{2}{l|}{\textbf{}}\\
			int&systemid&&\\
			String&deptname&&\\
			String&memo&&\\
			Collection$<$RoleInDept$>$&roles&&\\
			Date&createdate&&\\
			Date&lastupdate&&\\
			\hline
		\end{tabular}
	\end{adjustbox}
\end{table}
\begin{table}[H]
	\caption{Daftar attribut pada \textit{Class} Contact}
	\centering
	\small
	\begin{adjustbox}{width=1\textwidth}	
		\begin{tabular}{|p{2cm} p{2.1cm} p{5cm} p{3.1cm}|}
			\hline
			\multicolumn{2}{|l}{\textbf{Variabel:}}&\multicolumn{2}{l|}{\textbf{Variabel:}}\\
			long&systemid&String&initial\\
			String&lastname&int&amountofchildren\\
			int&maritalstatus&List$<$ContactAddress$>$&arrAddress\\
			String&middlename&Collection$<$ContactEducation$>$&arrEdu\\
			Date&birthday&String&notes\\
			String&birthplace&String&officeext\\
			String&bloodtype&String&passportid\\
			double&bodyheight&byte[]&photo\\
			double&bodyweight&String&prefixtitle\\
			String&citizenid&String&sms\\
			String&citizentype&String&suffixtitle\\
			Country&citizenship&Date&sys\_createdate\\
			String&email&String&sys\_creator\\
			String&firstname&Date&sys\_lastupdate\\
			int&gender&String&sys\_lastupdater\\
			String&homephone&String&taxid\\
			String&messaging&String&web\\
			\hline
		\end{tabular}
	\end{adjustbox}
\end{table}
\begin{table}[H]
	\caption{Daftar attribut pada \textit{Class} ContactAddress}
	\centering
	\small
	\begin{adjustbox}{width=1\textwidth}	
		\begin{tabular}{|p{4cm} p{2.1cm} p{3cm} p{3.1cm}|}
			\hline
			\multicolumn{2}{|l}{\textbf{Variabel:}}&\multicolumn{2}{l|}{\textbf{Variabel:}}\\
			Long&systemid&String&postcode\\
			AddressType&addresstype&Regency&regency\\
			String&street&boolean&asbillingaddress\\
			String&fax&Date&sys\_createdate\\
			Contact&owner&String&sys\_creator\\
			String&phone&Date&sys\_lastupdate\\
			String&sys\_lastupdater&&\\
			\hline
		\end{tabular}
	\end{adjustbox}
\end{table}
\begin{table}[H]
	\caption{Daftar attribut pada \textit{Class} AddressType}
	\centering
	\small
	\begin{adjustbox}{width=1\textwidth}	
		\begin{tabular}{|p{4cm} p{2.1cm} p{3cm} p{3.1cm}|}
			\hline
			\multicolumn{2}{|l}{\textbf{Variabel:}}&\multicolumn{2}{l|}{\textbf{}}\\
			Integer&systemid&&\\
			String&memo&&\\
			String&typename&&\\
			\hline
		\end{tabular}
	\end{adjustbox}
\end{table}
\begin{table}[H]
	\caption{Daftar attribut pada \textit{Class} ContactEducation}
	\centering
	\small
	\begin{adjustbox}{width=1\textwidth}	
		\begin{tabular}{|p{3cm} p{3.1cm} p{3cm} p{3.1cm}|}
			\hline
			\multicolumn{2}{|l}{\textbf{Variabel:}}&\multicolumn{2}{l|}{\textbf{\textbf{Variabel:}}}\\
			Date&finishrolling&Contact&owner\\
			String&gpa&String&sponsors\\
			int&grade&Date&startrolling\\
			String&institutionaladdress&Date&sys\_createdate\\
			String&institutionname&int&sys\_creator\\
			String&major&Date&sys\_lastupdate\\
			String&notes&int&sys\_lastupdater\\
			\hline
		\end{tabular}
	\end{adjustbox}
\end{table}
\begin{table}[H]
	\caption{Daftar attribut pada \textit{Class} Country}
	\centering
	\small
	\begin{adjustbox}{width=1\textwidth}	
		\begin{tabular}{|p{4cm} p{2.1cm} p{3cm} p{3.1cm}|}
			\hline
			\multicolumn{2}{|l}{\textbf{Variabel:}}&\multicolumn{2}{l|}{\textbf{}}\\
			Long&systemid&&\\
			String&name&&\\
			\hline
		\end{tabular}
	\end{adjustbox}
\end{table}
\begin{table}[H]
	\caption{Daftar attribut pada \textit{Class} Jobspeciality}
	\centering
	\small
	\begin{adjustbox}{width=1\textwidth}	
		\begin{tabular}{|p{5cm} p{3.1cm} p{2cm} p{2.1cm}|}
			\hline
			\multicolumn{2}{|l}{\textbf{Variabel:}}&\multicolumn{2}{l|}{\textbf{}}\\
			Long&systemid&&\\
			String&specialityName&&\\
			String&memo&&\\
			\hline
		\end{tabular}
	\end{adjustbox}
\end{table}
\begin{table}[H]
	\caption{Daftar attribut pada \textit{Class} RoleInDept}
	\centering
	\small
	\begin{adjustbox}{width=1\textwidth}	
		\begin{tabular}{|p{5cm} p{3.1cm} p{2cm} p{2.1cm}|}
			\hline
			\multicolumn{2}{|l}{\textbf{Variabel:}}&\multicolumn{2}{l|}{\textbf{}}\\
			int&systemid&&\\
			Department&department&&\\
			String&rolename&&\\
			String&abbreviation&&\\
			RoleInDept&parentRole&&\\
			String&memo&&\\
			\hline
		\end{tabular}
	\end{adjustbox}
\end{table}
\begin{table}[H]
	\caption{Daftar attribut pada \textit{Class} Province}
	\centering
	\small
	\begin{adjustbox}{width=1\textwidth}	
		\begin{tabular}{|p{5cm} p{3.1cm} p{2cm} p{2.1cm}|}
			\hline
			\multicolumn{2}{|l}{\textbf{Variabel:}}&\multicolumn{2}{l|}{\textbf{}}\\
			Long&systemid&&\\
			Country&countryCode&&\\
			String&name&&\\
			Set$<$Regency$>$&regencySet&&\\
			\hline
		\end{tabular}
	\end{adjustbox}
\end{table}
\begin{table}[H]
	\caption{Daftar attribut pada \textit{Class} Regency}
	\centering
	\small
	\begin{adjustbox}{width=1\textwidth}	
		\begin{tabular}{|p{5cm} p{3.1cm} p{2cm} p{2.1cm}|}
			\hline
			\multicolumn{2}{|l}{\textbf{Variabel:}}&\multicolumn{2}{l|}{\textbf{}}\\
			Long&systemid&&\\
			Province&provId&&\\
			String&name&&\\
			\hline
		\end{tabular}
	\end{adjustbox}
\end{table}
\subsubsection{\textit{Package} Repository}
\textit{Package} repository berisi \textit{class-class} yang menjadi representasi model dari tabel-tabel yang berada pada basis data. Berikut ini merupakan \textit{class-class} yang terdapat pada \textit{package} repository.
\begin{table}[H]
	\small
	\centering
	\caption{Daftar {\itshape Class} pada {\itshape Package} repository}
	\begin{adjustbox}{width=1\textwidth}
		\begin{tabular}{| p {3 cm} | p {8 cm} | p {3 cm} |}
			\hline
			{\bfseries Package} & {\bfseries Class} & {\bfseries Jenis Class} \\
			\hline
			\multirow{2}{*}{repository} & EmployeeRepository & {\itshape Interface} \\
			& JobspecialityRepository & {\itshape Interface} \\
			\hline
		\end{tabular}
	\end{adjustbox}
\end{table}
\subsubsection{\textit{Package} Service}
\textit{Package} service berisi \textit{class-class} yang menginisialisasi \textit{web service} human resource yang akan terbentuk dan digunakan pada \textit{package} controller. Berikut ini merupakan \textit{class-class} yang terdapat pada \textit{package} service.
\begin{table}[H]
	\small
	\centering
	\caption{Daftar {\itshape Class} pada {\itshape Package} service}
	\begin{adjustbox}{width=1\textwidth}
		\begin{tabular}{| p {3 cm} | p {8 cm} | p {3 cm} |}
			\hline
			{\bfseries Package} & {\bfseries Class} & {\bfseries Jenis Class} \\
			\hline
			\multirow{5}{*}{service} & CRUDService & {\itshape Interface} \\
			& EmployeeService & {\itshape Interface} \\
			& JobspecialityService & {\itshape Interface} \\
			& DefaultEmployeeService & {\itshape Class} \\
			& DefaultJobspecialityService & {\itshape Class} \\
			\hline
		\end{tabular}
	\end{adjustbox}
\end{table}
\subsection{\textit{Project} \textit{Medical Unit}}
\textit{Project Medical Unit} berisi susunan \textit{class} pembentuk \textit{webservice Medical Unit}. Dalam project ini terdapat 4 buah \textit{package} yang berfungsi untuk memisahkan \textit{class} sesuai dengan fungsinya masing-masing. Ke empat \textit{package} ini yaitu \textit{package} controller yang berisi method-method yang dijalankan pada \textit{web service}, \textit{package} model yang berisi entitas dan tabel, \textit{package} repository yang menghubungkan entitas di \textit{package} model dengan database, dan \textit{package} service yang menginisialisasi \textit{webserviec} yang akan terbentuk. Pada bagian ini akan dijelaskan mengenai \textit{class} dan \textit{method} yang digunakan pada pengembangan \textit{ webservice Unit Medis}:
\subsubsection{\textit{Package} Controller}
\textit{Package} controller berisi method API yang dapat digunakan untuk berkomunikasi dengan \textit{service} Human Resource. \textit{Class} yang terdapat dalam \textit{package} ini adalah \textit{class} UnitmedisController yang berisi 5 buah fungsi API. Di bawah ini merupakan daftar \textit{class} untuk \textit{package} controller pada \textit{webservice} Unit Medis.
\begin{table}[H]
	\small
	\centering
	\caption{Daftar {\itshape Class} pada {\itshape Package} Controller}
	\begin{adjustbox}{width=1\textwidth}
		\begin{tabular}{| p {3 cm} | p {8 cm} | p {3 cm} |}
			\hline
			{\bfseries Package} & {\bfseries Class} & {\bfseries Jenis Class} \\
			\hline
			\multirow{1}{*}{Controller} & UnitmedisController & {\itshape Class} \\
			\hline
		\end{tabular}
	\end{adjustbox}
\end{table}
\begin{table}[H]
	\caption{Daftar \textit{Method} pada \textit{Class} UnitmedisController}
	\centering
	\small
	\begin{adjustbox}{width=1\textwidth}	
		\begin{tabular}{|p{0.4cm}|p{2.8cm}|p{0.9cm}|p{1.8cm}|p{2.8cm}|p{2.5cm}|}
			\hline
			\multirow{2}{*}{\textbf{No}} & \multirow{2}{*}{\textit{\textbf{Method}}} & \multicolumn{2}{c|}{\textit{\textbf{Input}}} & \multirow{2}{*}{\textit{\textbf{Output}}} & 
			\multirow{2}{*}{\textbf{Keterangan}}\\
			\cline{3-4}
			& & \textbf{Tipe} & \textbf{Variabel} & & \\
			\hline
			1 & addMedicalUnit & void & MedicalUnit & MedicalUnit, HttpStatus & \textit{Method} ini digunakan untuk membuat \textit{object} unit medis yang baru melalui \textit{webservice}\\
			\hline
			2 & updateMedicalUnit & void & MedicalUnit & HttpStatus & \textit{Method} ini digunakan untuk mengubah attribut dari \textit{object} unit medis yang baru melalui \textit{webservice}\\
			\hline
			3 & getMedicalUnit & void & id & MedicalUnit, HttpStatus & \textit{Method} ini digunakan untuk mengambil satu \textit{object} unit medis yang baru melalui \textit{webservice}\\
			\hline
			4 & getAllMedicalUnit & void & - & $<$List
			$<$MedicalUnit$>$$>$, HttpStatus & \textit{Method} ini digunakan untuk mengambil semua \textit{list} \textit{object} unit medis yang baru melalui \textit{webservice}\\
			\hline
			5 & deleteMedicalUnit & void & id & HttpStatus & \textit{Method} ini digunakan untuk menghapus satu \textit{object} unit medis yang baru melalui \textit{webservice}\\
			\hline
		\end{tabular}
	\end{adjustbox}
\end{table}
\subsubsection{\textit{Package} Model Medical Unit}
\textit{Package} Medical Unit model berisi entitas-entitas penyusun dari \textit{service} Medical Unit. Berikut ini merupakan daftar \textit{class} untuk \textit{package} model.
\begin{table}[H]
	\small
	\centering
	\caption{Daftar {\itshape Class} pada {\itshape Package} model}
	\begin{adjustbox}{width=1\textwidth}
		\begin{tabular}{| p {3 cm} | p {8 cm} | p {3 cm} |}
			\hline
			{\bfseries Package} & {\bfseries Class} & {\bfseries Jenis Class} \\
			\hline
			\multirow{16}{*}{Model} & MedicalUnit & {\itshape Class} \\
			& WorkingUnit & {\itshape Class} \\
			& WorkingUnitMembership & {\itshape Class} \\
			& WorkingUnitMembershipPK & {\itshape Class} \\
			& WUGroup & {\itshape Class} \\
			& AddressType & {\itshape Class} \\
			& Contact & {\itshape Class} \\
			& ContactAddress & {\itshape Class} \\
			& ContactEducation & {\itshape Class} \\
			& Country & {\itshape Class} \\
			& Department & {\itshape Class} \\
			& Employee & {\itshape Class} \\
			& Employment & {\itshape Class} \\
			& Jobspeciality & {\itshape Class} \\
			& Province & {\itshape Class} \\
			& RoleInDept & {\itshape Class} \\
			& Regency & {\itshape Class} \\
			\hline
		\end{tabular}
	\end{adjustbox}
\end{table}
\begin{table}[H]
	\caption{Daftar attribut pada \textit{Class} WorkingUnit}
	\centering
	\small
	\begin{adjustbox}{width=1\textwidth}	
		\begin{tabular}{|p{4cm} p{2.1cm} p{3cm} p{3.1cm}|}
			\hline
			\multicolumn{2}{|l}{\textbf{Variabel:}}&\multicolumn{2}{l|}{\textbf{Variabel:}}\\
			Long&systemid&&\\
			WUGroup&wugroup&&\\
			String&workunit\_name&&\\
			String&memo&&\\
			Calendar&createdate&&\\
			Calendar&lastupdate&&\\
			\hline
		\end{tabular}
	\end{adjustbox}
\end{table}
\begin{table}[H]
	\caption{Daftar attribut pada \textit{Class} MedicalUnit}
	\centering
	\small
	\begin{adjustbox}{width=1\textwidth}	
		\begin{tabular}{|p{4cm} p{2.1cm} p{3cm} p{3.1cm}|}
			\hline
			\multicolumn{2}{|l}{\textbf{Variabel:}}&\multicolumn{2}{l|}{}\\
			Long&systemid&&\\
			\hline
		\end{tabular}
	\end{adjustbox}
\end{table}
\begin{table}[H]
	\caption{Daftar attribut pada \textit{Class} WorkingUnitMembership}
	\centering
	\small
	\begin{adjustbox}{width=1\textwidth}	
		\begin{tabular}{|p{4cm} p{2.1cm} p{3cm} p{3.1cm}|}
			\hline
			\multicolumn{2}{|l}{\textbf{Variabel:}}&\multicolumn{2}{l|}{}\\
			WorkingUnit&workingunit&&\\
			Employee&employee&&\\
			boolean&workingunitpersonincharge&&\\
			String&description&&\\
			Calendar&sys\_createdate&&\\
			Calendar&syssys\_lastupdate&&\\
			\hline
		\end{tabular}
	\end{adjustbox}
\end{table}
\begin{table}[H]
	\caption{Daftar attribut pada \textit{Class} WUGroup}
	\centering
	\small
	\begin{adjustbox}{width=1\textwidth}	
		\begin{tabular}{|p{4cm} p{2.1cm} p{3cm} p{3.1cm}|}
			\hline
			\multicolumn{2}{|l}{\textbf{Variabel:}}&\multicolumn{2}{l|}{}\\
			Long&systemid&&\\
			String&groupname&&\\
			String&memo&&\\
			\hline
		\end{tabular}
	\end{adjustbox}
\end{table}
\begin{table}[H]
	\caption{Daftar attribut pada \textit{Class} WorkingUnitMembershipPK}
	\centering
	\small
	\begin{adjustbox}{width=1\textwidth}	
		\begin{tabular}{|p{4cm} p{2.1cm} p{3cm} p{3.1cm}|}
			\hline
			\multicolumn{2}{|l}{\textbf{Variabel:}}&\multicolumn{2}{l|}{}\\
			Long&workingunit&&\\
			Long&employee&&\\
			\hline
		\end{tabular}
	\end{adjustbox}
\end{table}
\begin{table}[H]
	\caption{Daftar attribut pada \textit{Class} Employment}
	\centering
	\small
	\begin{adjustbox}{width=1\textwidth}	
		\begin{tabular}{|p{4cm} p{2.1cm} p{3cm} p{3.1cm}|}
			\hline
			\multicolumn{2}{|l}{\textbf{Variabel:}}&\multicolumn{2}{l|}{\textbf{Variabel:}}\\
			Long&systemid&Employee&employee\\
			String&ein&Date&hiredate\\
			BigInteger&basesalary&RoleInDept&roleIndept\\
			Integer&contracttype&String& sourceinstitution\\
			int&sourcetype&Date& createdate\\
			Date&lastupdate&Date& terminatedate\\
			\hline
		\end{tabular}
	\end{adjustbox}
\end{table}
\begin{table}[H]
	\caption{Daftar attribut pada \textit{Class} Employee}
	\centering
	\small
	\begin{adjustbox}{width=1\textwidth}	
		\begin{tabular}{|p{4cm} p{2.1cm} p{3cm} p{3.1cm}|}
			\hline
			\multicolumn{2}{|l}{\textbf{Variabel:}}&\multicolumn{2}{l|}{}\\
			String&m\_ein&&\\
			Jobspeciality&jobspeciality&&\\
			Collection$<$Employment$>$&employmentCollection&&\\
			\hline
		\end{tabular}
	\end{adjustbox}
\end{table}
\begin{table}[H]
	\caption{Daftar attribut pada \textit{Class} Departement}
	\centering
	\small
	\begin{adjustbox}{width=1\textwidth}	
		\begin{tabular}{|p{4cm} p{2.1cm} p{3cm} p{3.1cm}|}
			\hline
			\multicolumn{2}{|l}{\textbf{Variabel:}}&\multicolumn{2}{l|}{\textbf{}}\\
			int&systemid&&\\
			String&deptname&&\\
			String&memo&&\\
			Collection$<$RoleInDept$>$&roles&&\\
			Date&createdate&&\\
			Date&lastupdate&&\\
			\hline
		\end{tabular}
	\end{adjustbox}
\end{table}
\begin{table}[H]
\caption{Daftar attribut pada \textit{Class} ContactAddress}
\centering
\small
\begin{adjustbox}{width=1\textwidth}	
	\begin{tabular}{|p{4cm} p{2.1cm} p{3cm} p{3.1cm}|}
		\hline
		\multicolumn{2}{|l}{\textbf{Variabel:}}&\multicolumn{2}{l|}{\textbf{Variabel:}}\\
		Long&systemid&String&postcode\\
		AddressType&addresstype&Regency&regency\\
		String&street&boolean&asbillingaddress\\
		String&fax&Date&sys\_createdate\\
		Contact&owner&String&sys\_creator\\
		String&phone&Date&sys\_lastupdate\\
		String&sys\_lastupdater&&\\
		\hline
	\end{tabular}
\end{adjustbox}
\end{table}
\begin{table}[H]
	\caption{Daftar attribut pada \textit{Class} Contact}
	\centering
	\small
	\begin{adjustbox}{width=1\textwidth}	
		\begin{tabular}{|p{2cm} p{2.1cm} p{5cm} p{3.1cm}|}
			\hline
			\multicolumn{2}{|l}{\textbf{Variabel:}}&\multicolumn{2}{l|}{\textbf{Variabel:}}\\
			long&systemid&String&initial\\
			String&lastname&int&amountofchildren\\
			int&maritalstatus&List$<$ContactAddress$>$&arrAddress\\
			String&middlename&Collection$<$ContactEducation$>$&arrEdu\\
			Date&birthday&String&notes\\
			String&birthplace&String&officeext\\
			String&bloodtype&String&passportid\\
			double&bodyheight&byte[]&photo\\
			double&bodyweight&String&prefixtitle\\
			String&citizenid&String&sms\\
			String&citizentype&String&suffixtitle\\
			Country&citizenship&Date&sys\_createdate\\
			String&email&String&sys\_creator\\
			String&firstname&Date&sys\_lastupdate\\
			int&gender&String&sys\_lastupdater\\
			String&homephone&String&taxid\\
			String&messaging&String&web\\
			\hline
		\end{tabular}
	\end{adjustbox}
\end{table}
\begin{table}[H]
	\caption{Daftar attribut pada \textit{Class} AddressType}
	\centering
	\small
	\begin{adjustbox}{width=1\textwidth}	
		\begin{tabular}{|p{4cm} p{2.1cm} p{3cm} p{3.1cm}|}
			\hline
			\multicolumn{2}{|l}{\textbf{Variabel:}}&\multicolumn{2}{l|}{\textbf{}}\\
			Integer&systemid&&\\
			String&memo&&\\
			String&typename&&\\
			\hline
		\end{tabular}
	\end{adjustbox}
\end{table}
\begin{table}[H]
	\caption{Daftar attribut pada \textit{Class} ContactEducation}
	\centering
	\small
	\begin{adjustbox}{width=1\textwidth}	
		\begin{tabular}{|p{3cm} p{3.1cm} p{3cm} p{3.1cm}|}
			\hline
			\multicolumn{2}{|l}{\textbf{Variabel:}}&\multicolumn{2}{l|}{\textbf{\textbf{Variabel:}}}\\
			Date&finishrolling&Contact&owner\\
			String&gpa&String&sponsors\\
			int&grade&Date&startrolling\\
			String&institutionaladdress&Date&sys\_createdate\\
			String&institutionname&int&sys\_creator\\
			String&major&Date&sys\_lastupdate\\
			String&notes&int&sys\_lastupdater\\
			\hline
		\end{tabular}
	\end{adjustbox}
\end{table}
\begin{table}[H]
	\caption{Daftar attribut pada \textit{Class} Country}
	\centering
	\small
	\begin{adjustbox}{width=1\textwidth}	
		\begin{tabular}{|p{4cm} p{2.1cm} p{3cm} p{3.1cm}|}
			\hline
			\multicolumn{2}{|l}{\textbf{Variabel:}}&\multicolumn{2}{l|}{\textbf{}}\\
			Long&systemid&&\\
			String&name&&\\
			\hline
		\end{tabular}
	\end{adjustbox}
\end{table}
\begin{table}[H]
	\caption{Daftar attribut pada \textit{Class} Jobspeciality}
	\centering
	\small
	\begin{adjustbox}{width=1\textwidth}	
		\begin{tabular}{|p{5cm} p{3.1cm} p{2cm} p{2.1cm}|}
			\hline
			\multicolumn{2}{|l}{\textbf{Variabel:}}&\multicolumn{2}{l|}{\textbf{}}\\
			Long&systemid&&\\
			String&specialityName&&\\
			String&memo&&\\
			\hline
		\end{tabular}
	\end{adjustbox}
\end{table}
\begin{table}[H]
	\caption{Daftar attribut pada \textit{Class} RoleInDept}
	\centering
	\small
	\begin{adjustbox}{width=1\textwidth}	
		\begin{tabular}{|p{5cm} p{3.1cm} p{2cm} p{2.1cm}|}
			\hline
			\multicolumn{2}{|l}{\textbf{Variabel:}}&\multicolumn{2}{l|}{\textbf{}}\\
			int&systemid&&\\
			Department&department&&\\
			String&rolename&&\\
			String&abbreviation&&\\
			RoleInDept&parentRole&&\\
			String&memo&&\\
			\hline
		\end{tabular}
	\end{adjustbox}
\end{table}
\begin{table}[H]
	\caption{Daftar attribut pada \textit{Class} Province}
	\centering
	\small
	\begin{adjustbox}{width=1\textwidth}	
		\begin{tabular}{|p{5cm} p{3.1cm} p{2cm} p{2.1cm}|}
			\hline
			\multicolumn{2}{|l}{\textbf{Variabel:}}&\multicolumn{2}{l|}{\textbf{}}\\
			Long&systemid&&\\
			Country&countryCode&&\\
			String&name&&\\
			Set$<$Regency$>$&regencySet&&\\
			\hline
		\end{tabular}
	\end{adjustbox}
\end{table}
\begin{table}[H]
	\caption{Daftar attribut pada \textit{Class} Regency}
	\centering
	\small
	\begin{adjustbox}{width=1\textwidth}	
		\begin{tabular}{|p{5cm} p{3.1cm} p{2cm} p{2.1cm}|}
			\hline
			\multicolumn{2}{|l}{\textbf{Variabel:}}&\multicolumn{2}{l|}{\textbf{}}\\
			Long&systemid&&\\
			Province&provId&&\\
			String&name&&\\
			\hline
		\end{tabular}
	\end{adjustbox}
\end{table}
\subsubsection{\textit{Package} Repository}
\textit{Package} repository berisi \textit{class-class} yang menjadi representasi model dari tabel-tabel yang berada pada basis data. Berikut ini merupakan \textit{class-class} yang terdapat pada \textit{package} repository.
\begin{table}[H]
	\small
	\centering
	\caption{Daftar {\itshape Class} pada {\itshape Package} repository}
	\begin{adjustbox}{width=1\textwidth}
		\begin{tabular}{| p {3 cm} | p {8 cm} | p {3 cm} |}
			\hline
			{\bfseries Package} & {\bfseries Class} & {\bfseries Jenis Class} \\
			\hline
			\multirow{1}{*}{repository} & UnitmedisRepository & {\itshape Interface} \\
			\hline
		\end{tabular}
	\end{adjustbox}
\end{table}
\subsubsection{\textit{Package} Service}
\textit{Package} service berisi \textit{class-class} yang menginisialisasi \textit{web service} human resource yang akan terbentuk dan digunakan pada \textit{package} controller. Berikut ini merupakan \textit{class-class} yang terdapat pada \textit{package} service.
\begin{table}[H]
	\small
	\centering
	\caption{Daftar {\itshape Class} pada {\itshape Package} service}
	\begin{adjustbox}{width=1\textwidth}
		\begin{tabular}{| p {3 cm} | p {8 cm} | p {3 cm} |}
			\hline
			{\bfseries Package} & {\bfseries Class} & {\bfseries Jenis Class} \\
			\hline
			\multirow{3}{*}{service} & CRUDService & {\itshape Interface} \\
			& UnitmedisService & {\itshape Interface} \\
			& DefaultUnitmedisService & {\itshape Class} \\
			\hline
		\end{tabular}
	\end{adjustbox}
\end{table}
\subsection{\textit{Project} Rawat Jalan}
\textit{Project} Rawat Jalan berisi susunan \textit{class} pembentuk \textit{webservice} Rawat Jalan. Dalam project ini terdapat 4 buah \textit{package} yang berfungsi untuk memisahkan \textit{class} sesuai dengan fungsinya masing-masing. Ke empat \textit{package} ini yaitu \textit{package} controller yang berisi method-method yang dijalankan pada \textit{web service}, \textit{package} model yang berisi entitas dan tabel, \textit{package} repository yang menghubungkan entitas di \textit{package} model dengan database, dan \textit{package} service yang menginisialisasi \textit{webserviec} yang akan terbentuk. Pada bagian ini akan dijelaskan mengenai \textit{class} dan \textit{method} yang digunakan pada pengembangan \textit{webservice} Rawat Jalan:
\subsubsection{\textit{Package} Controller}

\textit{Package} controller berisi method API yang dapat digunakan untuk berkomunikasi dengan \textit{service} Rawat Jalan. \textit{Class} yang terdapat dalam \textit{package} ini adalah \textit{class} OutpatientWUQueueController yang berisi 5 buah fungsi API. Di bawah ini merupakan daftar \textit{class} untuk \textit{package} controller pada \textit{webservice} Rawat Jalan.
\begin{table}[H]
	\small
	\centering
	\caption{Daftar {\itshape Class} pada {\itshape Package} Controller}
	\begin{adjustbox}{width=1\textwidth}
		\begin{tabular}{| p {3 cm} | p {8 cm} | p {3 cm} |}
			\hline
			{\bfseries Package} & {\bfseries Class} & {\bfseries Jenis Class} \\
			\hline
			\multirow{1}{*}{Controller} & OutpatientWUQueueController & {\itshape Class} \\
			\hline
		\end{tabular}
	\end{adjustbox}
\end{table}
\begin{table}[H]
	\caption{Daftar \textit{Method} pada \textit{Class} OutpatientWUQueueController}
	\centering
	\small
	\begin{adjustbox}{width=1\textwidth}	
		\begin{tabular}{|p{0.4cm}|p{2.8cm}|p{0.9cm}|p{1.8cm}|p{2.8cm}|p{2.5cm}|}
			\hline
			\multirow{2}{*}{\textbf{No}} & \multirow{2}{*}{\textit{\textbf{Method}}} & \multicolumn{2}{c|}{\textit{\textbf{Input}}} & \multirow{2}{*}{\textit{\textbf{Output}}} & 
			\multirow{2}{*}{\textbf{Keterangan}}\\
			\cline{3-4}
			& & \textbf{Tipe} & \textbf{Variabel} & & \\
			\hline
			1 & addOutpatient
			Queue & void & Outpatient
			WUQueue & Outpatient
			WUQueue, HttpStatus & \textit{Method} ini digunakan untuk membuat \textit{object} rawat jalan yang baru melalui \textit{webservice}\\
			\hline
			2 & updateOutpatient
			Queue & void & Outpatient
			WUQueue & HttpStatus & \textit{Method} ini digunakan untuk mengubah attribut dari \textit{object} rawat jalan yang baru melalui \textit{webservice}\\
			\hline
			3 & getOutpatient
			Queue & void & id & Outpatient
			WUQueue, HttpStatus & \textit{Method} ini digunakan untuk mengambil satu \textit{object} rawat jalan yang baru melalui \textit{webservice}\\
			\hline
			4 & getAllOut
			patientQueue & void & - & $<$List
			$<$OutpatientWU
			Queue$>$$>$, HttpStatus & \textit{Method} ini digunakan untuk mengambil semua \textit{list} \textit{object} rawat jalan yang baru melalui \textit{webservice}\\
			\hline
			5 & deleteOutpatient
			Queue & void & id & HttpStatus & \textit{Method} ini digunakan untuk menghapus satu \textit{object} rawat jalan yang baru melalui \textit{webservice}\\
			\hline
		\end{tabular}
	\end{adjustbox}
\end{table}
\subsubsection{\textit{Package} Model Rawat Jalan}
\textit{Package} Rawat Jalan model berisi entitas-entitas penyusun dari \textit{service} Rawat Jalan. Berikut ini merupakan daftar \textit{class} untuk \textit{package} model.
\begin{table}[H]
	\small
	\centering
	\caption{Daftar {\itshape Class} pada {\itshape Package} model}
	\begin{adjustbox}{width=1\textwidth}
		\begin{tabular}{| p {3 cm} | p {8 cm} | p {3 cm} |}
			\hline
			{\bfseries Package} & {\bfseries Class} & {\bfseries Jenis Class} \\
			\hline
			\multirow{26}{*}{Model} & OutpatientWUQueue & {\itshape Class} \\
			& OutpatientWUQueueHistory & {\itshape Class} \\
			& OutpatientWUQueuePK & {\itshape Class} \\
			& WorkingUnit & {\itshape Class} \\
			& WorkingUnitMembership & {\itshape Class} \\
			& WorkingUnitMembershipPK & {\itshape Class} \\
			& WUGroup & {\itshape Class} \\
			& AddressType & {\itshape Class} \\
			& Contact & {\itshape Class} \\
			& ContactAddress & {\itshape Class} \\
			& ContactEducation & {\itshape Class} \\
			& Customer & {\itshape Class} \\
			& Customergroup & {\itshape Class} \\
			& HealthConsumer & {\itshape Class} \\
			& Insurance & {\itshape Class} \\
			& InsuranceBridgeConf & {\itshape Class} \\
			& Profession & {\itshape Class} \\
			& Country & {\itshape Class} \\
			& Department & {\itshape Class} \\
			& Employee & {\itshape Class} \\
			& Employment & {\itshape Class} \\
			& Jobspeciality & {\itshape Class} \\
			& Province & {\itshape Class} \\
			& RoleInDept & {\itshape Class} \\
			& Regency & {\itshape Class} \\
			& MedicalUnit & {\itshape Class} \\
			\hline
		\end{tabular}
	\end{adjustbox}
\end{table}
\begin{table}[H]
	\caption{Daftar attribut pada \textit{Class} OutpatientWUQueue}
	\centering
	\small
	\begin{adjustbox}{width=1\textwidth}	
		\begin{tabular}{|p{4.5cm} p{2.1cm} p{2.5cm} p{3.1cm}|}
			\hline
			\multicolumn{2}{|l}{\textbf{Variabel:}}&\multicolumn{2}{l|}{\textbf{\textbf{Variabel:}}}\\
			OutpatientWUQueue&hc&Calendar&syscreatetime\\
			MedicalUnit&medunit&Calendar&syslastupdate\\
			String&regis\_no&Insurance&insurance\\
			Date&registertimee&String&insuranceno\\
			Date&finishtime&String&insurancetype\\
			int&priority&String&via\\
			String&memo&String&rujukan\_tipe\\
			String&rujukan\_doc\_no&String&rujukan\_nama\_asal\\
			OutpatientWUQueueHistory&history&String&insuranceprg\\
			\hline
		\end{tabular}
	\end{adjustbox}
\end{table}
\begin{table}[H]
	\caption{Daftar attribut pada \textit{Class} outpatienthistory}
	\centering
	\small
	\begin{adjustbox}{width=1\textwidth}	
		\begin{tabular}{|p{3.5cm} p{3.1cm} p{2.5cm} p{3.1cm}|}
			\hline
			\multicolumn{2}{|l}{\textbf{Variabel:}}&\multicolumn{2}{l|}{\textbf{\textbf{Variabel:}}}\\
			OutpatientWUQueue&hc&Calendar&syscreatetime\\
			MedicalUnit&medunit&Calendar&syslastupdate\\
			String&regis\_no&Insurance&insurance\\
			Date&registertimee&String&insuranceno\\
			Date&finishtime&String&insurancetype\\
			int&priority&String&via\\
			String&memo&String&rujukan\_tipe\\
			String&rujukan\_nama\_asal&String&rujukan\_doc\_no\\
			String&insuranceprg&&\\
			\hline
		\end{tabular}
	\end{adjustbox}
\end{table}
\begin{table}[H]
	\caption{Daftar attribut pada \textit{Class} OutpatientWUQueuePK}
	\centering
	\small
	\begin{adjustbox}{width=1\textwidth}	
		\begin{tabular}{|p{4cm} p{2.1cm} p{3cm} p{3.1cm}|}
			\hline
			\multicolumn{2}{|l}{\textbf{Variabel:}}&\multicolumn{2}{l|}{\textbf{}}\\
			long&hc&&\\
			long&medunit&&\\
			\hline
		\end{tabular}
	\end{adjustbox}
\end{table}
\begin{table}[H]
	\caption{Daftar attribut pada \textit{Class} HealthConsumer}
	\centering
	\small
	\begin{adjustbox}{width=1\textwidth}	
		\begin{tabular}{|p{4cm} p{2.1cm} p{3cm} p{3.1cm}|}
			\hline
			\multicolumn{2}{|l}{\textbf{Variabel:}}&\multicolumn{2}{l|}{\textbf{}}\\
			String&regis\_no&&\\
			Profession&id\_prof&&\\
			Insurance&insurance&&\\
			String&insurance\_type&&\\
			String&insurance\_no&&\\
			String&insurance\_prg&&\\
			\hline
		\end{tabular}
	\end{adjustbox}
\end{table}
\begin{table}[H]
	\caption{Daftar attribut pada \textit{Class} Insurance}
	\centering
	\small
	\begin{adjustbox}{width=1\textwidth}	
		\begin{tabular}{|p{5cm} p{3.1cm} p{2cm} p{2.1cm}|}
			\hline
			\multicolumn{2}{|l}{\textbf{Variabel:}}&\multicolumn{2}{l|}{\textbf{}}\\
			List$<$InsuranceBridgeConf$>$&insuranceBrigdeConfList&&\\
			int&systemid&&\\
			String&insurance&&\\
			String&memo&&\\
			boolean&active&&\\
			boolean&sysbuiltin&&\\
			\hline
		\end{tabular}
	\end{adjustbox}
\end{table}
\begin{table}[H]
	\caption{Daftar attribut pada \textit{Class} InsuranceBridgeConf}
	\centering
	\small
	\begin{adjustbox}{width=1\textwidth}	
		\begin{tabular}{|p{5cm} p{3.1cm} p{2cm} p{2.1cm}|}
			\hline
			\multicolumn{2}{|l}{\textbf{Variabel:}}&\multicolumn{2}{l|}{\textbf{}}\\
			Long&systemid&&\\
			String&field&&\\
			String&def\_val&&\\
			String&memo&&\\
			\hline
		\end{tabular}
	\end{adjustbox}
\end{table}
\begin{table}[H]
	\caption{Daftar attribut pada \textit{Class} Profession}
	\centering
	\small
	\begin{adjustbox}{width=1\textwidth}	
		\begin{tabular}{|p{5cm} p{3.1cm} p{2cm} p{2.1cm}|}
			\hline
			\multicolumn{2}{|l}{\textbf{Variabel:}}&\multicolumn{2}{l|}{\textbf{}}\\
			Integer&systemid&&\\
			String&profname&&\\
			String&memo&&\\
			Calendar&sys\_createdate&&\\
			Calendar&last\_createdate&&\\
			\hline
		\end{tabular}
	\end{adjustbox}
\end{table}
\begin{table}[H]
	\caption{Daftar attribut pada \textit{Class} Customer}
	\centering
	\small
	\begin{adjustbox}{width=1\textwidth}	
		\begin{tabular}{|p{4cm} p{2.1cm} p{3cm} p{3.1cm}|}
			\hline
			\multicolumn{2}{|l}{\textbf{Variabel:}}&\multicolumn{2}{l|}{\textbf{Variabel:}}\\
			boolean&cust\_no&Date&registrationdate\\
			boolean&active&int&idPriceLevel\\
			double&creditlimit&double&disc\\
			Customergroup&customergroup&&\\
			\hline
		\end{tabular}
	\end{adjustbox}
\end{table}

\begin{table}[H]
	\caption{Daftar attribut pada \textit{Class} Customergroup}
	\centering
	\small
	\begin{adjustbox}{width=1\textwidth}	
		\begin{tabular}{|p{4cm} p{2.1cm} p{3cm} p{3.1cm}|}
			\hline
			\multicolumn{2}{|l}{\textbf{Variabel:}}&\multicolumn{2}{l|}{}\\
			long&systemid&&\\
			String&groupname&&\\
			String&memo&&\\
			\hline
		\end{tabular}
	\end{adjustbox}
\end{table}
\begin{table}[H]
	\caption{Daftar attribut pada \textit{Class} WorkingUnit}
	\centering
	\small
	\begin{adjustbox}{width=1\textwidth}	
		\begin{tabular}{|p{4cm} p{2.1cm} p{3cm} p{3.1cm}|}
			\hline
			\multicolumn{2}{|l}{\textbf{Variabel:}}&\multicolumn{2}{l|}{\textbf{}}\\
			Long&systemid&&\\
			WUGroup&wugroup&&\\
			String&workunit\_name&&\\
			String&memo&&\\
			Calendar&createdate&&\\
			Calendar&lastupdate&&\\
			\hline
		\end{tabular}
	\end{adjustbox}
\end{table}
\begin{table}[H]
	\caption{Daftar attribut pada \textit{Class} MedicalUnit}
	\centering
	\small
	\begin{adjustbox}{width=1\textwidth}	
		\begin{tabular}{|p{4cm} p{2.1cm} p{3cm} p{3.1cm}|}
			\hline
			\multicolumn{2}{|l}{\textbf{Variabel:}}&\multicolumn{2}{l|}{}\\
			Long&systemid&&\\
			\hline
		\end{tabular}
	\end{adjustbox}
\end{table}
\begin{table}[H]
	\caption{Daftar attribut pada \textit{Class} WorkingUnitMembership}
	\centering
	\small
	\begin{adjustbox}{width=1\textwidth}	
		\begin{tabular}{|p{4cm} p{2.1cm} p{3cm} p{3.1cm}|}
			\hline
			\multicolumn{2}{|l}{\textbf{Variabel:}}&\multicolumn{2}{l|}{}\\
			WorkingUnit&workingunit&&\\
			Employee&employee&&\\
			boolean&workingunitpersonincharge&&\\
			String&description&&\\
			Calendar&sys\_createdate&&\\
			Calendar&syssys\_lastupdate&&\\
			\hline
		\end{tabular}
	\end{adjustbox}
\end{table}
\begin{table}[H]
	\caption{Daftar attribut pada \textit{Class} WUGroup}
	\centering
	\small
	\begin{adjustbox}{width=1\textwidth}	
		\begin{tabular}{|p{4cm} p{2.1cm} p{3cm} p{3.1cm}|}
			\hline
			\multicolumn{2}{|l}{\textbf{Variabel:}}&\multicolumn{2}{l|}{}\\
			Long&systemid&&\\
			String&groupname&&\\
			String&memo&&\\
			\hline
		\end{tabular}
	\end{adjustbox}
\end{table}
\begin{table}[H]
	\caption{Daftar attribut pada \textit{Class} WorkingUnitMembershipPK}
	\centering
	\small
	\begin{adjustbox}{width=1\textwidth}	
		\begin{tabular}{|p{4cm} p{2.1cm} p{3cm} p{3.1cm}|}
			\hline
			\multicolumn{2}{|l}{\textbf{Variabel:}}&\multicolumn{2}{l|}{}\\
			Long&workingunit&&\\
			Long&employee&&\\
			\hline
		\end{tabular}
	\end{adjustbox}
\end{table}
\begin{table}[H]
	\caption{Daftar attribut pada \textit{Class} Employment}
	\centering
	\small
	\begin{adjustbox}{width=1\textwidth}	
		\begin{tabular}{|p{4cm} p{2.1cm} p{3cm} p{3.1cm}|}
			\hline
			\multicolumn{2}{|l}{\textbf{Variabel:}}&\multicolumn{2}{l|}{\textbf{Variabel:}}\\
			Long&systemid&Employee&employee\\
			String&ein&Date&hiredate\\
			BigInteger&basesalary&RoleInDept&roleIndept\\
			Integer&contracttype&String& sourceinstitution\\
			int&sourcetype&Date& createdate\\
			Date&lastupdate&Date& terminatedate\\
			\hline
		\end{tabular}
	\end{adjustbox}
\end{table}
\begin{table}[H]
	\caption{Daftar attribut pada \textit{Class} Employee}
	\centering
	\small
	\begin{adjustbox}{width=1\textwidth}	
		\begin{tabular}{|p{4cm} p{2.1cm} p{3cm} p{3.1cm}|}
			\hline
			\multicolumn{2}{|l}{\textbf{Variabel:}}&\multicolumn{2}{l|}{}\\
			String&m\_ein&&\\
			Jobspeciality&jobspeciality&&\\
			Collection$<$Employment$>$&employmentCollection&&\\
			\hline
		\end{tabular}
	\end{adjustbox}
\end{table}
\begin{table}[H]
	\caption{Daftar attribut pada \textit{Class} Departement}
	\centering
	\small
	\begin{adjustbox}{width=1\textwidth}	
		\begin{tabular}{|p{4cm} p{2.1cm} p{3cm} p{3.1cm}|}
			\hline
			\multicolumn{2}{|l}{\textbf{Variabel:}}&\multicolumn{2}{l|}{\textbf{}}\\
			int&systemid&&\\
			String&deptname&&\\
			String&memo&&\\
			Collection$<$RoleInDept$>$&roles&&\\
			Date&createdate&&\\
			Date&lastupdate&&\\
			\hline
		\end{tabular}
	\end{adjustbox}
\end{table}
\begin{table}[H]
\caption{Daftar attribut pada \textit{Class} ContactAddress}
\centering
\small
\begin{adjustbox}{width=1\textwidth}	
	\begin{tabular}{|p{4cm} p{2.1cm} p{3cm} p{3.1cm}|}
		\hline
		\multicolumn{2}{|l}{\textbf{Variabel:}}&\multicolumn{2}{l|}{\textbf{Variabel:}}\\
		Long&systemid&String&postcode\\
		AddressType&addresstype&Regency&regency\\
		String&street&boolean&asbillingaddress\\
		String&fax&Date&sys\_createdate\\
		Contact&owner&String&sys\_creator\\
		String&phone&Date&sys\_lastupdate\\
		String&sys\_lastupdater&&\\
		\hline
	\end{tabular}
\end{adjustbox}
\end{table}
\begin{table}[H]
	\caption{Daftar attribut pada \textit{Class} Contact}
	\centering
	\small
	\begin{adjustbox}{width=1\textwidth}	
		\begin{tabular}{|p{2cm} p{2.1cm} p{5cm} p{3.1cm}|}
			\hline
			\multicolumn{2}{|l}{\textbf{Variabel:}}&\multicolumn{2}{l|}{\textbf{Variabel:}}\\
			long&systemid&String&initial\\
			String&lastname&int&amountofchildren\\
			int&maritalstatus&List$<$ContactAddress$>$&arrAddress\\
			String&middlename&Collection$<$ContactEducation$>$&arrEdu\\
			Date&birthday&String&notes\\
			String&birthplace&String&officeext\\
			String&bloodtype&String&passportid\\
			double&bodyheight&byte[]&photo\\
			double&bodyweight&String&prefixtitle\\
			String&citizenid&String&sms\\
			String&citizentype&String&suffixtitle\\
			Country&citizenship&Date&sys\_createdate\\
			String&email&String&sys\_creator\\
			String&firstname&Date&sys\_lastupdate\\
			int&gender&String&sys\_lastupdater\\
			String&homephone&String&taxid\\
			String&messaging&String&web\\
			\hline
		\end{tabular}
	\end{adjustbox}
\end{table}
\begin{table}[H]
	\caption{Daftar attribut pada \textit{Class} AddressType}
	\centering
	\small
	\begin{adjustbox}{width=1\textwidth}	
		\begin{tabular}{|p{4cm} p{2.1cm} p{3cm} p{3.1cm}|}
			\hline
			\multicolumn{2}{|l}{\textbf{Variabel:}}&\multicolumn{2}{l|}{\textbf{}}\\
			Integer&systemid&&\\
			String&memo&&\\
			String&typename&&\\
			\hline
		\end{tabular}
	\end{adjustbox}
\end{table}
\begin{table}[H]
	\caption{Daftar attribut pada \textit{Class} ContactEducation}
	\centering
	\small
	\begin{adjustbox}{width=1\textwidth}	
		\begin{tabular}{|p{3cm} p{3.1cm} p{3cm} p{3.1cm}|}
			\hline
			\multicolumn{2}{|l}{\textbf{Variabel:}}&\multicolumn{2}{l|}{\textbf{\textbf{Variabel:}}}\\
			Date&finishrolling&Contact&owner\\
			String&gpa&String&sponsors\\
			int&grade&Date&startrolling\\
			String&institutionaladdress&Date&sys\_createdate\\
			String&institutionname&int&sys\_creator\\
			String&major&Date&sys\_lastupdate\\
			String&notes&int&sys\_lastupdater\\
			\hline
		\end{tabular}
	\end{adjustbox}
\end{table}
\begin{table}[H]
	\caption{Daftar attribut pada \textit{Class} Country}
	\centering
	\small
	\begin{adjustbox}{width=1\textwidth}	
		\begin{tabular}{|p{4cm} p{2.1cm} p{3cm} p{3.1cm}|}
			\hline
			\multicolumn{2}{|l}{\textbf{Variabel:}}&\multicolumn{2}{l|}{\textbf{}}\\
			Long&systemid&&\\
			String&name&&\\
			\hline
		\end{tabular}
	\end{adjustbox}
\end{table}
\begin{table}[H]
	\caption{Daftar attribut pada \textit{Class} Jobspeciality}
	\centering
	\small
	\begin{adjustbox}{width=1\textwidth}	
		\begin{tabular}{|p{5cm} p{3.1cm} p{2cm} p{2.1cm}|}
			\hline
			\multicolumn{2}{|l}{\textbf{Variabel:}}&\multicolumn{2}{l|}{\textbf{}}\\
			Long&systemid&&\\
			String&specialityName&&\\
			String&memo&&\\
			\hline
		\end{tabular}
	\end{adjustbox}
\end{table}
\begin{table}[H]
	\caption{Daftar attribut pada \textit{Class} RoleInDept}
	\centering
	\small
	\begin{adjustbox}{width=1\textwidth}	
		\begin{tabular}{|p{5cm} p{3.1cm} p{2cm} p{2.1cm}|}
			\hline
			\multicolumn{2}{|l}{\textbf{Variabel:}}&\multicolumn{2}{l|}{\textbf{}}\\
			int&systemid&&\\
			Department&department&&\\
			String&rolename&&\\
			String&abbreviation&&\\
			RoleInDept&parentRole&&\\
			String&memo&&\\
			\hline
		\end{tabular}
	\end{adjustbox}
\end{table}
\begin{table}[H]
	\caption{Daftar attribut pada \textit{Class} Province}
	\centering
	\small
	\begin{adjustbox}{width=1\textwidth}	
		\begin{tabular}{|p{5cm} p{3.1cm} p{2cm} p{2.1cm}|}
			\hline
			\multicolumn{2}{|l}{\textbf{Variabel:}}&\multicolumn{2}{l|}{\textbf{}}\\
			Long&systemid&&\\
			Country&countryCode&&\\
			String&name&&\\
			Set$<$Regency$>$&regencySet&&\\
			\hline
		\end{tabular}
	\end{adjustbox}
\end{table}
\begin{table}[H]
	\caption{Daftar attribut pada \textit{Class} Regency}
	\centering
	\small
	\begin{adjustbox}{width=1\textwidth}	
		\begin{tabular}{|p{5cm} p{3.1cm} p{2cm} p{2.1cm}|}
			\hline
			\multicolumn{2}{|l}{\textbf{Variabel:}}&\multicolumn{2}{l|}{\textbf{}}\\
			Long&systemid&&\\
			Province&provId&&\\
			String&name&&\\
			\hline
		\end{tabular}
	\end{adjustbox}
\end{table}
\subsubsection{\textit{Package} Repository}
\textit{Package} repository berisi \textit{class-class} yang menjadi representasi model dari tabel-tabel yang berada pada basis data. Berikut ini merupakan \textit{class-class} yang terdapat pada \textit{package} repository.
\begin{table}[H]
	\small
	\centering
	\caption{Daftar {\itshape Class} pada {\itshape Package} repository}
	\begin{adjustbox}{width=1\textwidth}
		\begin{tabular}{| p {3 cm} | p {8 cm} | p {3 cm} |}
			\hline
			{\bfseries Package} & {\bfseries Class} & {\bfseries Jenis Class} \\
			\hline
			\multirow{1}{*}{repository} & OutpatientWUQueueRepository & {\itshape Interface} \\
			\hline
		\end{tabular}
	\end{adjustbox}
\end{table}
\subsubsection{\textit{Package} Service}
\textit{Package} service berisi \textit{class-class} yang menginisialisasi \textit{web service} human resource yang akan terbentuk dan digunakan pada \textit{package} controller. Berikut ini merupakan \textit{class-class} yang terdapat pada \textit{package} service.
\begin{table}[H]
	\small
	\centering
	\caption{Daftar {\itshape Class} pada {\itshape Package} service}
	\begin{adjustbox}{width=1\textwidth}
		\begin{tabular}{| p {3 cm} | p {8 cm} | p {3 cm} |}
			\hline
			{\bfseries Package} & {\bfseries Class} & {\bfseries Jenis Class} \\
			\hline
			\multirow{3}{*}{service} & CRUDService & {\itshape Interface} \\
			& OutpatientWUQueueService & {\itshape Interface} \\
			& DefaultOutpatientWUQueueService & {\itshape Class} \\
			\hline
		\end{tabular}
	\end{adjustbox}
\end{table}
\section{Perbandingan Model Arsitektur Microservice dan Monolitik}
Pada bagian ini akan dilakukan perbandingan terhadap kedua jenis model arsitektur. Perbandingan dilakukan dengan memberikan analisa dan bukti dari bagaimana sistem bekerja pada kedua jenis arsitektur untuk setiap faktor uji yang terdiri dari : Performa, tingkat availabilitas, tingkat stabilitas, skalabilitas, dan reliabilitas dari sistem.\\
Berikut hasil analisa perbandingan dari kedua arsitektur :
\begin{enumerate}[leftmargin=*]
	\item Perbandingan Performa. Kemampuan sistem untuk menangani masalah yang diberikan dan seberapa efisien sistem menggunakan sumber daya yang tersedia (seperti \textit{hardware} dan komponen infrastruktur lainnya). 
	\begin{table}[H]
		\small
		\begin{adjustbox}{width=1\textwidth}
			\begin{tabular}{| p {2 cm} | p {10 cm} |}
				\hline
				Arsitektur & Perbandingan Performa\\
				\hline
				\textbf{Monolitik} & Sistem Apertura dengan desain arsitektur monolitik membutuhkan 1 buah server berkapasitas ... GB dengan ... VRAM dan kecepatan core sampai ... Biaya yang dibutuhkan untuk infrastruktur sistem Apertura adalah ... Dengan spesifikasi \textit{hardware} tersebut, sistem Apertura dapat bekerja dengan baik. Sebagai contoh sistem Apertura dapat membuka halaman \textit{Human Resources} dengan banyak data 200 data dengan kecepatan ... detik.\\
				\hline
				\textbf{Microservice} & Dengan menggunakan arsitektur microservice, modul-modul layanan dibagi menjadi 4 buah service yang akan di\textit{deploy} pada ... server masing-masing dengan kapasitas ... GB, ... VRAM dengan kecepatan core ... Biaya tiap \textit{server} yang dibutuhkan adalah... Dengan spesifikasi \textit{hardware} tersebut, sistem dipastikan dapat bekerja dengan baik. Sebagai contoh sistem dengan desain microservice dapat membuka halaman \textit{Human Resources} dengan banyak data 200 data dengan kecepatan 2 detik. Namun dengan membutuhkan lebih dari 1 \textit{server}, akan meningkatkan \textit{cost} yang dibutuhkan, juga biaya perawatan dan \textit{maintenance}.\\
				\hline
			\end{tabular}
		\end{adjustbox}
	\end{table}
	\item Tingkat Stabilitas. Sistem harus dipastikan selalu memberikan kondisi yang aman untuk tahap \textit{development, deployment}, sampai tahap pengenalan. 
		\begin{table}[H]
		\small
		\begin{adjustbox}{width=1\textwidth}
			\begin{tabular}{| p {2 cm} | p {10 cm} |}
				\hline
				Arsitektur & Perbandingan Performa\\
				\hline
				\textbf{Monolitik} & Sistem Apertura memberikan performa yang stabil dari tahap pengembangan sampai dengan tahap \textit{deployment}. Sistem sangat jarang mengalami masalah sistem yang menyebabkan sistem mengalami \textit{crash}. \\
				\hline
				\textbf{Microservice} & Dengan perancangan sistem yang baru, prototype dari desain microservice memberikan performa yang stabil. Service yang dijalankan pada server sangat jarang mengalami masalah dan \textit{RESTul API} dapat menangani \textit{request} dan \textit{response} dengan baik dan stabil.\\
				\hline
			\end{tabular}
		\end{adjustbox}
	\end{table}
	\item Tingkat Reliabilitas. Sebuah sistem yang \textit{reliable} harus dapat memberikan data yang dapat dipercaya oleh \textit{clients} dan ekosistem tempat sistem itu berada. \textit{Request} dan \textit{response} harus sampai tepat pada tujuan dan \textit{error} harus dapat diatasi secara benar dan aman.
		\begin{table}[H]
		\small
		\begin{adjustbox}{width=1\textwidth}
			\begin{tabular}{| p {2 cm} | p {10 cm} |}
				\hline
				Arsitektur & Perbandingan Performa\\
				\hline
				\textbf{Monolitik} & Sistem Apertura menampilkan dan menerima data dengan baik. Namun arsitektur monolitik seperti Apertura menyimpanan data pada satu buah server terpusat. Cara penyimpanan ini memiliki resiko kerusakan data yang lebih besar, juga memiliki \textit{issue} keamanan yang lebih tinggi. Misalnya apabila sistem keamanan database ditembus, maka informasi pada semua data dapat dirusak atau terancam.\\
				\hline
				\textbf{Microservice} & Dengan menggunakan arsitektur microservice, \textit{prototype} dapat menampilkan dan menerima data degan baik. Perancangan microservice yang membagi server menjadi lebih dari 1 memiliki keunggulan, yaitu tingkat keamanan data yang lebih tinggi. Misalnya dengan menyimpan data-data penting seperti data personal terpisah dari data umum, maka data personal akan tetap aman apabila database penyimpanan data umum mengalami kerusakan.\\
				\hline
			\end{tabular}
		\end{adjustbox}
	\end{table}
	\item Skalabilitas. Kemampuan sistem untuk menangani permintaan yang besar dalam satu waktu bersamaan. Untuk dapat memastikan bahwa sistem \textit{scalable}, perlu diketahui seberapa besar ukuran sistem secara kuantitatif (misal seberapa banyak \textit{request} per detik yang dapat ditangani sistem).
		\begin{table}[H]
		\small
		\begin{adjustbox}{width=1\textwidth}
			\begin{tabular}{| p {2 cm} | p {10 cm} |}
				\hline
				Arsitektur & Perbandingan Performa\\
				\hline
				\textbf{Monolitik} & Sistem Apertura dibuat dalam bentuk aplikasi \textit{desktop} dan dapat menangani permintaan data dengan baik. Namun apabila \textit{request} menjadi banyak, maka akan terdapat masalah \textit{bottle neck} pada sistem dikarenakan semua \textit{request} dilakukan pada 1 buah \textit{handler}, hal ini dapat mempengaruhi performa dari sistem.\\
				\hline
				\textbf{Microservice} & Dengan membagi modul menjadi 4 bagian dan tiap modul di \textit{deploy} pada server yang berbeda, prototype dengan arsitektur microservice dapat menangani permintaan lebih cepat apabila \textit{request} per detik lebih tinggi karena tiap permintaan di setiap modul akan di \textit{handle} pada \textit{server}nya masing-masing.\\
				\hline
			\end{tabular}
		\end{adjustbox}
	\end{table}
	\item Tingkat Availabilitas. Perbandingan waktu \textit{downtime} dan \textit{uptime} dari sistem. \textit{Downtime} adalah total waktu aplikasi tidak bekerja sedangkan \textit{uptime} adalah total waktu sistem bekerja dengan baik. Tingkat availabilitas dihitung dari \textit{uptime} dibagi dengan \textit{downtime} + \textit{uptime}. Tingkat availabilitas dapat digunakan untuk melihat seberapa baik sistem bekerja.
		\begin{table}[H]
		\small
		\begin{adjustbox}{width=1\textwidth}
			\begin{tabular}{| p {2 cm} | p {10 cm} |}
				\hline
				Arsitektur & Perbandingan Performa\\
				\hline
				\textbf{Monolitik} & Untuk membandingkan seberapa besar tingkat availabilitas dari sistem Apertura, maka akan diangkat sebuah skenario masalah yang pernah terjadi pada sistem, yaitu kemampuan sistem menangani kondisi dimana sistem harus \textit{down} beberapa saat karena terjadi \textit{maintenance} yang harus dilakukan. Dalam desain monolitik, maka keseluruhan fitur dan halaman dalam sistem menjadi tidak dapat diakses. Untuk rawat jalan, misal terdapat 
				\hline
				\textbf{Microservice} & Dengan membagi modul dan melakukan \textit{deployment} pada ... server maka akan memperbesar presentase availabilitas dari keseluruhan sistem. Apabila sebuah server mengalami \textit{down}, modul lain yang terdapat di \textit{server} lain akan tetap dapat dijalankan dengan baik, misalnya modul rawat jalan \textit{down}, sistem tetap dapat melihat unit medis yang bertugas, melihat data dokter dan pasien. Prototype dengan bentuk development \textit{web application} memberikan kelebihan untuk dapat lebih \textit{available} untuk digunakan pada \textit{platform} lain.\\
				\hline
			\end{tabular}
		\end{adjustbox}
	\end{table}
\end{enumerate}
\newpage
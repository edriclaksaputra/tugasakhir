%-----------------------------------------------------------------------------%
\chapter{IMPLEMENTASI DAN PENGUJIAN}
%-----------------------------------------------------------------------------%

%
\vspace{4.5pt}
Pada bab ini akan menjelaskan tentang pengimplementasian dan pengujian terhadap analisis sentimen yang telah dibangun berdasarkan bab-bab sebelumnya.
\section{Lingkungan Aplikasi}
Dalam aplikasi terbagi menjadi dua bagian, yaitu lingkungan implementasi perangkat keras dan perangkat lunak. Di dalam penelitian ini, perangkat keras yang digunakan adalah:
\begin{enumerate}[leftmargin=*]
	\item Asus A45A
	\item Processor Intel i3-2370M CPU 2.40GHz
	\item RAM 6 GB.
\end{enumerate}

Spesifikasi perangkat lunak yang digunakan untuk pengembangan sistem adalah:
\begin{enumerate}[leftmargin=*]
	\item Sistem Operasi\quad\quad\quad\,: Windows 10 Enterprise 1709 64-bit.
	\item Tool Pengembangan\quad: Eclipse Java EE IDE for Web Developers.
	\item Versi\quad\quad\quad\quad\quad\quad\quad\,: Neon.3 Release (4.6.3).
\end{enumerate}

\section{Daftar \textit{Project}, \textit{Class} dan \textit{Method}}
Pada bagian ini akan dijelaskan mengenai \textit{Project}, \textit{class} dan \textit{method} yang digunakan dalam pengembangan sistem analisis.
\subsection{\textit{Project} Customer}
\textit{Project Customer} berisi susunan \textit{class} pembentuk \textit{webservice customer}. Dalam project ini terdapat 4 buah \textit{package} yang berfungsi untuk memisahkan \textit{class} sesuai dengan fungsinya masing-masing. Ke empat \textit{package} ini yaitu \textit{package} controller yang berisi method-method yang dijalankan pada \textit{web service}, \textit{package} model yang berisi entitas dan tabel, \textit{package} repository yang menghubungkan entitas di \textit{package} model dengan database, dan \textit{package} service yang menginisialisasi \textit{webservice} yang akan terbentuk. Pada bagian ini akan dijelaskan mengenai \textit{class} dan \textit{method} yang digunakan pada pengembangan \textit{ webservice customer}:
\subsubsection{\textit{Package} Controller}
\textit{Package} controller berisi method API yang dapat digunakan untuk berkomunikasi dengan \textit{service} customer. \textit{Class} yang terdapat dalam \textit{package} ini adalah \textit{class} CustomerController dan CustomergroupController yang berisi 5 buah fungsi API. Di bawah ini merupakan daftar \textit{class} untuk \textit{package} controller pada \textit{webservice} customer.
\begin{table}[H]
	\small
	\centering
	\caption{Daftar {\itshape Class} pada {\itshape Package} Controller}
	\begin{adjustbox}{width=1\textwidth}
		\begin{tabular}{| p {3 cm} | p {8 cm} | p {3 cm} |}
			\hline
			{\bfseries Package} & {\bfseries Class} & {\bfseries Jenis Class} \\
			\hline
			\multirow{2}{*}{Controller} & CustomerController & {\itshape Class} \\
			& CustomergroupController & {\itshape Class} \\
			\hline
		\end{tabular}
	\end{adjustbox}
\end{table}
\begin{table}[H]
	\caption{Daftar \textit{Method} pada \textit{Class} CustomerController}
	\centering
	\small
	\begin{adjustbox}{width=1\textwidth}	
	\begin{tabular}{|p{0.4cm}|p{3.2cm}|p{1.4cm}|p{1.7cm}|p{1.55cm}|p{3cm}|}
		\hline
		\multirow{2}{*}{\textbf{No}} & \multirow{2}{*}{\textit{\textbf{Method}}} & \multicolumn{2}{c|}{\textit{\textbf{Input}}} & \multirow{2}{*}{\textit{\textbf{Output}}} & 
		\multirow{2}{*}{\textbf{Keterangan}}\\
		\cline{3-4}
		& & \textbf{Tipe} & \textbf{Variabel} & & \\
		\hline
		1 & addCustomer & void & Customer & Customer, HttpStatus & \textit{Method} ini digunakan untuk membuat \textit{object} customer yang baru melalui \textit{webservice}\\
		\hline
		2 & updateCustomer & void & Customer & HttpStatus & \textit{Method} ini digunakan untuk mengubah attribut dari \textit{object} customer yang baru melalui \textit{webservice}\\
		\hline
		3 & getCustomer & void & id & Customer, HttpStatus & \textit{Method} ini digunakan untuk mengambil satu \textit{object} customer yang baru melalui \textit{webservice}\\
		\hline
	\end{tabular}
	\end{adjustbox}
\end{table}
\begin{table}[H]
	\centering
	\small
	\begin{adjustbox}{width=1\textwidth}	
		\begin{tabular}{|p{0.4cm}|p{2.5cm}|p{1cm}|p{1.1cm}|p{3.4cm}|p{3cm}|}
			\hline
			4 & getAllCustomer & void & - & $<$List$<$Customer$>$$>$, HttpStatus & \textit{Method} ini digunakan untuk mengambil semua \textit{list} \textit{object} customer yang baru melalui \textit{webservice}\\
			\hline
			5 & deleteCustomer & void & id &HttpStatus & \textit{Method} ini digunakan untuk menghapus satu \textit{object} customer yang baru melalui \textit{webservice}\\
			\hline
		\end{tabular}
	\end{adjustbox}
\end{table}
\begin{table}[H]
	\caption{Daftar \textit{Method} pada \textit{Class} CustomergroupController}
	\centering
	\small
	\begin{adjustbox}{width=1\textwidth}	
		\begin{tabular}{|p{0.4cm}|p{3.5cm}|p{1.4cm}|p{1.7cm}|p{1.55cm}|p{3cm}|}
			\hline
			\multirow{2}{*}{\textbf{No}} & \multirow{2}{*}{\textit{\textbf{Method}}} & \multicolumn{2}{c|}{\textit{\textbf{Input}}} & \multirow{2}{*}{\textit{\textbf{Output}}} & 
			\multirow{2}{*}{\textbf{Keterangan}}\\
			\cline{3-4}
			& & \textbf{Tipe} & \textbf{Variabel} & & \\
			\hline
			1 & addCustomerGroup & Customer & customer & customer, HttpStatus & \textit{Method} ini digunakan untuk membuat \textit{object} CustomerGroup yang baru melalui \textit{webservice}\\
			\hline
			2 & updateCustomerGroup & Customer & customer & HttpStatus & \textit{Method} ini digunakan untuk mengubah attribut dari \textit{object} CustomerGroup yang baru melalui \textit{webservice}\\
			\hline
			3 & getCustomerGroup & Customer & customer & customer, HttpStatus & \textit{Method} ini digunakan untuk mengambil satu \textit{object} CustomerGroup yang baru melalui \textit{webservice}\\
			\hline
			4 & getAllCustomerGroup & Customer & customer & customer, HttpStatus & \textit{Method} ini digunakan untuk mengambil semua \textit{list} \textit{object} CustomerGroup yang baru melalui \textit{webservice}\\
			\hline
		\end{tabular}
	\end{adjustbox}
\end{table}
\begin{table}[H]
	\centering
	\small
	\begin{adjustbox}{width=1\textwidth}	
		\begin{tabular}{|p{0.4cm}|p{3.5cm}|p{1.4cm}|p{1.7cm}|p{1.55cm}|p{3cm}|}
			\hline
			5 & deleteCustomerGroup & Customer & customer & customer, HttpStatus & \textit{Method} ini digunakan untuk menghapus satu \textit{object} CustomerGroup yang baru melalui \textit{webservice}\\
			\hline
		\end{tabular}
	\end{adjustbox}
\end{table}
\subsubsection{\textit{Package} Model Customer}
\textit{Package} customer model berisi entitas-entitas penyusun dari \textit{service} customer. Berikut ini merupakan daftar \textit{class} untuk \textit{package} model.
\begin{table}[H]
	\small
	\centering
	\caption{Daftar {\itshape Class} pada {\itshape Package} model}
	\begin{adjustbox}{width=1\textwidth}
		\begin{tabular}{| p {3 cm} | p {8 cm} | p {3 cm} |}
			\hline
			{\bfseries Package} & {\bfseries Class} & {\bfseries Jenis Class} \\
			\hline
			\multirow{13}{*}{Model} & Regency & {\itshape Class} \\
			& AddressType & {\itshape Class} \\
			& Contact & {\itshape Class} \\
			& ContactAddress & {\itshape Class} \\
			& ContactEducation & {\itshape Class} \\
			& Country & {\itshape Class} \\
			& Customer & {\itshape Class} \\
			& Customergroup & {\itshape Class} \\
			& HealthConsumer & {\itshape Class} \\
			& Insurance & {\itshape Class} \\
			& InsuranceBridgeConf & {\itshape Class} \\
			& Profession & {\itshape Class} \\
			& Province & {\itshape Class} \\
			\hline
		\end{tabular}
	\end{adjustbox}
\end{table}
\begin{table}[H]
	\caption{Daftar attribut pada \textit{Class} Customer}
	\centering
	\small
	\begin{adjustbox}{width=1\textwidth}	
		\begin{tabular}{|p{4cm} p{2.1cm} p{3cm} p{3.1cm}|}
			\hline
			\multicolumn{2}{|l}{\textbf{Variabel:}}&\multicolumn{2}{l|}{\textbf{Variabel:}}\\
			boolean&cust\_no&Date&registrationdate\\
			boolean&active&int&idPriceLevel\\
			double&creditlimit&double&disc\\
			Customergroup&customergroup&&\\
			\hline
		\end{tabular}
	\end{adjustbox}
\end{table}

\begin{table}[H]
	\caption{Daftar attribut pada \textit{Class} Customergroup}
	\centering
	\small
	\begin{adjustbox}{width=1\textwidth}	
		\begin{tabular}{|p{4cm} p{2.1cm} p{3cm} p{3.1cm}|}
			\hline
			\multicolumn{2}{|l}{\textbf{Variabel:}}&\multicolumn{2}{l|}{}\\
			long&systemid&&\\
			String&groupname&&\\
			String&memo&&\\
			\hline
		\end{tabular}
	\end{adjustbox}
\end{table}

\begin{table}[H]
	\caption{Daftar attribut pada \textit{Class} Contact}
	\centering
	\small
	\begin{adjustbox}{width=1\textwidth}	
		\begin{tabular}{|p{2cm} p{2.1cm} p{5cm} p{3.1cm}|}
			\hline
			\multicolumn{2}{|l}{\textbf{Variabel:}}&\multicolumn{2}{l|}{\textbf{Variabel:}}\\
			long&systemid&String&initial\\
			String&lastname&int&amountofchildren\\
			int&maritalstatus&List$<$ContactAddress$>$&arrAddress\\
			String&middlename&Collection$<$ContactEducation$>$&arrEdu\\
			Date&birthday&String&notes\\
			String&birthplace&String&officeext\\
			String&bloodtype&String&passportid\\
			double&bodyheight&byte[]&photo\\
			double&bodyweight&String&prefixtitle\\
			String&citizenid&String&sms\\
			String&citizentype&String&suffixtitle\\
			Country&citizenship&Date&sys\_createdate\\
			String&email&String&sys\_creator\\
			String&firstname&Date&sys\_lastupdate\\
			int&gender&String&sys\_lastupdater\\
			String&homephone&String&taxid\\
			String&messaging&String&web\\
			\hline
		\end{tabular}
	\end{adjustbox}
\end{table}
\begin{table}[H]
	\caption{Daftar attribut pada \textit{Class} Contact Address}
	\centering
	\small
	\begin{adjustbox}{width=1\textwidth}	
		\begin{tabular}{|p{4cm} p{2.1cm} p{3cm} p{3.1cm}|}
			\hline
			\multicolumn{2}{|l}{\textbf{Variabel:}}&\multicolumn{2}{l|}{\textbf{Variabel:}}\\
			Long&systemid&String&postcode\\
			AddressType&addresstype&Regency&regency\\
			String&street&boolean&asbillingaddress\\
			String&fax&Date&sys\_createdate\\
			Contact&owner&String&sys\_creator\\
			String&phone&Date&sys\_lastupdate\\
			String&sys\_lastupdater&&\\
			\hline
		\end{tabular}
	\end{adjustbox}
\end{table}
\begin{table}[H]
	\caption{Daftar attribut pada \textit{Class} Address Type}
	\centering
	\small
	\begin{adjustbox}{width=1\textwidth}	
		\begin{tabular}{|p{4cm} p{2.1cm} p{3cm} p{3.1cm}|}
			\hline
			\multicolumn{2}{|l}{\textbf{Variabel:}}&\multicolumn{2}{l|}{\textbf{}}\\
			Integer&systemid&&\\
			String&memo&&\\
			String&typename&&\\
			\hline
		\end{tabular}
	\end{adjustbox}
\end{table}
\begin{table}[H]
	\caption{Daftar attribut pada \textit{Class} Contact Education}
	\centering
	\small
	\begin{adjustbox}{width=1\textwidth}	
		\begin{tabular}{|p{3cm} p{3.1cm} p{3cm} p{3.1cm}|}
			\hline
			\multicolumn{2}{|l}{\textbf{Variabel:}}&\multicolumn{2}{l|}{\textbf{\textbf{Variabel:}}}\\
			Date&finishrolling&Contact&owner\\
			String&gpa&String&sponsors\\
			int&grade&Date&startrolling\\
			String&institutionaladdress&Date&sys\_createdate\\
			String&institutionname&int&sys\_creator\\
			String&major&Date&sys\_lastupdate\\
			String&notes&int&sys\_lastupdater\\
			\hline
		\end{tabular}
	\end{adjustbox}
\end{table}
\begin{table}[H]
	\caption{Daftar attribut pada \textit{Class} Country}
	\centering
	\small
	\begin{adjustbox}{width=1\textwidth}	
		\begin{tabular}{|p{4cm} p{2.1cm} p{3cm} p{3.1cm}|}
			\hline
			\multicolumn{2}{|l}{\textbf{Variabel:}}&\multicolumn{2}{l|}{\textbf{}}\\
			Long&systemid&&\\
			String&name&&\\
			\hline
		\end{tabular}
	\end{adjustbox}
\end{table}
\begin{table}[H]
	\caption{Daftar attribut pada \textit{Class} HealthConsumer}
	\centering
	\small
	\begin{adjustbox}{width=1\textwidth}	
		\begin{tabular}{|p{4cm} p{2.1cm} p{3cm} p{3.1cm}|}
			\hline
			\multicolumn{2}{|l}{\textbf{Variabel:}}&\multicolumn{2}{l|}{\textbf{}}\\
			String&regis\_no&&\\
			Profession&id\_prof&&\\
			Insurance&insurance&&\\
			String&insurance\_type&&\\
			String&insurance\_no&&\\
			String&insurance\_prg&&\\
			\hline
		\end{tabular}
	\end{adjustbox}
\end{table}
\begin{table}[H]
	\caption{Daftar attribut pada \textit{Class} Insurance}
	\centering
	\small
	\begin{adjustbox}{width=1\textwidth}	
		\begin{tabular}{|p{5cm} p{3.1cm} p{2cm} p{2.1cm}|}
			\hline
			\multicolumn{2}{|l}{\textbf{Variabel:}}&\multicolumn{2}{l|}{\textbf{}}\\
			List$<$InsuranceBridgeConf$>$&insuranceBrigdeConfList&&\\
			int&systemid&&\\
			String&insurance&&\\
			String&memo&&\\
			boolean&active&&\\
			boolean&sysbuiltin&&\\
			\hline
		\end{tabular}
	\end{adjustbox}
\end{table}
\begin{table}[H]
	\caption{Daftar attribut pada \textit{Class} InsuranceBridgeConf}
	\centering
	\small
	\begin{adjustbox}{width=1\textwidth}	
		\begin{tabular}{|p{5cm} p{3.1cm} p{2cm} p{2.1cm}|}
			\hline
			\multicolumn{2}{|l}{\textbf{Variabel:}}&\multicolumn{2}{l|}{\textbf{}}\\
			Long&systemid&&\\
			String&field&&\\
			String&def\_val&&\\
			String&memo&&\\
			\hline
		\end{tabular}
	\end{adjustbox}
\end{table}
\begin{table}[H]
	\caption{Daftar attribut pada \textit{Class} Profession}
	\centering
	\small
	\begin{adjustbox}{width=1\textwidth}	
		\begin{tabular}{|p{5cm} p{3.1cm} p{2cm} p{2.1cm}|}
			\hline
			\multicolumn{2}{|l}{\textbf{Variabel:}}&\multicolumn{2}{l|}{\textbf{}}\\
			Integer&systemid&&\\
			String&profname&&\\
			String&memo&&\\
			Calendar&sys\_createdate&&\\
			Calendar&last\_createdate&&\\
			\hline
		\end{tabular}
	\end{adjustbox}
\end{table}
\begin{table}[H]
	\caption{Daftar attribut pada \textit{Class} Province}
	\centering
	\small
	\begin{adjustbox}{width=1\textwidth}	
		\begin{tabular}{|p{5cm} p{3.1cm} p{2cm} p{2.1cm}|}
			\hline
			\multicolumn{2}{|l}{\textbf{Variabel:}}&\multicolumn{2}{l|}{\textbf{}}\\
			Long&systemid&&\\
			Country&countryCode&&\\
			String&name&&\\
			Set$<$Regency$>$&regencySet&&\\
			\hline
		\end{tabular}
	\end{adjustbox}
\end{table}
\begin{table}[H]
	\caption{Daftar attribut pada \textit{Class} Regency}
	\centering
	\small
	\begin{adjustbox}{width=1\textwidth}	
		\begin{tabular}{|p{5cm} p{3.1cm} p{2cm} p{2.1cm}|}
			\hline
			\multicolumn{2}{|l}{\textbf{Variabel:}}&\multicolumn{2}{l|}{\textbf{}}\\
			Long&systemid&&\\
			Province&provId&&\\
			String&name&&\\
			\hline
		\end{tabular}
	\end{adjustbox}
\end{table}
\subsubsection{\textit{Package} Repository}
\textit{Package} repository berisi \textit{class-class} yang menjadi representasi model dari tabel-tabel yang berada pada basis data. Berikut ini merupakan \textit{class-class} yang terdapat pada \textit{package} repository.
\begin{table}[H]
	\small
	\centering
	\caption{Daftar {\itshape Class} pada {\itshape Package} repository}
	\begin{adjustbox}{width=1\textwidth}
		\begin{tabular}{| p {3 cm} | p {8 cm} | p {3 cm} |}
			\hline
			{\bfseries Package} & {\bfseries Class} & {\bfseries Jenis Class} \\
			\hline
			\multirow{2}{*}{repository} & CustomerRepository & {\itshape Interface} \\
			& CustomergroupRepository & {\itshape Interface} \\
			\hline
		\end{tabular}
	\end{adjustbox}
\end{table}
\subsubsection{\textit{Package} Service}
\textit{Package} service berisi \textit{class-class} yang menginisialisasi \textit{web service} customer yang akan terbentuk dan digunakan pada \textit{package} controller. Berikut ini merupakan \textit{class-class} yang terdapat pada \textit{package} service.
\begin{table}[H]
	\small
	\centering
	\caption{Daftar {\itshape Class} pada {\itshape Package} serviec}
	\begin{adjustbox}{width=1\textwidth}
		\begin{tabular}{| p {3 cm} | p {8 cm} | p {3 cm} |}
			\hline
			{\bfseries Package} & {\bfseries Class} & {\bfseries Jenis Class} \\
			\hline
			\multirow{5}{*}{service} & CRUDService & {\itshape Interface} \\
			& CustomergroupService & {\itshape Interface} \\
			& CustomerService & {\itshape Interface} \\
			& DefaultCustomergroupService & {\itshape Class} \\
			& DefaultCustomerService & {\itshape Class} \\
			\hline
		\end{tabular}
	\end{adjustbox}
\end{table}
\subsection{\textit{Project} \textit{Human Resource}}
\textit{Project Human Resource} berisi susunan \textit{class} pembentuk \textit{webservice Human Resource}. Dalam project ini terdapat 4 buah \textit{package} yang berfungsi untuk memisahkan \textit{class} sesuai dengan fungsinya masing-masing. Ke empat \textit{package} ini yaitu \textit{package} controller yang berisi method-method yang dijalankan pada \textit{web service}, \textit{package} model yang berisi entitas dan tabel, \textit{package} repository yang menghubungkan entitas di \textit{package} model dengan database, dan \textit{package} service yang menginisialisasi \textit{webserviec} yang akan terbentuk. Pada bagian ini akan dijelaskan mengenai \textit{class} dan \textit{method} yang digunakan pada pengembangan \textit{ webservice customer}:
\subsubsection{\textit{Package} Controller}
\textit{Package} controller berisi method API yang dapat digunakan untuk berkomunikasi dengan \textit{service} Human Resource. \textit{Class} yang terdapat dalam \textit{package} ini adalah \textit{class} EmployeeController dan JobSpecialityController yang berisi 5 buah fungsi API. Di bawah ini merupakan daftar \textit{class} untuk \textit{package} controller pada \textit{webservice} \textit{Human Resource}.
\begin{table}[H]
	\small
	\centering
	\caption{Daftar {\itshape Class} pada {\itshape Package} Controller}
	\begin{adjustbox}{width=1\textwidth}
		\begin{tabular}{| p {3 cm} | p {8 cm} | p {3 cm} |}
			\hline
			{\bfseries Package} & {\bfseries Class} & {\bfseries Jenis Class} \\
			\hline
			\multirow{2}{*}{Controller} & EmployeeController & {\itshape Class} \\
			& JobspecialityController & {\itshape Class} \\
			\hline
		\end{tabular}
	\end{adjustbox}
\end{table}
\begin{table}[H]
	\caption{Daftar \textit{Method} pada \textit{Class} EmployeeController}
	\centering
	\small
	\begin{adjustbox}{width=1\textwidth}	
		\begin{tabular}{|p{0.4cm}|p{2.5cm}|p{0.9cm}|p{1.5cm}|p{3.4cm}|p{2.5cm}|}
			\hline
			\multirow{2}{*}{\textbf{No}} & \multirow{2}{*}{\textit{\textbf{Method}}} & \multicolumn{2}{c|}{\textit{\textbf{Input}}} & \multirow{2}{*}{\textit{\textbf{Output}}} & 
			\multirow{2}{*}{\textbf{Keterangan}}\\
			\cline{3-4}
			& & \textbf{Tipe} & \textbf{Variabel} & & \\
			\hline
			1 & addEmployee & void & Employee & Employee, HttpStatus & \textit{Method} ini digunakan untuk membuat \textit{object} employee yang baru melalui \textit{webservice}\\
			\hline
			2 & updateEmployee & void & Employee & HttpStatus & \textit{Method} ini digunakan untuk mengubah attribut dari \textit{object} employee yang baru melalui \textit{webservice}\\
			\hline
			3 & getEmployee & void & id & Employee, HttpStatus & \textit{Method} ini digunakan untuk mengambil satu \textit{object} employee yang baru melalui \textit{webservice}\\
			\hline
			4 & getAllEmployees & void & - & $<$List$<$Employee$>$$>$, HttpStatus & \textit{Method} ini digunakan untuk mengambil semua \textit{list} \textit{object} employee yang baru melalui \textit{webservice}\\
			\hline
		\end{tabular}
	\end{adjustbox}
\end{table}
\begin{table}[H]
	\centering
	\small
	\begin{adjustbox}{width=1\textwidth}	
		\begin{tabular}{|p{0.4cm}|p{2.5cm}|p{0.9cm}|p{1.5cm}|p{3.4cm}|p{2.5cm}|}
			\hline
			5 & deleteEmployee & void & id & HttpStatus & \textit{Method} ini digunakan untuk menghapus satu \textit{object} employee yang baru melalui \textit{webservice}\\
			\hline
		\end{tabular}
	\end{adjustbox}
\end{table}
\begin{table}[H]
	\caption{Daftar \textit{Method} pada \textit{Class} JobspecialityController}
	\centering
	\small
	\begin{adjustbox}{width=1\textwidth}	
		\begin{tabular}{|p{0.4cm}|p{3.2cm}|p{0.8cm}|p{1.9cm}|p{2.85cm}|p{2.4cm}|}
			\hline
			\multirow{2}{*}{\textbf{No}} & \multirow{2}{*}{\textit{\textbf{Method}}} & \multicolumn{2}{c|}{\textit{\textbf{Input}}} & \multirow{2}{*}{\textit{\textbf{Output}}} & 
			\multirow{2}{*}{\textbf{Keterangan}}\\
			\cline{3-4}
			& & \textbf{Tipe} & \textbf{Variabel} & & \\
			\hline
			1 & addJobSpeciality & void & Jobspeciality & Jobspeciality, HttpStatus & \textit{Method} ini digunakan untuk membuat \textit{object} Jobspeciality yang baru melalui \textit{webservice}\\
			\hline
			2 & updateJobSpeciality & void & Jobspeciality & HttpStatus & \textit{Method} ini digunakan untuk mengubah attribut dari \textit{object} Jobspeciality yang baru melalui \textit{webservice}\\
			\hline
			3 & getJobSpeciality & void & id & Jobspeciality, HttpStatus & \textit{Method} ini digunakan untuk mengambil satu \textit{object} Jobspeciality yang baru melalui \textit{webservice}\\
			\hline
		\end{tabular}
	\end{adjustbox}
\end{table}
\begin{table}[H]
	\centering
	\small
	\begin{adjustbox}{width=1\textwidth}	
		\begin{tabular}{|p{0.4cm}|p{3.2cm}|p{1.cm}|p{1.7cm}|p{2.75cm}|p{2.5cm}|}
			\hline
			4 & getAllJobsSpeciality & void & - & $<$List
			$<$Jobspeciality$>$$>$, HttpStatus & \textit{Method} ini digunakan untuk mengambil semua \textit{list} \textit{object} Jobspeciality yang baru melalui \textit{webservice}\\
			\hline
			5 & deleteJobSpeciality & void & id & Jobspeciality, HttpStatus & \textit{Method} ini digunakan untuk menghapus satu \textit{object} CustomerGroup yang baru melalui \textit{webservice}\\
			\hline
		\end{tabular}
	\end{adjustbox}
\end{table}
\subsubsection{\textit{Package} Model Human Resource}
\textit{Package} Human Resource model berisi entitas-entitas penyusun dari \textit{service} Human Resource. Berikut ini merupakan daftar \textit{class} untuk \textit{package} model.
\begin{table}[H]
	\small
	\centering
	\caption{Daftar {\itshape Class} pada {\itshape Package} model}
	\begin{adjustbox}{width=1\textwidth}
		\begin{tabular}{| p {3 cm} | p {8 cm} | p {3 cm} |}
			\hline
			{\bfseries Package} & {\bfseries Class} & {\bfseries Jenis Class} \\
			\hline
			\multirow{12}{*}{Model} & Regency & {\itshape Class} \\
			& AddressType & {\itshape Class} \\
			& Contact & {\itshape Class} \\
			& ContactAddress & {\itshape Class} \\
			& ContactEducation & {\itshape Class} \\
			& Country & {\itshape Class} \\
			& Department & {\itshape Class} \\
			& Employee & {\itshape Class} \\
			& Employment & {\itshape Class} \\
			& Jobspeciality & {\itshape Class} \\
			& Province & {\itshape Class} \\
			& RoleInDept & {\itshape Class} \\
			\hline
		\end{tabular}
	\end{adjustbox}
\end{table}
\begin{table}[H]
	\caption{Daftar attribut pada \textit{Class} Employment}
	\centering
	\small
	\begin{adjustbox}{width=1\textwidth}	
		\begin{tabular}{|p{4cm} p{2.1cm} p{3cm} p{3.1cm}|}
			\hline
			\multicolumn{2}{|l}{\textbf{Variabel:}}&\multicolumn{2}{l|}{\textbf{Variabel:}}\\
			Long&systemid&Employee&employee\\
			String&ein&Date&hiredate\\
			BigInteger&basesalary&RoleInDept&roleIndept\\
			Integer&contracttype&String& sourceinstitution\\
			int&sourcetype&Date& createdate\\
			Date&lastupdate&Date& terminatedate\\
			\hline
		\end{tabular}
	\end{adjustbox}
\end{table}
\begin{table}[H]
	\caption{Daftar attribut pada \textit{Class} Employee}
	\centering
	\small
	\begin{adjustbox}{width=1\textwidth}	
		\begin{tabular}{|p{4cm} p{2.1cm} p{3cm} p{3.1cm}|}
			\hline
			\multicolumn{2}{|l}{\textbf{Variabel:}}&\multicolumn{2}{l|}{}\\
			String&m\_ein&&\\
			Jobspeciality&jobspeciality&&\\
			Collection$<$Employment$>$&employmentCollection&&\\
			\hline
		\end{tabular}
	\end{adjustbox}
\end{table}
\begin{table}[H]
	\caption{Daftar attribut pada \textit{Class} Departement}
	\centering
	\small
	\begin{adjustbox}{width=1\textwidth}	
		\begin{tabular}{|p{4cm} p{2.1cm} p{3cm} p{3.1cm}|}
			\hline
			\multicolumn{2}{|l}{\textbf{Variabel:}}&\multicolumn{2}{l|}{\textbf{}}\\
			int&systemid&&\\
			String&deptname&&\\
			String&memo&&\\
			Collection$<$RoleInDept$>$&roles&&\\
			Date&createdate&&\\
			Date&lastupdate&&\\
			\hline
		\end{tabular}
	\end{adjustbox}
\end{table}
\begin{table}[H]
	\caption{Daftar attribut pada \textit{Class} Contact}
	\centering
	\small
	\begin{adjustbox}{width=1\textwidth}	
		\begin{tabular}{|p{2cm} p{2.1cm} p{5cm} p{3.1cm}|}
			\hline
			\multicolumn{2}{|l}{\textbf{Variabel:}}&\multicolumn{2}{l|}{\textbf{Variabel:}}\\
			long&systemid&String&initial\\
			String&lastname&int&amountofchildren\\
			int&maritalstatus&List$<$ContactAddress$>$&arrAddress\\
			String&middlename&Collection$<$ContactEducation$>$&arrEdu\\
			Date&birthday&String&notes\\
			String&birthplace&String&officeext\\
			String&bloodtype&String&passportid\\
			double&bodyheight&byte[]&photo\\
			double&bodyweight&String&prefixtitle\\
			String&citizenid&String&sms\\
			String&citizentype&String&suffixtitle\\
			Country&citizenship&Date&sys\_createdate\\
			String&email&String&sys\_creator\\
			String&firstname&Date&sys\_lastupdate\\
			int&gender&String&sys\_lastupdater\\
			String&homephone&String&taxid\\
			String&messaging&String&web\\
			\hline
		\end{tabular}
	\end{adjustbox}
\end{table}
\begin{table}[H]
	\caption{Daftar attribut pada \textit{Class} ContactAddress}
	\centering
	\small
	\begin{adjustbox}{width=1\textwidth}	
		\begin{tabular}{|p{4cm} p{2.1cm} p{3cm} p{3.1cm}|}
			\hline
			\multicolumn{2}{|l}{\textbf{Variabel:}}&\multicolumn{2}{l|}{\textbf{Variabel:}}\\
			Long&systemid&String&postcode\\
			AddressType&addresstype&Regency&regency\\
			String&street&boolean&asbillingaddress\\
			String&fax&Date&sys\_createdate\\
			Contact&owner&String&sys\_creator\\
			String&phone&Date&sys\_lastupdate\\
			String&sys\_lastupdater&&\\
			\hline
		\end{tabular}
	\end{adjustbox}
\end{table}
\begin{table}[H]
	\caption{Daftar attribut pada \textit{Class} AddressType}
	\centering
	\small
	\begin{adjustbox}{width=1\textwidth}	
		\begin{tabular}{|p{4cm} p{2.1cm} p{3cm} p{3.1cm}|}
			\hline
			\multicolumn{2}{|l}{\textbf{Variabel:}}&\multicolumn{2}{l|}{\textbf{}}\\
			Integer&systemid&&\\
			String&memo&&\\
			String&typename&&\\
			\hline
		\end{tabular}
	\end{adjustbox}
\end{table}
\begin{table}[H]
	\caption{Daftar attribut pada \textit{Class} ContactEducation}
	\centering
	\small
	\begin{adjustbox}{width=1\textwidth}	
		\begin{tabular}{|p{3cm} p{3.1cm} p{3cm} p{3.1cm}|}
			\hline
			\multicolumn{2}{|l}{\textbf{Variabel:}}&\multicolumn{2}{l|}{\textbf{\textbf{Variabel:}}}\\
			Date&finishrolling&Contact&owner\\
			String&gpa&String&sponsors\\
			int&grade&Date&startrolling\\
			String&institutionaladdress&Date&sys\_createdate\\
			String&institutionname&int&sys\_creator\\
			String&major&Date&sys\_lastupdate\\
			String&notes&int&sys\_lastupdater\\
			\hline
		\end{tabular}
	\end{adjustbox}
\end{table}
\begin{table}[H]
	\caption{Daftar attribut pada \textit{Class} Country}
	\centering
	\small
	\begin{adjustbox}{width=1\textwidth}	
		\begin{tabular}{|p{4cm} p{2.1cm} p{3cm} p{3.1cm}|}
			\hline
			\multicolumn{2}{|l}{\textbf{Variabel:}}&\multicolumn{2}{l|}{\textbf{}}\\
			Long&systemid&&\\
			String&name&&\\
			\hline
		\end{tabular}
	\end{adjustbox}
\end{table}
\begin{table}[H]
	\caption{Daftar attribut pada \textit{Class} Jobspeciality}
	\centering
	\small
	\begin{adjustbox}{width=1\textwidth}	
		\begin{tabular}{|p{5cm} p{3.1cm} p{2cm} p{2.1cm}|}
			\hline
			\multicolumn{2}{|l}{\textbf{Variabel:}}&\multicolumn{2}{l|}{\textbf{}}\\
			Long&systemid&&\\
			String&specialityName&&\\
			String&memo&&\\
			\hline
		\end{tabular}
	\end{adjustbox}
\end{table}
\begin{table}[H]
	\caption{Daftar attribut pada \textit{Class} RoleInDept}
	\centering
	\small
	\begin{adjustbox}{width=1\textwidth}	
		\begin{tabular}{|p{5cm} p{3.1cm} p{2cm} p{2.1cm}|}
			\hline
			\multicolumn{2}{|l}{\textbf{Variabel:}}&\multicolumn{2}{l|}{\textbf{}}\\
			int&systemid&&\\
			Department&department&&\\
			String&rolename&&\\
			String&abbreviation&&\\
			RoleInDept&parentRole&&\\
			String&memo&&\\
			\hline
		\end{tabular}
	\end{adjustbox}
\end{table}
\begin{table}[H]
	\caption{Daftar attribut pada \textit{Class} Province}
	\centering
	\small
	\begin{adjustbox}{width=1\textwidth}	
		\begin{tabular}{|p{5cm} p{3.1cm} p{2cm} p{2.1cm}|}
			\hline
			\multicolumn{2}{|l}{\textbf{Variabel:}}&\multicolumn{2}{l|}{\textbf{}}\\
			Long&systemid&&\\
			Country&countryCode&&\\
			String&name&&\\
			Set$<$Regency$>$&regencySet&&\\
			\hline
		\end{tabular}
	\end{adjustbox}
\end{table}
\begin{table}[H]
	\caption{Daftar attribut pada \textit{Class} Regency}
	\centering
	\small
	\begin{adjustbox}{width=1\textwidth}	
		\begin{tabular}{|p{5cm} p{3.1cm} p{2cm} p{2.1cm}|}
			\hline
			\multicolumn{2}{|l}{\textbf{Variabel:}}&\multicolumn{2}{l|}{\textbf{}}\\
			Long&systemid&&\\
			Province&provId&&\\
			String&name&&\\
			\hline
		\end{tabular}
	\end{adjustbox}
\end{table}
\subsubsection{\textit{Package} Repository}
\textit{Package} repository berisi \textit{class-class} yang menjadi representasi model dari tabel-tabel yang berada pada basis data. Berikut ini merupakan \textit{class-class} yang terdapat pada \textit{package} repository.
\begin{table}[H]
	\small
	\centering
	\caption{Daftar {\itshape Class} pada {\itshape Package} repository}
	\begin{adjustbox}{width=1\textwidth}
		\begin{tabular}{| p {3 cm} | p {8 cm} | p {3 cm} |}
			\hline
			{\bfseries Package} & {\bfseries Class} & {\bfseries Jenis Class} \\
			\hline
			\multirow{2}{*}{repository} & EmployeeRepository & {\itshape Interface} \\
			& JobspecialityRepository & {\itshape Interface} \\
			\hline
		\end{tabular}
	\end{adjustbox}
\end{table}
\subsubsection{\textit{Package} Service}
\textit{Package} service berisi \textit{class-class} yang menginisialisasi \textit{web service} human resource yang akan terbentuk dan digunakan pada \textit{package} controller. Berikut ini merupakan \textit{class-class} yang terdapat pada \textit{package} service.
\begin{table}[H]
	\small
	\centering
	\caption{Daftar {\itshape Class} pada {\itshape Package} service}
	\begin{adjustbox}{width=1\textwidth}
		\begin{tabular}{| p {3 cm} | p {8 cm} | p {3 cm} |}
			\hline
			{\bfseries Package} & {\bfseries Class} & {\bfseries Jenis Class} \\
			\hline
			\multirow{5}{*}{service} & CRUDService & {\itshape Interface} \\
			& EmployeeService & {\itshape Interface} \\
			& JobspecialityService & {\itshape Interface} \\
			& DefaultEmployeeService & {\itshape Class} \\
			& DefaultJobspecialityService & {\itshape Class} \\
			\hline
		\end{tabular}
	\end{adjustbox}
\end{table}
\subsection{\textit{Project} \textit{Medical Unit}}
\textit{Project Medical Unit} berisi susunan \textit{class} pembentuk \textit{webservice Medical Unit}. Dalam project ini terdapat 4 buah \textit{package} yang berfungsi untuk memisahkan \textit{class} sesuai dengan fungsinya masing-masing. Ke empat \textit{package} ini yaitu \textit{package} controller yang berisi method-method yang dijalankan pada \textit{web service}, \textit{package} model yang berisi entitas dan tabel, \textit{package} repository yang menghubungkan entitas di \textit{package} model dengan database, dan \textit{package} service yang menginisialisasi \textit{webserviec} yang akan terbentuk. Pada bagian ini akan dijelaskan mengenai \textit{class} dan \textit{method} yang digunakan pada pengembangan \textit{ webservice Unit Medis}:
\subsubsection{\textit{Package} Controller}
\textit{Package} controller berisi method API yang dapat digunakan untuk berkomunikasi dengan \textit{service} Human Resource. \textit{Class} yang terdapat dalam \textit{package} ini adalah \textit{class} UnitmedisController yang berisi 5 buah fungsi API. Di bawah ini merupakan daftar \textit{class} untuk \textit{package} controller pada \textit{webservice} Unit Medis.
\begin{table}[H]
	\small
	\centering
	\caption{Daftar {\itshape Class} pada {\itshape Package} Controller}
	\begin{adjustbox}{width=1\textwidth}
		\begin{tabular}{| p {3 cm} | p {8 cm} | p {3 cm} |}
			\hline
			{\bfseries Package} & {\bfseries Class} & {\bfseries Jenis Class} \\
			\hline
			\multirow{1}{*}{Controller} & UnitmedisController & {\itshape Class} \\
			\hline
		\end{tabular}
	\end{adjustbox}
\end{table}
\begin{table}[H]
	\caption{Daftar \textit{Method} pada \textit{Class} UnitmedisController}
	\centering
	\small
	\begin{adjustbox}{width=1\textwidth}	
		\begin{tabular}{|p{0.4cm}|p{2.8cm}|p{0.9cm}|p{1.8cm}|p{2.8cm}|p{2.5cm}|}
			\hline
			\multirow{2}{*}{\textbf{No}} & \multirow{2}{*}{\textit{\textbf{Method}}} & \multicolumn{2}{c|}{\textit{\textbf{Input}}} & \multirow{2}{*}{\textit{\textbf{Output}}} & 
			\multirow{2}{*}{\textbf{Keterangan}}\\
			\cline{3-4}
			& & \textbf{Tipe} & \textbf{Variabel} & & \\
			\hline
			1 & addMedicalUnit & void & MedicalUnit & MedicalUnit, HttpStatus & \textit{Method} ini digunakan untuk membuat \textit{object} unit medis yang baru melalui \textit{webservice}\\
			\hline
			2 & updateMedicalUnit & void & MedicalUnit & HttpStatus & \textit{Method} ini digunakan untuk mengubah attribut dari \textit{object} unit medis yang baru melalui \textit{webservice}\\
			\hline
			3 & getMedicalUnit & void & id & MedicalUnit, HttpStatus & \textit{Method} ini digunakan untuk mengambil satu \textit{object} unit medis yang baru melalui \textit{webservice}\\
			\hline
			4 & getAllMedicalUnit & void & - & $<$List
			$<$MedicalUnit$>$$>$, HttpStatus & \textit{Method} ini digunakan untuk mengambil semua \textit{list} \textit{object} unit medis yang baru melalui \textit{webservice}\\
			\hline
			5 & deleteMedicalUnit & void & id & HttpStatus & \textit{Method} ini digunakan untuk menghapus satu \textit{object} unit medis yang baru melalui \textit{webservice}\\
			\hline
		\end{tabular}
	\end{adjustbox}
\end{table}
\subsubsection{\textit{Package} Model Medical Unit}
\textit{Package} Medical Unit model berisi entitas-entitas penyusun dari \textit{service} Medical Unit. Berikut ini merupakan daftar \textit{class} untuk \textit{package} model.
\begin{table}[H]
	\small
	\centering
	\caption{Daftar {\itshape Class} pada {\itshape Package} model}
	\begin{adjustbox}{width=1\textwidth}
		\begin{tabular}{| p {3 cm} | p {8 cm} | p {3 cm} |}
			\hline
			{\bfseries Package} & {\bfseries Class} & {\bfseries Jenis Class} \\
			\hline
			\multirow{16}{*}{Model} & MedicalUnit & {\itshape Class} \\
			& WorkingUnit & {\itshape Class} \\
			& WorkingUnitMembership & {\itshape Class} \\
			& WorkingUnitMembershipPK & {\itshape Class} \\
			& WUGroup & {\itshape Class} \\
			& AddressType & {\itshape Class} \\
			& Contact & {\itshape Class} \\
			& ContactAddress & {\itshape Class} \\
			& ContactEducation & {\itshape Class} \\
			& Country & {\itshape Class} \\
			& Department & {\itshape Class} \\
			& Employee & {\itshape Class} \\
			& Employment & {\itshape Class} \\
			& Jobspeciality & {\itshape Class} \\
			& Province & {\itshape Class} \\
			& RoleInDept & {\itshape Class} \\
			& Regency & {\itshape Class} \\
			\hline
		\end{tabular}
	\end{adjustbox}
\end{table}
\begin{table}[H]
	\caption{Daftar attribut pada \textit{Class} WorkingUnit}
	\centering
	\small
	\begin{adjustbox}{width=1\textwidth}	
		\begin{tabular}{|p{4cm} p{2.1cm} p{3cm} p{3.1cm}|}
			\hline
			\multicolumn{2}{|l}{\textbf{Variabel:}}&\multicolumn{2}{l|}{\textbf{Variabel:}}\\
			Long&systemid&&\\
			WUGroup&wugroup&&\\
			String&workunit\_name&&\\
			String&memo&&\\
			Calendar&createdate&&\\
			Calendar&lastupdate&&\\
			\hline
		\end{tabular}
	\end{adjustbox}
\end{table}
\begin{table}[H]
	\caption{Daftar attribut pada \textit{Class} MedicalUnit}
	\centering
	\small
	\begin{adjustbox}{width=1\textwidth}	
		\begin{tabular}{|p{4cm} p{2.1cm} p{3cm} p{3.1cm}|}
			\hline
			\multicolumn{2}{|l}{\textbf{Variabel:}}&\multicolumn{2}{l|}{}\\
			Long&systemid&&\\
			\hline
		\end{tabular}
	\end{adjustbox}
\end{table}
\begin{table}[H]
	\caption{Daftar attribut pada \textit{Class} WorkingUnitMembership}
	\centering
	\small
	\begin{adjustbox}{width=1\textwidth}	
		\begin{tabular}{|p{4cm} p{2.1cm} p{3cm} p{3.1cm}|}
			\hline
			\multicolumn{2}{|l}{\textbf{Variabel:}}&\multicolumn{2}{l|}{}\\
			WorkingUnit&workingunit&&\\
			Employee&employee&&\\
			boolean&workingunitpersonincharge&&\\
			String&description&&\\
			Calendar&sys\_createdate&&\\
			Calendar&syssys\_lastupdate&&\\
			\hline
		\end{tabular}
	\end{adjustbox}
\end{table}
\begin{table}[H]
	\caption{Daftar attribut pada \textit{Class} WUGroup}
	\centering
	\small
	\begin{adjustbox}{width=1\textwidth}	
		\begin{tabular}{|p{4cm} p{2.1cm} p{3cm} p{3.1cm}|}
			\hline
			\multicolumn{2}{|l}{\textbf{Variabel:}}&\multicolumn{2}{l|}{}\\
			Long&systemid&&\\
			String&groupname&&\\
			String&memo&&\\
			\hline
		\end{tabular}
	\end{adjustbox}
\end{table}
\begin{table}[H]
	\caption{Daftar attribut pada \textit{Class} WorkingUnitMembershipPK}
	\centering
	\small
	\begin{adjustbox}{width=1\textwidth}	
		\begin{tabular}{|p{4cm} p{2.1cm} p{3cm} p{3.1cm}|}
			\hline
			\multicolumn{2}{|l}{\textbf{Variabel:}}&\multicolumn{2}{l|}{}\\
			Long&workingunit&&\\
			Long&employee&&\\
			\hline
		\end{tabular}
	\end{adjustbox}
\end{table}
\begin{table}[H]
	\caption{Daftar attribut pada \textit{Class} Employment}
	\centering
	\small
	\begin{adjustbox}{width=1\textwidth}	
		\begin{tabular}{|p{4cm} p{2.1cm} p{3cm} p{3.1cm}|}
			\hline
			\multicolumn{2}{|l}{\textbf{Variabel:}}&\multicolumn{2}{l|}{\textbf{Variabel:}}\\
			Long&systemid&Employee&employee\\
			String&ein&Date&hiredate\\
			BigInteger&basesalary&RoleInDept&roleIndept\\
			Integer&contracttype&String& sourceinstitution\\
			int&sourcetype&Date& createdate\\
			Date&lastupdate&Date& terminatedate\\
			\hline
		\end{tabular}
	\end{adjustbox}
\end{table}
\begin{table}[H]
	\caption{Daftar attribut pada \textit{Class} Employee}
	\centering
	\small
	\begin{adjustbox}{width=1\textwidth}	
		\begin{tabular}{|p{4cm} p{2.1cm} p{3cm} p{3.1cm}|}
			\hline
			\multicolumn{2}{|l}{\textbf{Variabel:}}&\multicolumn{2}{l|}{}\\
			String&m\_ein&&\\
			Jobspeciality&jobspeciality&&\\
			Collection$<$Employment$>$&employmentCollection&&\\
			\hline
		\end{tabular}
	\end{adjustbox}
\end{table}
\begin{table}[H]
	\caption{Daftar attribut pada \textit{Class} Departement}
	\centering
	\small
	\begin{adjustbox}{width=1\textwidth}	
		\begin{tabular}{|p{4cm} p{2.1cm} p{3cm} p{3.1cm}|}
			\hline
			\multicolumn{2}{|l}{\textbf{Variabel:}}&\multicolumn{2}{l|}{\textbf{}}\\
			int&systemid&&\\
			String&deptname&&\\
			String&memo&&\\
			Collection$<$RoleInDept$>$&roles&&\\
			Date&createdate&&\\
			Date&lastupdate&&\\
			\hline
		\end{tabular}
	\end{adjustbox}
\end{table}
\begin{table}[H]
\caption{Daftar attribut pada \textit{Class} ContactAddress}
\centering
\small
\begin{adjustbox}{width=1\textwidth}	
	\begin{tabular}{|p{4cm} p{2.1cm} p{3cm} p{3.1cm}|}
		\hline
		\multicolumn{2}{|l}{\textbf{Variabel:}}&\multicolumn{2}{l|}{\textbf{Variabel:}}\\
		Long&systemid&String&postcode\\
		AddressType&addresstype&Regency&regency\\
		String&street&boolean&asbillingaddress\\
		String&fax&Date&sys\_createdate\\
		Contact&owner&String&sys\_creator\\
		String&phone&Date&sys\_lastupdate\\
		String&sys\_lastupdater&&\\
		\hline
	\end{tabular}
\end{adjustbox}
\end{table}
\begin{table}[H]
	\caption{Daftar attribut pada \textit{Class} Contact}
	\centering
	\small
	\begin{adjustbox}{width=1\textwidth}	
		\begin{tabular}{|p{2cm} p{2.1cm} p{5cm} p{3.1cm}|}
			\hline
			\multicolumn{2}{|l}{\textbf{Variabel:}}&\multicolumn{2}{l|}{\textbf{Variabel:}}\\
			long&systemid&String&initial\\
			String&lastname&int&amountofchildren\\
			int&maritalstatus&List$<$ContactAddress$>$&arrAddress\\
			String&middlename&Collection$<$ContactEducation$>$&arrEdu\\
			Date&birthday&String&notes\\
			String&birthplace&String&officeext\\
			String&bloodtype&String&passportid\\
			double&bodyheight&byte[]&photo\\
			double&bodyweight&String&prefixtitle\\
			String&citizenid&String&sms\\
			String&citizentype&String&suffixtitle\\
			Country&citizenship&Date&sys\_createdate\\
			String&email&String&sys\_creator\\
			String&firstname&Date&sys\_lastupdate\\
			int&gender&String&sys\_lastupdater\\
			String&homephone&String&taxid\\
			String&messaging&String&web\\
			\hline
		\end{tabular}
	\end{adjustbox}
\end{table}
\begin{table}[H]
	\caption{Daftar attribut pada \textit{Class} AddressType}
	\centering
	\small
	\begin{adjustbox}{width=1\textwidth}	
		\begin{tabular}{|p{4cm} p{2.1cm} p{3cm} p{3.1cm}|}
			\hline
			\multicolumn{2}{|l}{\textbf{Variabel:}}&\multicolumn{2}{l|}{\textbf{}}\\
			Integer&systemid&&\\
			String&memo&&\\
			String&typename&&\\
			\hline
		\end{tabular}
	\end{adjustbox}
\end{table}
\begin{table}[H]
	\caption{Daftar attribut pada \textit{Class} ContactEducation}
	\centering
	\small
	\begin{adjustbox}{width=1\textwidth}	
		\begin{tabular}{|p{3cm} p{3.1cm} p{3cm} p{3.1cm}|}
			\hline
			\multicolumn{2}{|l}{\textbf{Variabel:}}&\multicolumn{2}{l|}{\textbf{\textbf{Variabel:}}}\\
			Date&finishrolling&Contact&owner\\
			String&gpa&String&sponsors\\
			int&grade&Date&startrolling\\
			String&institutionaladdress&Date&sys\_createdate\\
			String&institutionname&int&sys\_creator\\
			String&major&Date&sys\_lastupdate\\
			String&notes&int&sys\_lastupdater\\
			\hline
		\end{tabular}
	\end{adjustbox}
\end{table}
\begin{table}[H]
	\caption{Daftar attribut pada \textit{Class} Country}
	\centering
	\small
	\begin{adjustbox}{width=1\textwidth}	
		\begin{tabular}{|p{4cm} p{2.1cm} p{3cm} p{3.1cm}|}
			\hline
			\multicolumn{2}{|l}{\textbf{Variabel:}}&\multicolumn{2}{l|}{\textbf{}}\\
			Long&systemid&&\\
			String&name&&\\
			\hline
		\end{tabular}
	\end{adjustbox}
\end{table}
\begin{table}[H]
	\caption{Daftar attribut pada \textit{Class} Jobspeciality}
	\centering
	\small
	\begin{adjustbox}{width=1\textwidth}	
		\begin{tabular}{|p{5cm} p{3.1cm} p{2cm} p{2.1cm}|}
			\hline
			\multicolumn{2}{|l}{\textbf{Variabel:}}&\multicolumn{2}{l|}{\textbf{}}\\
			Long&systemid&&\\
			String&specialityName&&\\
			String&memo&&\\
			\hline
		\end{tabular}
	\end{adjustbox}
\end{table}
\begin{table}[H]
	\caption{Daftar attribut pada \textit{Class} RoleInDept}
	\centering
	\small
	\begin{adjustbox}{width=1\textwidth}	
		\begin{tabular}{|p{5cm} p{3.1cm} p{2cm} p{2.1cm}|}
			\hline
			\multicolumn{2}{|l}{\textbf{Variabel:}}&\multicolumn{2}{l|}{\textbf{}}\\
			int&systemid&&\\
			Department&department&&\\
			String&rolename&&\\
			String&abbreviation&&\\
			RoleInDept&parentRole&&\\
			String&memo&&\\
			\hline
		\end{tabular}
	\end{adjustbox}
\end{table}
\begin{table}[H]
	\caption{Daftar attribut pada \textit{Class} Province}
	\centering
	\small
	\begin{adjustbox}{width=1\textwidth}	
		\begin{tabular}{|p{5cm} p{3.1cm} p{2cm} p{2.1cm}|}
			\hline
			\multicolumn{2}{|l}{\textbf{Variabel:}}&\multicolumn{2}{l|}{\textbf{}}\\
			Long&systemid&&\\
			Country&countryCode&&\\
			String&name&&\\
			Set$<$Regency$>$&regencySet&&\\
			\hline
		\end{tabular}
	\end{adjustbox}
\end{table}
\begin{table}[H]
	\caption{Daftar attribut pada \textit{Class} Regency}
	\centering
	\small
	\begin{adjustbox}{width=1\textwidth}	
		\begin{tabular}{|p{5cm} p{3.1cm} p{2cm} p{2.1cm}|}
			\hline
			\multicolumn{2}{|l}{\textbf{Variabel:}}&\multicolumn{2}{l|}{\textbf{}}\\
			Long&systemid&&\\
			Province&provId&&\\
			String&name&&\\
			\hline
		\end{tabular}
	\end{adjustbox}
\end{table}
\subsubsection{\textit{Package} Repository}
\textit{Package} repository berisi \textit{class-class} yang menjadi representasi model dari tabel-tabel yang berada pada basis data. Berikut ini merupakan \textit{class-class} yang terdapat pada \textit{package} repository.
\begin{table}[H]
	\small
	\centering
	\caption{Daftar {\itshape Class} pada {\itshape Package} repository}
	\begin{adjustbox}{width=1\textwidth}
		\begin{tabular}{| p {3 cm} | p {8 cm} | p {3 cm} |}
			\hline
			{\bfseries Package} & {\bfseries Class} & {\bfseries Jenis Class} \\
			\hline
			\multirow{1}{*}{repository} & UnitmedisRepository & {\itshape Interface} \\
			\hline
		\end{tabular}
	\end{adjustbox}
\end{table}
\subsubsection{\textit{Package} Service}
\textit{Package} service berisi \textit{class-class} yang menginisialisasi \textit{web service} human resource yang akan terbentuk dan digunakan pada \textit{package} controller. Berikut ini merupakan \textit{class-class} yang terdapat pada \textit{package} service.
\begin{table}[H]
	\small
	\centering
	\caption{Daftar {\itshape Class} pada {\itshape Package} service}
	\begin{adjustbox}{width=1\textwidth}
		\begin{tabular}{| p {3 cm} | p {8 cm} | p {3 cm} |}
			\hline
			{\bfseries Package} & {\bfseries Class} & {\bfseries Jenis Class} \\
			\hline
			\multirow{3}{*}{service} & CRUDService & {\itshape Interface} \\
			& UnitmedisService & {\itshape Interface} \\
			& DefaultUnitmedisService & {\itshape Class} \\
			\hline
		\end{tabular}
	\end{adjustbox}
\end{table}
\subsection{\textit{Project} Rawat Jalan}
\textit{Project} Rawat Jalan berisi susunan \textit{class} pembentuk \textit{webservice} Rawat Jalan. Dalam project ini terdapat 4 buah \textit{package} yang berfungsi untuk memisahkan \textit{class} sesuai dengan fungsinya masing-masing. Ke empat \textit{package} ini yaitu \textit{package} controller yang berisi method-method yang dijalankan pada \textit{web service}, \textit{package} model yang berisi entitas dan tabel, \textit{package} repository yang menghubungkan entitas di \textit{package} model dengan database, dan \textit{package} service yang menginisialisasi \textit{webserviec} yang akan terbentuk. Pada bagian ini akan dijelaskan mengenai \textit{class} dan \textit{method} yang digunakan pada pengembangan \textit{webservice} Rawat Jalan:
\subsubsection{\textit{Package} Controller}

\textit{Package} controller berisi method API yang dapat digunakan untuk berkomunikasi dengan \textit{service} Rawat Jalan. \textit{Class} yang terdapat dalam \textit{package} ini adalah \textit{class} OutpatientWUQueueController yang berisi 5 buah fungsi API. Di bawah ini merupakan daftar \textit{class} untuk \textit{package} controller pada \textit{webservice} Rawat Jalan.
\begin{table}[H]
	\small
	\centering
	\caption{Daftar {\itshape Class} pada {\itshape Package} Controller}
	\begin{adjustbox}{width=1\textwidth}
		\begin{tabular}{| p {3 cm} | p {8 cm} | p {3 cm} |}
			\hline
			{\bfseries Package} & {\bfseries Class} & {\bfseries Jenis Class} \\
			\hline
			\multirow{1}{*}{Controller} & OutpatientWUQueueController & {\itshape Class} \\
			\hline
		\end{tabular}
	\end{adjustbox}
\end{table}
\begin{table}[H]
	\caption{Daftar \textit{Method} pada \textit{Class} OutpatientWUQueueController}
	\centering
	\small
	\begin{adjustbox}{width=1\textwidth}	
		\begin{tabular}{|p{0.4cm}|p{2.8cm}|p{0.9cm}|p{1.8cm}|p{2.8cm}|p{2.5cm}|}
			\hline
			\multirow{2}{*}{\textbf{No}} & \multirow{2}{*}{\textit{\textbf{Method}}} & \multicolumn{2}{c|}{\textit{\textbf{Input}}} & \multirow{2}{*}{\textit{\textbf{Output}}} & 
			\multirow{2}{*}{\textbf{Keterangan}}\\
			\cline{3-4}
			& & \textbf{Tipe} & \textbf{Variabel} & & \\
			\hline
			1 & addOutpatient
			Queue & void & Outpatient
			WUQueue & Outpatient
			WUQueue, HttpStatus & \textit{Method} ini digunakan untuk membuat \textit{object} rawat jalan yang baru melalui \textit{webservice}\\
			\hline
			2 & updateOutpatient
			Queue & void & Outpatient
			WUQueue & HttpStatus & \textit{Method} ini digunakan untuk mengubah attribut dari \textit{object} rawat jalan yang baru melalui \textit{webservice}\\
			\hline
			3 & getOutpatient
			Queue & void & id & Outpatient
			WUQueue, HttpStatus & \textit{Method} ini digunakan untuk mengambil satu \textit{object} rawat jalan yang baru melalui \textit{webservice}\\
			\hline
			4 & getAllOut
			patientQueue & void & - & $<$List
			$<$OutpatientWU
			Queue$>$$>$, HttpStatus & \textit{Method} ini digunakan untuk mengambil semua \textit{list} \textit{object} rawat jalan yang baru melalui \textit{webservice}\\
			\hline
			5 & deleteOutpatient
			Queue & void & id & HttpStatus & \textit{Method} ini digunakan untuk menghapus satu \textit{object} rawat jalan yang baru melalui \textit{webservice}\\
			\hline
		\end{tabular}
	\end{adjustbox}
\end{table}
\subsubsection{\textit{Package} Model Rawat Jalan}
\textit{Package} Rawat Jalan model berisi entitas-entitas penyusun dari \textit{service} Rawat Jalan. Berikut ini merupakan daftar \textit{class} untuk \textit{package} model.
\begin{table}[H]
	\small
	\centering
	\caption{Daftar {\itshape Class} pada {\itshape Package} model}
	\begin{adjustbox}{width=1\textwidth}
		\begin{tabular}{| p {3 cm} | p {8 cm} | p {3 cm} |}
			\hline
			{\bfseries Package} & {\bfseries Class} & {\bfseries Jenis Class} \\
			\hline
			\multirow{26}{*}{Model} & OutpatientWUQueue & {\itshape Class} \\
			& OutpatientWUQueueHistory & {\itshape Class} \\
			& OutpatientWUQueuePK & {\itshape Class} \\
			& WorkingUnit & {\itshape Class} \\
			& WorkingUnitMembership & {\itshape Class} \\
			& WorkingUnitMembershipPK & {\itshape Class} \\
			& WUGroup & {\itshape Class} \\
			& AddressType & {\itshape Class} \\
			& Contact & {\itshape Class} \\
			& ContactAddress & {\itshape Class} \\
			& ContactEducation & {\itshape Class} \\
			& Customer & {\itshape Class} \\
			& Customergroup & {\itshape Class} \\
			& HealthConsumer & {\itshape Class} \\
			& Insurance & {\itshape Class} \\
			& InsuranceBridgeConf & {\itshape Class} \\
			& Profession & {\itshape Class} \\
			& Country & {\itshape Class} \\
			& Department & {\itshape Class} \\
			& Employee & {\itshape Class} \\
			& Employment & {\itshape Class} \\
			& Jobspeciality & {\itshape Class} \\
			& Province & {\itshape Class} \\
			& RoleInDept & {\itshape Class} \\
			& Regency & {\itshape Class} \\
			& MedicalUnit & {\itshape Class} \\
			\hline
		\end{tabular}
	\end{adjustbox}
\end{table}
\begin{table}[H]
	\caption{Daftar attribut pada \textit{Class} OutpatientWUQueue}
	\centering
	\small
	\begin{adjustbox}{width=1\textwidth}	
		\begin{tabular}{|p{4.5cm} p{2.1cm} p{2.5cm} p{3.1cm}|}
			\hline
			\multicolumn{2}{|l}{\textbf{Variabel:}}&\multicolumn{2}{l|}{\textbf{\textbf{Variabel:}}}\\
			OutpatientWUQueue&hc&Calendar&syscreatetime\\
			MedicalUnit&medunit&Calendar&syslastupdate\\
			String&regis\_no&Insurance&insurance\\
			Date&registertimee&String&insuranceno\\
			Date&finishtime&String&insurancetype\\
			int&priority&String&via\\
			String&memo&String&rujukan\_tipe\\
			String&rujukan\_doc\_no&String&rujukan\_nama\_asal\\
			OutpatientWUQueueHistory&history&String&insuranceprg\\
			\hline
		\end{tabular}
	\end{adjustbox}
\end{table}
\begin{table}[H]
	\caption{Daftar attribut pada \textit{Class} outpatienthistory}
	\centering
	\small
	\begin{adjustbox}{width=1\textwidth}	
		\begin{tabular}{|p{3.5cm} p{3.1cm} p{2.5cm} p{3.1cm}|}
			\hline
			\multicolumn{2}{|l}{\textbf{Variabel:}}&\multicolumn{2}{l|}{\textbf{\textbf{Variabel:}}}\\
			OutpatientWUQueue&hc&Calendar&syscreatetime\\
			MedicalUnit&medunit&Calendar&syslastupdate\\
			String&regis\_no&Insurance&insurance\\
			Date&registertimee&String&insuranceno\\
			Date&finishtime&String&insurancetype\\
			int&priority&String&via\\
			String&memo&String&rujukan\_tipe\\
			String&rujukan\_nama\_asal&String&rujukan\_doc\_no\\
			String&insuranceprg&&\\
			\hline
		\end{tabular}
	\end{adjustbox}
\end{table}
\begin{table}[H]
	\caption{Daftar attribut pada \textit{Class} OutpatientWUQueuePK}
	\centering
	\small
	\begin{adjustbox}{width=1\textwidth}	
		\begin{tabular}{|p{4cm} p{2.1cm} p{3cm} p{3.1cm}|}
			\hline
			\multicolumn{2}{|l}{\textbf{Variabel:}}&\multicolumn{2}{l|}{\textbf{}}\\
			long&hc&&\\
			long&medunit&&\\
			\hline
		\end{tabular}
	\end{adjustbox}
\end{table}
\begin{table}[H]
	\caption{Daftar attribut pada \textit{Class} HealthConsumer}
	\centering
	\small
	\begin{adjustbox}{width=1\textwidth}	
		\begin{tabular}{|p{4cm} p{2.1cm} p{3cm} p{3.1cm}|}
			\hline
			\multicolumn{2}{|l}{\textbf{Variabel:}}&\multicolumn{2}{l|}{\textbf{}}\\
			String&regis\_no&&\\
			Profession&id\_prof&&\\
			Insurance&insurance&&\\
			String&insurance\_type&&\\
			String&insurance\_no&&\\
			String&insurance\_prg&&\\
			\hline
		\end{tabular}
	\end{adjustbox}
\end{table}
\begin{table}[H]
	\caption{Daftar attribut pada \textit{Class} Insurance}
	\centering
	\small
	\begin{adjustbox}{width=1\textwidth}	
		\begin{tabular}{|p{5cm} p{3.1cm} p{2cm} p{2.1cm}|}
			\hline
			\multicolumn{2}{|l}{\textbf{Variabel:}}&\multicolumn{2}{l|}{\textbf{}}\\
			List$<$InsuranceBridgeConf$>$&insuranceBrigdeConfList&&\\
			int&systemid&&\\
			String&insurance&&\\
			String&memo&&\\
			boolean&active&&\\
			boolean&sysbuiltin&&\\
			\hline
		\end{tabular}
	\end{adjustbox}
\end{table}
\begin{table}[H]
	\caption{Daftar attribut pada \textit{Class} InsuranceBridgeConf}
	\centering
	\small
	\begin{adjustbox}{width=1\textwidth}	
		\begin{tabular}{|p{5cm} p{3.1cm} p{2cm} p{2.1cm}|}
			\hline
			\multicolumn{2}{|l}{\textbf{Variabel:}}&\multicolumn{2}{l|}{\textbf{}}\\
			Long&systemid&&\\
			String&field&&\\
			String&def\_val&&\\
			String&memo&&\\
			\hline
		\end{tabular}
	\end{adjustbox}
\end{table}
\begin{table}[H]
	\caption{Daftar attribut pada \textit{Class} Profession}
	\centering
	\small
	\begin{adjustbox}{width=1\textwidth}	
		\begin{tabular}{|p{5cm} p{3.1cm} p{2cm} p{2.1cm}|}
			\hline
			\multicolumn{2}{|l}{\textbf{Variabel:}}&\multicolumn{2}{l|}{\textbf{}}\\
			Integer&systemid&&\\
			String&profname&&\\
			String&memo&&\\
			Calendar&sys\_createdate&&\\
			Calendar&last\_createdate&&\\
			\hline
		\end{tabular}
	\end{adjustbox}
\end{table}
\begin{table}[H]
	\caption{Daftar attribut pada \textit{Class} Customer}
	\centering
	\small
	\begin{adjustbox}{width=1\textwidth}	
		\begin{tabular}{|p{4cm} p{2.1cm} p{3cm} p{3.1cm}|}
			\hline
			\multicolumn{2}{|l}{\textbf{Variabel:}}&\multicolumn{2}{l|}{\textbf{Variabel:}}\\
			boolean&cust\_no&Date&registrationdate\\
			boolean&active&int&idPriceLevel\\
			double&creditlimit&double&disc\\
			Customergroup&customergroup&&\\
			\hline
		\end{tabular}
	\end{adjustbox}
\end{table}

\begin{table}[H]
	\caption{Daftar attribut pada \textit{Class} Customergroup}
	\centering
	\small
	\begin{adjustbox}{width=1\textwidth}	
		\begin{tabular}{|p{4cm} p{2.1cm} p{3cm} p{3.1cm}|}
			\hline
			\multicolumn{2}{|l}{\textbf{Variabel:}}&\multicolumn{2}{l|}{}\\
			long&systemid&&\\
			String&groupname&&\\
			String&memo&&\\
			\hline
		\end{tabular}
	\end{adjustbox}
\end{table}
\begin{table}[H]
	\caption{Daftar attribut pada \textit{Class} WorkingUnit}
	\centering
	\small
	\begin{adjustbox}{width=1\textwidth}	
		\begin{tabular}{|p{4cm} p{2.1cm} p{3cm} p{3.1cm}|}
			\hline
			\multicolumn{2}{|l}{\textbf{Variabel:}}&\multicolumn{2}{l|}{\textbf{}}\\
			Long&systemid&&\\
			WUGroup&wugroup&&\\
			String&workunit\_name&&\\
			String&memo&&\\
			Calendar&createdate&&\\
			Calendar&lastupdate&&\\
			\hline
		\end{tabular}
	\end{adjustbox}
\end{table}
\begin{table}[H]
	\caption{Daftar attribut pada \textit{Class} MedicalUnit}
	\centering
	\small
	\begin{adjustbox}{width=1\textwidth}	
		\begin{tabular}{|p{4cm} p{2.1cm} p{3cm} p{3.1cm}|}
			\hline
			\multicolumn{2}{|l}{\textbf{Variabel:}}&\multicolumn{2}{l|}{}\\
			Long&systemid&&\\
			\hline
		\end{tabular}
	\end{adjustbox}
\end{table}
\begin{table}[H]
	\caption{Daftar attribut pada \textit{Class} WorkingUnitMembership}
	\centering
	\small
	\begin{adjustbox}{width=1\textwidth}	
		\begin{tabular}{|p{4cm} p{2.1cm} p{3cm} p{3.1cm}|}
			\hline
			\multicolumn{2}{|l}{\textbf{Variabel:}}&\multicolumn{2}{l|}{}\\
			WorkingUnit&workingunit&&\\
			Employee&employee&&\\
			boolean&workingunitpersonincharge&&\\
			String&description&&\\
			Calendar&sys\_createdate&&\\
			Calendar&syssys\_lastupdate&&\\
			\hline
		\end{tabular}
	\end{adjustbox}
\end{table}
\begin{table}[H]
	\caption{Daftar attribut pada \textit{Class} WUGroup}
	\centering
	\small
	\begin{adjustbox}{width=1\textwidth}	
		\begin{tabular}{|p{4cm} p{2.1cm} p{3cm} p{3.1cm}|}
			\hline
			\multicolumn{2}{|l}{\textbf{Variabel:}}&\multicolumn{2}{l|}{}\\
			Long&systemid&&\\
			String&groupname&&\\
			String&memo&&\\
			\hline
		\end{tabular}
	\end{adjustbox}
\end{table}
\begin{table}[H]
	\caption{Daftar attribut pada \textit{Class} WorkingUnitMembershipPK}
	\centering
	\small
	\begin{adjustbox}{width=1\textwidth}	
		\begin{tabular}{|p{4cm} p{2.1cm} p{3cm} p{3.1cm}|}
			\hline
			\multicolumn{2}{|l}{\textbf{Variabel:}}&\multicolumn{2}{l|}{}\\
			Long&workingunit&&\\
			Long&employee&&\\
			\hline
		\end{tabular}
	\end{adjustbox}
\end{table}
\begin{table}[H]
	\caption{Daftar attribut pada \textit{Class} Employment}
	\centering
	\small
	\begin{adjustbox}{width=1\textwidth}	
		\begin{tabular}{|p{4cm} p{2.1cm} p{3cm} p{3.1cm}|}
			\hline
			\multicolumn{2}{|l}{\textbf{Variabel:}}&\multicolumn{2}{l|}{\textbf{Variabel:}}\\
			Long&systemid&Employee&employee\\
			String&ein&Date&hiredate\\
			BigInteger&basesalary&RoleInDept&roleIndept\\
			Integer&contracttype&String& sourceinstitution\\
			int&sourcetype&Date& createdate\\
			Date&lastupdate&Date& terminatedate\\
			\hline
		\end{tabular}
	\end{adjustbox}
\end{table}
\begin{table}[H]
	\caption{Daftar attribut pada \textit{Class} Employee}
	\centering
	\small
	\begin{adjustbox}{width=1\textwidth}	
		\begin{tabular}{|p{4cm} p{2.1cm} p{3cm} p{3.1cm}|}
			\hline
			\multicolumn{2}{|l}{\textbf{Variabel:}}&\multicolumn{2}{l|}{}\\
			String&m\_ein&&\\
			Jobspeciality&jobspeciality&&\\
			Collection$<$Employment$>$&employmentCollection&&\\
			\hline
		\end{tabular}
	\end{adjustbox}
\end{table}
\begin{table}[H]
	\caption{Daftar attribut pada \textit{Class} Departement}
	\centering
	\small
	\begin{adjustbox}{width=1\textwidth}	
		\begin{tabular}{|p{4cm} p{2.1cm} p{3cm} p{3.1cm}|}
			\hline
			\multicolumn{2}{|l}{\textbf{Variabel:}}&\multicolumn{2}{l|}{\textbf{}}\\
			int&systemid&&\\
			String&deptname&&\\
			String&memo&&\\
			Collection$<$RoleInDept$>$&roles&&\\
			Date&createdate&&\\
			Date&lastupdate&&\\
			\hline
		\end{tabular}
	\end{adjustbox}
\end{table}
\begin{table}[H]
\caption{Daftar attribut pada \textit{Class} ContactAddress}
\centering
\small
\begin{adjustbox}{width=1\textwidth}	
	\begin{tabular}{|p{4cm} p{2.1cm} p{3cm} p{3.1cm}|}
		\hline
		\multicolumn{2}{|l}{\textbf{Variabel:}}&\multicolumn{2}{l|}{\textbf{Variabel:}}\\
		Long&systemid&String&postcode\\
		AddressType&addresstype&Regency&regency\\
		String&street&boolean&asbillingaddress\\
		String&fax&Date&sys\_createdate\\
		Contact&owner&String&sys\_creator\\
		String&phone&Date&sys\_lastupdate\\
		String&sys\_lastupdater&&\\
		\hline
	\end{tabular}
\end{adjustbox}
\end{table}
\begin{table}[H]
	\caption{Daftar attribut pada \textit{Class} Contact}
	\centering
	\small
	\begin{adjustbox}{width=1\textwidth}	
		\begin{tabular}{|p{2cm} p{2.1cm} p{5cm} p{3.1cm}|}
			\hline
			\multicolumn{2}{|l}{\textbf{Variabel:}}&\multicolumn{2}{l|}{\textbf{Variabel:}}\\
			long&systemid&String&initial\\
			String&lastname&int&amountofchildren\\
			int&maritalstatus&List$<$ContactAddress$>$&arrAddress\\
			String&middlename&Collection$<$ContactEducation$>$&arrEdu\\
			Date&birthday&String&notes\\
			String&birthplace&String&officeext\\
			String&bloodtype&String&passportid\\
			double&bodyheight&byte[]&photo\\
			double&bodyweight&String&prefixtitle\\
			String&citizenid&String&sms\\
			String&citizentype&String&suffixtitle\\
			Country&citizenship&Date&sys\_createdate\\
			String&email&String&sys\_creator\\
			String&firstname&Date&sys\_lastupdate\\
			int&gender&String&sys\_lastupdater\\
			String&homephone&String&taxid\\
			String&messaging&String&web\\
			\hline
		\end{tabular}
	\end{adjustbox}
\end{table}
\begin{table}[H]
	\caption{Daftar attribut pada \textit{Class} AddressType}
	\centering
	\small
	\begin{adjustbox}{width=1\textwidth}	
		\begin{tabular}{|p{4cm} p{2.1cm} p{3cm} p{3.1cm}|}
			\hline
			\multicolumn{2}{|l}{\textbf{Variabel:}}&\multicolumn{2}{l|}{\textbf{}}\\
			Integer&systemid&&\\
			String&memo&&\\
			String&typename&&\\
			\hline
		\end{tabular}
	\end{adjustbox}
\end{table}
\begin{table}[H]
	\caption{Daftar attribut pada \textit{Class} ContactEducation}
	\centering
	\small
	\begin{adjustbox}{width=1\textwidth}	
		\begin{tabular}{|p{3cm} p{3.1cm} p{3cm} p{3.1cm}|}
			\hline
			\multicolumn{2}{|l}{\textbf{Variabel:}}&\multicolumn{2}{l|}{\textbf{\textbf{Variabel:}}}\\
			Date&finishrolling&Contact&owner\\
			String&gpa&String&sponsors\\
			int&grade&Date&startrolling\\
			String&institutionaladdress&Date&sys\_createdate\\
			String&institutionname&int&sys\_creator\\
			String&major&Date&sys\_lastupdate\\
			String&notes&int&sys\_lastupdater\\
			\hline
		\end{tabular}
	\end{adjustbox}
\end{table}
\begin{table}[H]
	\caption{Daftar attribut pada \textit{Class} Country}
	\centering
	\small
	\begin{adjustbox}{width=1\textwidth}	
		\begin{tabular}{|p{4cm} p{2.1cm} p{3cm} p{3.1cm}|}
			\hline
			\multicolumn{2}{|l}{\textbf{Variabel:}}&\multicolumn{2}{l|}{\textbf{}}\\
			Long&systemid&&\\
			String&name&&\\
			\hline
		\end{tabular}
	\end{adjustbox}
\end{table}
\begin{table}[H]
	\caption{Daftar attribut pada \textit{Class} Jobspeciality}
	\centering
	\small
	\begin{adjustbox}{width=1\textwidth}	
		\begin{tabular}{|p{5cm} p{3.1cm} p{2cm} p{2.1cm}|}
			\hline
			\multicolumn{2}{|l}{\textbf{Variabel:}}&\multicolumn{2}{l|}{\textbf{}}\\
			Long&systemid&&\\
			String&specialityName&&\\
			String&memo&&\\
			\hline
		\end{tabular}
	\end{adjustbox}
\end{table}
\begin{table}[H]
	\caption{Daftar attribut pada \textit{Class} RoleInDept}
	\centering
	\small
	\begin{adjustbox}{width=1\textwidth}	
		\begin{tabular}{|p{5cm} p{3.1cm} p{2cm} p{2.1cm}|}
			\hline
			\multicolumn{2}{|l}{\textbf{Variabel:}}&\multicolumn{2}{l|}{\textbf{}}\\
			int&systemid&&\\
			Department&department&&\\
			String&rolename&&\\
			String&abbreviation&&\\
			RoleInDept&parentRole&&\\
			String&memo&&\\
			\hline
		\end{tabular}
	\end{adjustbox}
\end{table}
\begin{table}[H]
	\caption{Daftar attribut pada \textit{Class} Province}
	\centering
	\small
	\begin{adjustbox}{width=1\textwidth}	
		\begin{tabular}{|p{5cm} p{3.1cm} p{2cm} p{2.1cm}|}
			\hline
			\multicolumn{2}{|l}{\textbf{Variabel:}}&\multicolumn{2}{l|}{\textbf{}}\\
			Long&systemid&&\\
			Country&countryCode&&\\
			String&name&&\\
			Set$<$Regency$>$&regencySet&&\\
			\hline
		\end{tabular}
	\end{adjustbox}
\end{table}
\begin{table}[H]
	\caption{Daftar attribut pada \textit{Class} Regency}
	\centering
	\small
	\begin{adjustbox}{width=1\textwidth}	
		\begin{tabular}{|p{5cm} p{3.1cm} p{2cm} p{2.1cm}|}
			\hline
			\multicolumn{2}{|l}{\textbf{Variabel:}}&\multicolumn{2}{l|}{\textbf{}}\\
			Long&systemid&&\\
			Province&provId&&\\
			String&name&&\\
			\hline
		\end{tabular}
	\end{adjustbox}
\end{table}
\subsubsection{\textit{Package} Repository}
\textit{Package} repository berisi \textit{class-class} yang menjadi representasi model dari tabel-tabel yang berada pada basis data. Berikut ini merupakan \textit{class-class} yang terdapat pada \textit{package} repository.
\begin{table}[H]
	\small
	\centering
	\caption{Daftar {\itshape Class} pada {\itshape Package} repository}
	\begin{adjustbox}{width=1\textwidth}
		\begin{tabular}{| p {3 cm} | p {8 cm} | p {3 cm} |}
			\hline
			{\bfseries Package} & {\bfseries Class} & {\bfseries Jenis Class} \\
			\hline
			\multirow{1}{*}{repository} & OutpatientWUQueueRepository & {\itshape Interface} \\
			\hline
		\end{tabular}
	\end{adjustbox}
\end{table}
\subsubsection{\textit{Package} Service}
\textit{Package} service berisi \textit{class-class} yang menginisialisasi \textit{web service} human resource yang akan terbentuk dan digunakan pada \textit{package} controller. Berikut ini merupakan \textit{class-class} yang terdapat pada \textit{package} service.
\begin{table}[H]
	\small
	\centering
	\caption{Daftar {\itshape Class} pada {\itshape Package} service}
	\begin{adjustbox}{width=1\textwidth}
		\begin{tabular}{| p {3 cm} | p {8 cm} | p {3 cm} |}
			\hline
			{\bfseries Package} & {\bfseries Class} & {\bfseries Jenis Class} \\
			\hline
			\multirow{3}{*}{service} & CRUDService & {\itshape Interface} \\
			& OutpatientWUQueueService & {\itshape Interface} \\
			& DefaultOutpatientWUQueueService & {\itshape Class} \\
			\hline
		\end{tabular}
	\end{adjustbox}
\end{table}
\subsection{\textit{Project} \textit{Medical Records}}
\textit{Project Medical Records} berisi susunan \textit{class} pembentuk \textit{webservice Medical Records}. Dalam project ini terdapat 4 buah \textit{package} yang berfungsi untuk memisahkan \textit{class} sesuai dengan fungsinya masing-masing. Ke empat \textit{package} ini yaitu \textit{package} controller yang berisi method-method yang dijalankan pada \textit{web service}, \textit{package} model yang berisi entitas dan tabel, \textit{package} repository yang menghubungkan entitas di \textit{package} model dengan database, dan \textit{package} service yang menginisialisasi \textit{webserviec} yang akan terbentuk. Pada bagian ini akan dijelaskan mengenai \textit{class} dan \textit{method} yang digunakan pada pengembangan \textit{ webservice} Medical Records:
\subsubsection{\textit{Package} Controller}
\textit{Package} controller berisi method API yang dapat digunakan untuk berkomunikasi dengan \textit{service} Rawat Jalan. \textit{Class} yang terdapat dalam \textit{package} ini adalah \textit{class} OutpatientWUQueueController yang berisi 5 buah fungsi API. Di bawah ini merupakan daftar \textit{class} untuk \textit{package} controller pada \textit{webservice} Rawat Jalan.
\begin{table}[H]
	\small
	\centering
	\caption{Daftar {\itshape Class} pada {\itshape Package} Controller}
	\begin{adjustbox}{width=1\textwidth}
		\begin{tabular}{| p {3 cm} | p {8 cm} | p {3 cm} |}
			\hline
			{\bfseries Package} & {\bfseries Class} & {\bfseries Jenis Class} \\
			\hline
			\multirow{1}{*}{Controller} & MedicalRecordController & {\itshape Class} \\
			\hline
		\end{tabular}
	\end{adjustbox}
\end{table}
\begin{table}[H]
	\caption{Daftar \textit{Method} pada \textit{Class} OutpatientWUQueueController}
	\centering
	\small
	\begin{adjustbox}{width=1\textwidth}	
		\begin{tabular}{|p{0.4cm}|p{2.8cm}|p{0.9cm}|p{1.8cm}|p{2.8cm}|p{2.5cm}|}
			\hline
			\multirow{2}{*}{\textbf{No}} & \multirow{2}{*}{\textit{\textbf{Method}}} & \multicolumn{2}{c|}{\textit{\textbf{Input}}} & \multirow{2}{*}{\textit{\textbf{Output}}} & 
			\multirow{2}{*}{\textbf{Keterangan}}\\
			\cline{3-4}
			& & \textbf{Tipe} & \textbf{Variabel} & & \\
			\hline
			1 & addMedicalRecord & void & MR
			Examination & MRExamination, HttpStatus & \textit{Method} ini digunakan untuk membuat \textit{object} rekam medis yang baru melalui \textit{webservice}\\
			\hline
			2 & update
			MedicalRecord & void & MR
			Examination & HttpStatus & \textit{Method} ini digunakan untuk mengubah attribut dari \textit{object} rekam medis yang baru melalui \textit{webservice}\\
			\hline
			3 & getMedicalRecord & void & id & MRExamination, HttpStatus & \textit{Method} ini digunakan untuk mengambil satu \textit{object} rekam medis yang baru melalui \textit{webservice}\\
			\hline
			4 & get
			AllMedicalRecord & void & - & $<$List
			$<$MRExamination
			$>$$>$, HttpStatus & \textit{Method} ini digunakan untuk mengambil semua \textit{list} \textit{object} rekam medis yang baru melalui \textit{webservice}\\
			\hline
			5 & delete
			MedicalRecord & void & id & HttpStatus & \textit{Method} ini digunakan untuk menghapus satu \textit{object} rekam medis yang baru melalui \textit{webservice}\\
			\hline
		\end{tabular}
	\end{adjustbox}
\end{table}
\subsubsection{\textit{Package} Model \textit{Medical Record}}
\textit{Package} Rawat Jalan model berisi entitas-entitas penyusun dari \textit{service} Rawat Jalan. Berikut ini merupakan daftar \textit{class} untuk \textit{package} model.
\begin{table}[H]
	\small
	\centering
	\caption{Daftar {\itshape Class} pada {\itshape Package} model}
	\begin{adjustbox}{width=1\textwidth}
		\begin{tabular}{| p {3 cm} | p {8 cm} | p {3 cm} |}
			\hline
			{\bfseries Package} & {\bfseries Class} & {\bfseries Jenis Class} \\
			\hline
			\multirow{23}{*}{Model} & MRMasterRecord & {\itshape Class} \\
			& MRExamination & {\itshape Class} \\
			& MRExamResult & {\itshape Class} \\
			& MRSupportDoc & {\itshape Class} \\
			& AddressType & {\itshape Class} \\
			& Contact & {\itshape Class} \\
			& ContactAddress & {\itshape Class} \\
			& ContactEducation & {\itshape Class} \\
			& Country & {\itshape Class} \\
			& Customer & {\itshape Class} \\
			& Customergroup & {\itshape Class} \\
			& Department & {\itshape Class} \\
			& Employee & {\itshape Class} \\
			& Employment & {\itshape Class} \\
			& HealthConsumer & {\itshape Class} \\
			& Insurance & {\itshape Class} \\
			& InsuranceBridgeConf & {\itshape Class} \\
			& Jobspeciality & {\itshape Class} \\
			& Profession & {\itshape Class} \\
			& Province & {\itshape Class} \\
			& RoleInDept & {\itshape Class} \\
			& Regency & {\itshape Class} \\
			\hline
		\end{tabular}
	\end{adjustbox}
\end{table}
\begin{table}[H]
	\caption{Daftar attribut pada \textit{Class} MRMasterRecord}
	\centering
	\small
	\begin{adjustbox}{width=1\textwidth}	
		\begin{tabular}{|p{2cm} p{2.1cm} p{5cm} p{3.1cm}|}
			\hline
			\multicolumn{2}{|l}{\textbf{Variabel:}}&\multicolumn{2}{l|}{\textbf{\textbf{Variabel:}}}\\
			Long&systemid&int&ref\_type\\
			String&ref\_id&HealthConsumer&owner\\
			Employee&conductedby&Date&transactiondate\\
			String&memo&Collection$<$MRSupportDoc$>$&supportDocs\\
			Date&sys\_createdate&Date&sys\_lastupdate\\
			\hline
		\end{tabular}
	\end{adjustbox}
\end{table}
\begin{table}[H]
	\caption{Daftar attribut pada \textit{Class} MRExamination}
	\centering
	\small
	\begin{adjustbox}{width=1\textwidth}	
		\begin{tabular}{|p{4cm} p{2.1cm} p{3cm} p{3.1cm}|}
			\hline
			\multicolumn{2}{|l}{\textbf{Variabel:}}&\multicolumn{2}{l|}{\textbf{}}\\
			String&examname&&\\
			List$<$MRExamResult$>$&mRExamResultCollection&&\\
			\hline
		\end{tabular}
	\end{adjustbox}
\end{table}
\begin{table}[H]
	\caption{Daftar attribut pada \textit{Class} MRExamResult}
	\centering
	\small
	\begin{adjustbox}{width=1\textwidth}	
		\begin{tabular}{|p{4cm} p{2.1cm} p{3cm} p{3.1cm}|}
			\hline
			\multicolumn{2}{|l}{\textbf{Variabel:}}&\multicolumn{2}{l|}{\textbf{}}\\
			String&param\_name&&\\
			MRExamination&parent&&\\
			String&param\_value&&\\
			String&param\_metric&&\\
			\hline
		\end{tabular}
	\end{adjustbox}
\end{table}
\begin{table}[H]
	\caption{Daftar attribut pada \textit{Class} MRSupportDoc}
	\centering
	\small
	\begin{adjustbox}{width=1\textwidth}	
		\begin{tabular}{|p{2.2cm} p{3.1cm} p{3cm} p{3.9cm}|}
			\hline
			\multicolumn{2}{|l}{\textbf{Variabel:}}&\multicolumn{2}{l|}{\textbf{\textbf{Variabel:}}}\\
			Long&systemid&MRMasterRecord&parent\\
			String&filename&byte[]&filebytes\\
			String&memo&Date&sysLastupdate\\
			String&param\_metric&String&relativepathonfilestorage\\
			BigIntegertring&approxfilesizeinbytes&Date&sysCreatedate\\
			\hline
		\end{tabular}
	\end{adjustbox}
\end{table}
\begin{table}[H]
	\caption{Daftar attribut pada \textit{Class} HealthConsumer}
	\centering
	\small
	\begin{adjustbox}{width=1\textwidth}	
		\begin{tabular}{|p{4cm} p{2.1cm} p{3cm} p{3.1cm}|}
			\hline
			\multicolumn{2}{|l}{\textbf{Variabel:}}&\multicolumn{2}{l|}{\textbf{}}\\
			String&regis\_no&&\\
			Profession&id\_prof&&\\
			Insurance&insurance&&\\
			String&insurance\_type&&\\
			String&insurance\_no&&\\
			String&insurance\_prg&&\\
			\hline
		\end{tabular}
	\end{adjustbox}
\end{table}
\begin{table}[H]
	\caption{Daftar attribut pada \textit{Class} Insurance}
	\centering
	\small
	\begin{adjustbox}{width=1\textwidth}	
		\begin{tabular}{|p{5cm} p{3.1cm} p{2cm} p{2.1cm}|}
			\hline
			\multicolumn{2}{|l}{\textbf{Variabel:}}&\multicolumn{2}{l|}{\textbf{}}\\
			List$<$InsuranceBridgeConf$>$&insuranceBrigdeConfList&&\\
			int&systemid&&\\
			String&insurance&&\\
			String&memo&&\\
			boolean&active&&\\
			boolean&sysbuiltin&&\\
			\hline
		\end{tabular}
	\end{adjustbox}
\end{table}
\begin{table}[H]
	\caption{Daftar attribut pada \textit{Class} InsuranceBridgeConf}
	\centering
	\small
	\begin{adjustbox}{width=1\textwidth}	
		\begin{tabular}{|p{5cm} p{3.1cm} p{2cm} p{2.1cm}|}
			\hline
			\multicolumn{2}{|l}{\textbf{Variabel:}}&\multicolumn{2}{l|}{\textbf{}}\\
			Long&systemid&&\\
			String&field&&\\
			String&def\_val&&\\
			String&memo&&\\
			\hline
		\end{tabular}
	\end{adjustbox}
\end{table}
\begin{table}[H]
	\caption{Daftar attribut pada \textit{Class} Profession}
	\centering
	\small
	\begin{adjustbox}{width=1\textwidth}	
		\begin{tabular}{|p{5cm} p{3.1cm} p{2cm} p{2.1cm}|}
			\hline
			\multicolumn{2}{|l}{\textbf{Variabel:}}&\multicolumn{2}{l|}{\textbf{}}\\
			Integer&systemid&&\\
			String&profname&&\\
			String&memo&&\\
			Calendar&sys\_createdate&&\\
			Calendar&last\_createdate&&\\
			\hline
		\end{tabular}
	\end{adjustbox}
\end{table}
\begin{table}[H]
	\caption{Daftar attribut pada \textit{Class} Customer}
	\centering
	\small
	\begin{adjustbox}{width=1\textwidth}	
		\begin{tabular}{|p{4cm} p{2.1cm} p{3cm} p{3.1cm}|}
			\hline
			\multicolumn{2}{|l}{\textbf{Variabel:}}&\multicolumn{2}{l|}{\textbf{Variabel:}}\\
			boolean&cust\_no&Date&registrationdate\\
			boolean&active&int&idPriceLevel\\
			double&creditlimit&double&disc\\
			Customergroup&customergroup&&\\
			\hline
		\end{tabular}
	\end{adjustbox}
\end{table}

\begin{table}[H]
	\caption{Daftar attribut pada \textit{Class} Customergroup}
	\centering
	\small
	\begin{adjustbox}{width=1\textwidth}	
		\begin{tabular}{|p{4cm} p{2.1cm} p{3cm} p{3.1cm}|}
			\hline
			\multicolumn{2}{|l}{\textbf{Variabel:}}&\multicolumn{2}{l|}{}\\
			long&systemid&&\\
			String&groupname&&\\
			String&memo&&\\
			\hline
		\end{tabular}
	\end{adjustbox}
\end{table}
\begin{table}[H]
	\caption{Daftar attribut pada \textit{Class} Employment}
	\centering
	\small
	\begin{adjustbox}{width=1\textwidth}	
		\begin{tabular}{|p{4cm} p{2.1cm} p{3cm} p{3.1cm}|}
			\hline
			\multicolumn{2}{|l}{\textbf{Variabel:}}&\multicolumn{2}{l|}{\textbf{Variabel:}}\\
			Long&systemid&Employee&employee\\
			String&ein&Date&hiredate\\
			BigInteger&basesalary&RoleInDept&roleIndept\\
			Integer&contracttype&String& sourceinstitution\\
			int&sourcetype&Date& createdate\\
			Date&lastupdate&Date& terminatedate\\
			\hline
		\end{tabular}
	\end{adjustbox}
\end{table}
\begin{table}[H]
	\caption{Daftar attribut pada \textit{Class} Employee}
	\centering
	\small
	\begin{adjustbox}{width=1\textwidth}	
		\begin{tabular}{|p{4cm} p{2.1cm} p{3cm} p{3.1cm}|}
			\hline
			\multicolumn{2}{|l}{\textbf{Variabel:}}&\multicolumn{2}{l|}{}\\
			String&m\_ein&&\\
			Jobspeciality&jobspeciality&&\\
			Collection$<$Employment$>$&employmentCollection&&\\
			\hline
		\end{tabular}
	\end{adjustbox}
\end{table}
\begin{table}[H]
	\caption{Daftar attribut pada \textit{Class} Departement}
	\centering
	\small
	\begin{adjustbox}{width=1\textwidth}	
		\begin{tabular}{|p{4cm} p{2.1cm} p{3cm} p{3.1cm}|}
			\hline
			\multicolumn{2}{|l}{\textbf{Variabel:}}&\multicolumn{2}{l|}{\textbf{}}\\
			int&systemid&&\\
			String&deptname&&\\
			String&memo&&\\
			Collection$<$RoleInDept$>$&roles&&\\
			Date&createdate&&\\
			Date&lastupdate&&\\
			\hline
		\end{tabular}
	\end{adjustbox}
\end{table}
\begin{table}[H]
	\caption{Daftar attribut pada \textit{Class} ContactAddress}
	\centering
	\small
	\begin{adjustbox}{width=1\textwidth}	
		\begin{tabular}{|p{4cm} p{2.1cm} p{3cm} p{3.1cm}|}
			\hline
			\multicolumn{2}{|l}{\textbf{Variabel:}}&\multicolumn{2}{l|}{\textbf{Variabel:}}\\
			Long&systemid&String&postcode\\
			AddressType&addresstype&Regency&regency\\
			String&street&boolean&asbillingaddress\\
			String&fax&Date&sys\_createdate\\
			Contact&owner&String&sys\_creator\\
			String&phone&Date&sys\_lastupdate\\
			String&sys\_lastupdater&&\\
			\hline
		\end{tabular}
	\end{adjustbox}
\end{table}
\begin{table}[H]
	\caption{Daftar attribut pada \textit{Class} Contact}
	\centering
	\small
	\begin{adjustbox}{width=1\textwidth}	
		\begin{tabular}{|p{2cm} p{2.1cm} p{5cm} p{3.1cm}|}
			\hline
			\multicolumn{2}{|l}{\textbf{Variabel:}}&\multicolumn{2}{l|}{\textbf{Variabel:}}\\
			long&systemid&String&initial\\
			String&lastname&int&amountofchildren\\
			int&maritalstatus&List$<$ContactAddress$>$&arrAddress\\
			String&middlename&Collection$<$ContactEducation$>$&arrEdu\\
			Date&birthday&String&notes\\
			String&birthplace&String&officeext\\
			String&bloodtype&String&passportid\\
			double&bodyheight&byte[]&photo\\
			double&bodyweight&String&prefixtitle\\
			String&citizenid&String&sms\\
			String&citizentype&String&suffixtitle\\
			Country&citizenship&Date&sys\_createdate\\
			String&email&String&sys\_creator\\
			String&firstname&Date&sys\_lastupdate\\
			int&gender&String&sys\_lastupdater\\
			String&homephone&String&taxid\\
			String&messaging&String&web\\
			\hline
		\end{tabular}
	\end{adjustbox}
\end{table}
\begin{table}[H]
	\caption{Daftar attribut pada \textit{Class} AddressType}
	\centering
	\small
	\begin{adjustbox}{width=1\textwidth}	
		\begin{tabular}{|p{4cm} p{2.1cm} p{3cm} p{3.1cm}|}
			\hline
			\multicolumn{2}{|l}{\textbf{Variabel:}}&\multicolumn{2}{l|}{\textbf{}}\\
			Integer&systemid&&\\
			String&memo&&\\
			String&typename&&\\
			\hline
		\end{tabular}
	\end{adjustbox}
\end{table}
\begin{table}[H]
	\caption{Daftar attribut pada \textit{Class} ContactEducation}
	\centering
	\small
	\begin{adjustbox}{width=1\textwidth}	
		\begin{tabular}{|p{3cm} p{3.1cm} p{3cm} p{3.1cm}|}
			\hline
			\multicolumn{2}{|l}{\textbf{Variabel:}}&\multicolumn{2}{l|}{\textbf{\textbf{Variabel:}}}\\
			Date&finishrolling&Contact&owner\\
			String&gpa&String&sponsors\\
			int&grade&Date&startrolling\\
			String&institutionaladdress&Date&sys\_createdate\\
			String&institutionname&int&sys\_creator\\
			String&major&Date&sys\_lastupdate\\
			String&notes&int&sys\_lastupdater\\
			\hline
		\end{tabular}
	\end{adjustbox}
\end{table}
\begin{table}[H]
	\caption{Daftar attribut pada \textit{Class} Country}
	\centering
	\small
	\begin{adjustbox}{width=1\textwidth}	
		\begin{tabular}{|p{4cm} p{2.1cm} p{3cm} p{3.1cm}|}
			\hline
			\multicolumn{2}{|l}{\textbf{Variabel:}}&\multicolumn{2}{l|}{\textbf{}}\\
			Long&systemid&&\\
			String&name&&\\
			\hline
		\end{tabular}
	\end{adjustbox}
\end{table}
\begin{table}[H]
	\caption{Daftar attribut pada \textit{Class} Jobspeciality}
	\centering
	\small
	\begin{adjustbox}{width=1\textwidth}	
		\begin{tabular}{|p{5cm} p{3.1cm} p{2cm} p{2.1cm}|}
			\hline
			\multicolumn{2}{|l}{\textbf{Variabel:}}&\multicolumn{2}{l|}{\textbf{}}\\
			Long&systemid&&\\
			String&specialityName&&\\
			String&memo&&\\
			\hline
		\end{tabular}
	\end{adjustbox}
\end{table}
\begin{table}[H]
	\caption{Daftar attribut pada \textit{Class} RoleInDept}
	\centering
	\small
	\begin{adjustbox}{width=1\textwidth}	
		\begin{tabular}{|p{5cm} p{3.1cm} p{2cm} p{2.1cm}|}
			\hline
			\multicolumn{2}{|l}{\textbf{Variabel:}}&\multicolumn{2}{l|}{\textbf{}}\\
			int&systemid&&\\
			Department&department&&\\
			String&rolename&&\\
			String&abbreviation&&\\
			RoleInDept&parentRole&&\\
			String&memo&&\\
			\hline
		\end{tabular}
	\end{adjustbox}
\end{table}
\begin{table}[H]
	\caption{Daftar attribut pada \textit{Class} Province}
	\centering
	\small
	\begin{adjustbox}{width=1\textwidth}	
		\begin{tabular}{|p{5cm} p{3.1cm} p{2cm} p{2.1cm}|}
			\hline
			\multicolumn{2}{|l}{\textbf{Variabel:}}&\multicolumn{2}{l|}{\textbf{}}\\
			Long&systemid&&\\
			Country&countryCode&&\\
			String&name&&\\
			Set$<$Regency$>$&regencySet&&\\
			\hline
		\end{tabular}
	\end{adjustbox}
\end{table}
\begin{table}[H]
	\caption{Daftar attribut pada \textit{Class} Regency}
	\centering
	\small
	\begin{adjustbox}{width=1\textwidth}	
		\begin{tabular}{|p{5cm} p{3.1cm} p{2cm} p{2.1cm}|}
			\hline
			\multicolumn{2}{|l}{\textbf{Variabel:}}&\multicolumn{2}{l|}{\textbf{}}\\
			Long&systemid&&\\
			Province&provId&&\\
			String&name&&\\
			\hline
		\end{tabular}
	\end{adjustbox}
\end{table}
\subsubsection{\textit{Package} Repository}
\textit{Package} repository berisi \textit{class-class} yang menjadi representasi model dari tabel-tabel yang berada pada basis data. Berikut ini merupakan \textit{class-class} yang terdapat pada \textit{package} repository.
\begin{table}[H]
	\small
	\centering
	\caption{Daftar {\itshape Class} pada {\itshape Package} repository}
	\begin{adjustbox}{width=1\textwidth}
		\begin{tabular}{| p {3 cm} | p {8 cm} | p {3 cm} |}
			\hline
			{\bfseries Package} & {\bfseries Class} & {\bfseries Jenis Class} \\
			\hline
			\multirow{1}{*}{repository} & MRMasterRecordRepository & {\itshape Interface} \\
			\hline
		\end{tabular}
	\end{adjustbox}
\end{table}
\subsubsection{\textit{Package} Service}
\textit{Package} service berisi \textit{class-class} yang menginisialisasi \textit{web service} human resource yang akan terbentuk dan digunakan pada \textit{package} controller. Berikut ini merupakan \textit{class-class} yang terdapat pada \textit{package} service.
\begin{table}[H]
	\small
	\centering
	\caption{Daftar {\itshape Class} pada {\itshape Package} service}
	\begin{adjustbox}{width=1\textwidth}
		\begin{tabular}{| p {3 cm} | p {8 cm} | p {3 cm} |}
			\hline
			{\bfseries Package} & {\bfseries Class} & {\bfseries Jenis Class} \\
			\hline
			\multirow{3}{*}{service} & CRUDService & {\itshape Interface} \\
			& MRMasterRecordService & {\itshape Interface} \\
			& DefaultMRMasterRecordService & {\itshape Class} \\
			\hline
		\end{tabular}
	\end{adjustbox}
\end{table}
\section{Pengujian}
Pada bab ini, akan dilakukan berbagai pengujian untuk menentukan performa dari arsitektur microservice dibandingkan dengan monolitik. Pengujian akan dilakukan dengan membuat \textit{test plan} yang kemudian akan didokumentasikan ke dalam\textit{matrix traceability}. Dari hasil \textit{matrix traceability} tersebut dapat ditentukan apakah desain microservice yang baru memberikan performa yang lebih baik dibandingkan dengan model monolitik yang lama.
\subsection{Pengujian dan Perbandingan Model Arsitektur Microservice dan Monolitik}
Pada bagian ini akan dilakukan pengujian terhadap kedua jenis model arsitektur yang didokumentasikan ke dalam \textit{matrix of traceability}. Pengujian akan dilakukan dengan membandingkan hasil \textit{test plan} yang telah dibuat. \textit{Test plan} yang dibuat akan mengacu pada masalah \textit{deployability, reliability, availability, scalability,} dan \textit{modifiability.}
\begin{table}[H]
	\centering
	\caption{Tabel \textit{matrix of traceability} pengujian model arsitektur}
	\label{my-label}
	\begin{adjustbox}{width=1\textwidth}
		\begin{tabular}{|p{1cm}|p{8.5cm}|p{1cm}|p{1cm}|p{1cm}|p{1cm}|p{1cm}|}
			\hline
			\multicolumn{1}{|c|}{\multirow{2}{*}{\textbf{No}}} & \multicolumn{1}{c|}{\multirow{2}{*}{\textbf{Test Plan}}} & \multicolumn{2}{c|}{\textbf{Microservice}}& \multicolumn{2}{c|}{\textbf{Monolitik}} & \multicolumn{1}{c|}{\multirow{2}{*}{\textbf{Harapan}}}\\ \cline{3-6} 
			\multicolumn{1}{|c|}{}                  & \multicolumn{1}{c|}{}                  & \multicolumn{1}{c|}{\textbf{Ya}} & \multicolumn{1}{c|}{\textbf{Tidak}} & \multicolumn{1}{c|}{\textbf{Ya}} & \multicolumn{1}{c|}{\textbf{Tidak}} & \multicolumn{1}{c|}{}\\ \hline
			\multicolumn{1}{|c|}{1}&Perubahan yang dilakukan pada class dalam sebuah modul harus diketahui oleh modul lain yang saling melakukan pertukaran data&&\checkmark&\checkmark&&\multicolumn{1}{c|}{\textbf{Tidak}}\\ \hline
			\multicolumn{1}{|c|}{2}&Apabila terjadi masalah terhadap database, maka masalah tersebut akan turut dirasakan oleh keseluruhan modul&&\checkmark&\checkmark&&\multicolumn{1}{c|}{\textbf{Tidak}}\\ \hline
			\multicolumn{1}{|c|}{3}&Tingkat sekuritas yang lebih baik&\checkmark&&&\checkmark&\multicolumn{1}{c|}{\textbf{Ya}}\\ \hline
			\multicolumn{1}{|c|}{4}&Tingkat ketergantungan yang besar apabila aplikasi dikerjakan oleh beberapa tim berbeda&&\checkmark&\checkmark&&\multicolumn{1}{c|}{\textbf{Tidak}}\\ \hline
			\multicolumn{1}{|c|}{5}&Cara komunikasi antar modul lebih mudah untuk diimplementasikan&&\checkmark&\checkmark&&\multicolumn{1}{c|}{\textbf{Ya}}\\ \hline
			\multicolumn{1}{|c|}{6}&Konsumen yang mengkakses \textit{method} yang diubah harus mengetahui apa perubahan yang terjadi terhadap \textit{method} tersebut&&\checkmark&\checkmark&&\multicolumn{1}{c|}{\textbf{Tidak}}\\ \hline
			\multicolumn{1}{|c|}{7}&Waktu siklus \textit{build-test-build} yang relatif lebih cepat&\checkmark&&&\checkmark&\multicolumn{1}{c|}{\textbf{Ya}}\\ \hline
			\multicolumn{1}{|c|}{8}&Tidak saling berebut \textit{resources} untuk setiap modulnya&\checkmark&&&\checkmark&\multicolumn{1}{c|}{\textbf{Ya}}\\ \hline
			\multicolumn{1}{|c|}{9}&Mudah untuk melakukan \textit{querry join} dari setiap modul yang ada&&\checkmark&\checkmark&&\multicolumn{1}{c|}{\textbf{Ya}}\\ \hline
		\end{tabular}
	\end{adjustbox}
\end{table}
\begin{adjustbox}{width=1\textwidth}
	\begin{tabular}{|p{1cm}|p{8.5cm}|p{1cm}|p{1cm}|p{1cm}|p{1cm}|p{1cm}|}
		\hline
		\multicolumn{1}{|c|}{10}&Relatif dapat menangani permasalahan \textit{big data} dan \textit{multiple user}&\checkmark&&&\checkmark&\multicolumn{1}{c|}{\textbf{Ya}}\\ \hline
		\multicolumn{1}{|c|}{11}&Aplikasi menjadi lebih \textit{long lasting} dan relatif tidak memiliki resiko perombakan major apabila terdapat perubahan pada aplikasi&\checkmark&&&\checkmark&\multicolumn{1}{c|}{\textbf{Ya}}\\ \hline
		\multicolumn{1}{|c|}{12}&Mudah untuk menambahkan modul baru&\checkmark&&\checkmark&&\multicolumn{1}{c|}{\textbf{Ya}}\\ \hline
		\multicolumn{1}{|c|}{13}&Lebih mudah dan cepat beradaptasi apabila aplikasi menjadi lebih besar&\checkmark&&&\checkmark&\multicolumn{1}{c|}{\textbf{Ya}}\\ \hline
		\multicolumn{1}{|c|}{14}&Aplikasi lebih mudah bila ingin dikembangkan menjadi lintas \textit{platform}&\checkmark&&&\checkmark&\multicolumn{1}{c|}{\textbf{Ya}}\\ \hline
		\multicolumn{1}{|c|}{15}&Aplikasi lebih mudah untuk melakukan \textit{deploy} ke \textit{server}&\checkmark&&&\checkmark&\multicolumn{1}{c|}{\textbf{Ya}}\\ \hline
		\multicolumn{1}{|c|}{16}&Pengujian alpikasi mudah dan cepat&\checkmark&&\checkmark&&\multicolumn{1}{c|}{\textbf{Ya}}\\ \hline
		\multicolumn{1}{|c|}{17}&Lebih mudah untuk menerapkan teknologi baru&\checkmark&&&\checkmark&\multicolumn{1}{c|}{\textbf{Ya}}\\ \hline
		\multicolumn{1}{|c|}{18}&Konfigurasi aplikasi mudah&&\checkmark&\checkmark&&\multicolumn{1}{c|}{\textbf{Ya}}\\ \hline
		\multicolumn{1}{|c|}{19}&Perlu dilakukan banyak automasi agar aplikasi dapat memberikan performa terbaiknya&&\checkmark&\checkmark&&\multicolumn{1}{c|}{\textbf{Ya}}\\ \hline
		\multicolumn{1}{|c|}{20}&Cocok untuk sistem yang dinamis dan konstan berkembang&\checkmark&&&\checkmark&\multicolumn{1}{c|}{\textbf{Ya}}\\ \hline
		\multicolumn{1}{|c|}{21}&Jumlah waktu \textit{downtime} yang dibutuhkan aplikasi apabila terjadi \textit{maintenance} lebih sedikit&\checkmark&&&\checkmark&\multicolumn{1}{c|}{\textbf{Ya}}\\ \hline
	\end{tabular}
\end{adjustbox}

Berdasarkan hasil pengujian di atas, desain arsitektur microservice memenuhi 17 dari 21 kasus \textit{test plan} yang dibuat, sedangkan desain arsitektur monolitik memenuhi 6 dari 21 kasus \textit{test plan}.
\newpage
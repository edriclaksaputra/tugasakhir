%-----------------------------------------------------------------------------%
\chapter{IMPLEMENTASI DAN PENGUJIAN}
%-----------------------------------------------------------------------------%

%

\vspace{4.5pt}
Pada bab ini akan menjelaskan tentang pengimplementasian dan pengujian terhadap analisis sentimen yang telah dibangun berdasarkan bab-bab sebelumnya.
\section{Lingkungan Aplikasi}
Dalam aplikasi terbagi menjadi dua bagian, yaitu lingkungan implementasi perangkat keras dan perangkat lunak. Di dalam penelitian ini, perangkat keras yang digunakan adalah:
\begin{enumerate}[leftmargin=*]
	\item Asus A45A
	\item Processor Intel i3-2370M CPU 2.40GHz
	\item RAM 6 GB.
\end{enumerate}

Spesifikasi perangkat lunak yang digunakan untuk pengembangan sistem adalah:
\begin{enumerate}[leftmargin=*]
	\item Sistem Operasi\quad\quad\quad\,: Windows 10 Enterprise 1709 64-bit.
	\item Tool Pengembangan\quad: Eclipse Java EE IDE for Web Developers.
	\item Versi\quad\quad\quad\quad\quad\quad\,: Neon.3 Release (4.6.3).
\end{enumerate}

\section{Daftar \textit{Project}, \textit{Class} dan \textit{Method}}
Pada bagian ini akan dijelaskan mengenai \textit{Project}, \textit{class} dan \textit{method} yang digunakan dalam pengembangan sistem analisis sentimen
\subsection{\textit{Project} Customer}
\textit{Project Customer} berisi susunan \textit{class} pembentuk \textit{webservice customer}. Pada bagian ini akan dijelaskan mengenai \textit{class} dan \textit{method} yang digunakan pada pengembangan \textit{ webservice customer}:
\subsubsection{\textit{Class} CustomerController}
\begin{table}[H]
	\caption{Daftar \textit{Method} pada \textit{Class} CustomerController}
	\centering
	\small
	\begin{adjustbox}{width=1\textwidth}	
	\begin{tabular}{|p{0.4cm}|p{3.2cm}|p{1.4cm}|p{1.7cm}|p{1.55cm}|p{3cm}|}
		\hline
		\multirow{2}{*}{\textbf{No}} & \multirow{2}{*}{\textit{\textbf{Method}}} & \multicolumn{2}{c|}{\textit{\textbf{Input}}} & \multirow{2}{*}{\textit{\textbf{Output}}} & 
		\multirow{2}{*}{\textbf{Keterangan}}\\
		\cline{3-4}
		& & \textbf{Tipe} & \textbf{Variabel} & & \\
		\hline
		1 & addEmployee & Customer & customer & customer, HttpStatus & \textit{Method} ini digunakan untuk membuat \textit{object} customer yang baru melalui \textit{webservice}\\
		\hline
		2 & updateEmployee & Customer & customer & HttpStatus & \textit{Method} ini digunakan untuk membuat \textit{object} customer yang baru melalui \textit{webservice}\\
		\hline
		3 & getEmployee & Customer & customer & customer, HttpStatus & \textit{Method} ini digunakan untuk membuat \textit{object} customer yang baru melalui \textit{webservice}\\
		\hline
		4 & getAllEmployee & Customer & customer & customer, HttpStatus & \textit{Method} ini digunakan untuk membuat \textit{object} customer yang baru melalui \textit{webservice}\\
		\hline
		5 & deleteEmployee & Customer & customer & customer, HttpStatus & \textit{Method} ini digunakan untuk membuat \textit{object} customer yang baru melalui \textit{webservice}\\
		\hline
	\end{tabular}
	\end{adjustbox}
\end{table}
\subsection{\textit{Class} Preprocessing}
\textit{Class} preprocessing merupakan \textit{class} yang digunakan untuk meminimalisir kata, mengurangi bahasa yang non-formal, dan menghapus huruf ataupun tanda baca yang tidak digunakan. Berikut ini adalah daftar \textit{method }pada \textit{class} preprocessing:

\begin{table}[H]
	\caption{Daftar \textit{Method} pada \textit{Class} Preprocessing}
	\centering
	\small
	\begin{adjustbox}{width=1\textwidth}	
	\begin{tabular}{|p{4cm} p{2.1cm} p{3cm} p{3.1cm}|}
		\hline
		\multicolumn{2}{|l}{\textbf{Variabel:}}&\multicolumn{2}{l|}{\textbf{Variabel:}}\\
		Stemmer (Class Lib)&stemmer&Array&data\\
		Array (String)&question\_dic&Array&label\\
		Array (String)&negasi&Int&size\\
		Array (String)&stopword&Array(String)&abbreviation\_dic\\
		\hline
	\end{tabular}
	\end{adjustbox}
\end{table}

\begin{small}
	\begin{longtable}{@{\extracolsep{\fill}}|p{0.4cm}|p{3.2cm}|p{1.4cm}|p{1.7cm}|p{1cm}|p{3.55cm}|}
		\hline
		\multirow{2}{*}{\textbf{No}} & \multirow{2}{*}{\textit{\textbf{Method}}} & \multicolumn{2}{c|}{\textit{\textbf{Input}}} & \multirow{2}{*}{\textit{\textbf{Output}}} & 
		\multirow{2}{*}{\textbf{Keterangan}}\\
		\cline{3-4}
		& & \textbf{Tipe} & \textbf{Variabel} & & \\
		\hline
		\endhead
		1 & case\_folding & 
		\begin{itemize}[leftmargin=*,label={-}]
			\item String
		\end{itemize}
		& \begin{itemize}[leftmargin=*,label={-}]
			\item tweet
		\end{itemize}
		& String & \textit{Method} ini digunakan untuk mengecilkan setiap huruf pada \textit{tweet}. \\
		\hline
		2 & remove\_hashtag\_ mention\_url & 
		\begin{itemize}[leftmargin=*,label={-}]
			\item String
		\end{itemize}
		& \begin{itemize}[leftmargin=*,label={-}]
			\item tweet
		\end{itemize}
		& String & \textit{Method} ini digunakan untuk menghapus semua textit{hashtag}, \textit{mention}, URL pada \textit{tweet}. \\
		\hline
		3 & remove\_punctuation & 
		\begin{itemize}[leftmargin=*,label={-}]
			\item String
		\end{itemize}
		& \begin{itemize}[leftmargin=*,label={-}]
			\item tweet
		\end{itemize}
		& String & \textit{Method} ini digunakan untuk menghapus semua tanda baca yang ada pada \textit{tweet} kecuali tanda seru (!), tanda tanya (?), tanda petik ("), tanda petik tunggal ('), dan tanda pemisah (-). 
		\\
		\hline
		4 & tokenize & 
		\begin{itemize}[leftmargin=*,label={-}]
			\item String
		\end{itemize}
		& \begin{itemize}[leftmargin=*,label={-}]
			\item tweet
		\end{itemize}
		& Array & \textit{Method} ini digunakan untuk memisahkan teks menjadi token-token yang terdiri dari satu kata. \\
		\hline
		5 & misuse\_of\_word & 
		\begin{itemize}[leftmargin=*,label={-}]
			\item String
		\end{itemize}
		& \begin{itemize}[leftmargin=*,label={-}]
			\item tweet
		\end{itemize}
		& String & \textit{Method} ini digunakan untuk menghilangkan huruf sama yang saling bersebelahan. \\
		\hline
		6 & abbreviation\_word & 
		\begin{itemize}[leftmargin=*,label={-}]
			\item String
		\end{itemize}
		& \begin{itemize}[leftmargin=*,label={-}]
			\item tweet
		\end{itemize}
		& String & \textit{Method} ini berguna untuk mengubah kata-kata singkatan menjadi kata persamaanya. \\
		\hline
		7 & stopword\_removal & 
		\begin{itemize}[leftmargin=*,label={-}]
			\item String
		\end{itemize}
		& \begin{itemize}[leftmargin=*,label={-}]
			\item tweet\end{itemize}
		& String & \textit{Method} ini digunakan untuk menghilangkan kata-kata yang dianggap tidak memiliki makna. \\
		\hline
		8 & stemming & 
		\begin{itemize}[leftmargin=*,label={-}]
			\item String
		\end{itemize}
		& \begin{itemize}[leftmargin=*,label={-}]
			\item tweet
		\end{itemize}
		& String & \textit{Method} ini digunakan untuk mengubah kata menjadi kata dasar. \\
		\hline
		9 & get\_count\_max & 
		\begin{itemize}[leftmargin=*,label={-}]
			\item String
		\end{itemize}
		& \begin{itemize}[leftmargin=*,label={-}]
			\item tweet
		\end{itemize}
		& Int, Int, Int, Int & \textit{Method} ini digunakan untuk mendapatkan kemunculan terbanyak tanda baca tanya (?), tanda seru (!), tanda \textit{quotation} (", '), dan kata kapital. \\
		\hline
		10 & write\_file & 
		\begin{itemize}[leftmargin=*,label={-}]
			\item Array\item Array
		\end{itemize}
		& \begin{itemize}[leftmargin=*,label={-}]
			\item data\item label
		\end{itemize}
		& Array, Array & \textit{Method} ini digunakan untuk melakukan random terhadap data dan label, kemudian menyimpan data dan label kedalam \textit{file} .txt. \\
		\hline
		11 & get\_size & \begin{itemize}[leftmargin=*,label={-}]
			\item Array\end{itemize}
		& \begin{itemize}[leftmargin=*,label={-}]
			\item label\end{itemize}
		& Int & \textit{Method} ini digunakan untuk menghitung jumlah data yang akan digunakan sebagai data \textit{training}, dengan mengalikan jumlah keseluruhan data dengan 75\%. \\
		\hline
		12 & Fit & \begin{itemize}[leftmargin=*,label={-}]
			\item Array\item Array\end{itemize}
		& \begin{itemize}[leftmargin=*,label={-}]
			\item data\item label\end{itemize}
		& Int, Int, Int, Int, Array & Fitur ini digunakan untuk melakukan \textit{preprocessing} secara keseluruhan terhadap data \textit{tweet}, serta mengembalikan hasil kemunculan maksimal tanda baca, dan kata kapital. \\
		\hline
	\end{longtable}
\end{small}
	

\addtocounter{table}{-1}
\subsection{\textit{Class} Features}
\textit{Class }features merupakan \textit{class }yang digunakan untuk menangani hal terkait \textit{feature}, dimulai dari penambahan \textit{feature set}, penyimpanan nilai fitur ke dalam \textit{file}, pengambilan \textit{feature }berdasarkan jenis klasifikasi. Jenis klasifikasi yang ada pada sistem ini adalah klasifikasi 4 kelas 
(positif, negatif, netral, sarkasme), klasifikasi 3 kelas (positif, negatif, netral), 1 kelas (sarkasme/non-sarkasme). Berikut ini adalah \textit{method }pada \textit{class }features:
\begin{table}[H]
	\caption{Daftar \textit{Method} pada \textit{Class} Features}
	\centering
	\small
	\begin{adjustbox}{width=1\textwidth}	
	\begin{tabular}{|p{4.2cm} p{3cm} p{2cm} p{3cm}|}
		\hline
		\multicolumn{2}{|l}{\textbf{Variabel:}}&\multicolumn{2}{l|}{\textbf{Variabel:}}\\
		FeatureExtraction (Class)&feature\_extraction&Array&IDF\\
		Array&feature\_training&Array&label\_training\\
		Topic (Class)&topic\_mod& &\\
		\hline
	\end{tabular}
	\end{adjustbox}
\end{table}
\begin{small}
	\begin{longtable}{@{\extracolsep{\fill}}|p{0.4cm}|p{3.3cm}|p{1.4cm}|p{1.4cm}|p{1.20cm}|p{3.55cm}|}
		\hline
		\multirow{2}{*}{\textbf{No}} & \multirow{2}{*}{\textit{\textbf{Method}}} & \multicolumn{2}{c|}{\textit{\textbf{Input}}} & \multirow{2}{*}{\textit{\textbf{Output}}} & 
		\multirow{2}{*}{\textbf{Keterangan}}\\
		\cline{3-4}
		& & \textbf{Tipe} & \textbf{Variabel} & & \\
		\hline
		\endhead
		1 & get\_sar\_feature & \begin{itemize}[leftmargin=*,label={-}]
			\item Array\item Array\end{itemize}
		& \begin{itemize}[leftmargin=*,label={-}]
			\item data\item label\end{itemize}
		& Array, Array & \textit{Method} ini digunakan untuk menghapus semua data kecuali data positif dan sarkasme. \\
		\hline
		2 & get\_net\_feature & \begin{itemize}[leftmargin=*,label={-}]
			\item Array\item Array\end{itemize}
		& \begin{itemize}[leftmargin=*,label={-}]
			\item data\item label\end{itemize}
		& Array, Array & \textit{Method} ini digunakan untuk menghapus semua data sarkasme. \\
		\hline
		3 & add\_non\_sar\_feature & \begin{itemize}[leftmargin=*,label={-}]
			\item String\end{itemize}
		& \begin{itemize}[leftmargin=*,label={-}]
			\item tweet\end{itemize}
		& Array & \textit{Method} ini digunakan untuk mendapatkan semua nilai fitur untuk kelas positif, negatif dan netral. \\
		\hline
		4 & add\_sar\_feature & \begin{itemize}[leftmargin=*,label={-}]
			\item String\end{itemize}
		& \begin{itemize}[leftmargin=*,label={-}]
			\item tweet\end{itemize}
		& Array & \textit{Method} ini digunakan untuk mendapatkan semua nilai fitur untuk kelas sarkasme. \\
		\hline
		5 & Fit & \begin{itemize}[leftmargin=*,label={-}]
			\item Array\item String\end{itemize}
		& \begin{itemize}[leftmargin=*,label={-}]
			\item data\item type\end{itemize}
		& Array & \textit{Method} ini digunakan untuk memanggil \textit{method} add\_sar\_feature dan add\_non\_sar\_feature untuk menambahkan fitur dari data \textit{training }berdasarkan \textit{type}, jika \textit{type }sama dengan sarkasme, maka akan menambahkan fitur sarkasme, dan sebaliknya. \\
		\hline
		6 & save\_feature & \begin{itemize}[leftmargin=*,label={-}]
			\item Int
			\item Array
			\item Array
		\end{itemize}
		& \begin{itemize}[leftmargin=*,label={-}]
			\item n\_ classify
			\item temp\_ train
			\item temp\_ test
		\end{itemize}
		& - & \textit{Method} ini digunakan untuk menyimpan hasil fitur ekstraksi data \textit{training} dan labelnya ke dalam \textit{file} .txt. \\
		\hline
		7 & get\_train\_feature & 
		\begin{itemize}[leftmargin=*,label={-}]
			\item Int
			\item Array
			\item Array
			\item Array
			\item Array
		\end{itemize}
		& \begin{itemize}[leftmargin=*,label={-}]
			\item n\_ classify
			\item data\_ train
			\item label\_ train
			\item data\_ test
			\item label\_ test
		\end{itemize}
		& Array, Array, Array, Array & \textit{Method} ini digunakan untuk memisahkan dan mengekstraksi fitur dari data yang akan digunakan menjadi dua, yaitu data untuk fitur ekstraksi non-sarkasme, dan data untuk fitur ekstraksi sarkasme. \\
		\hline
		8 & get\_test\_feature & 
		\begin{itemize}[leftmargin=*,label={-}]
			\item Int
			\item Array
			\item Array
		\end{itemize}
		& \begin{itemize}[leftmargin=*,label={-}]
			\item n\_ classify
			\item data\_ test
			\item label\_ test
		\end{itemize}
		& Array, Array & \textit{Method} ini digunakan untuk menghapus data sarkasme pada klasifikasi tiga kelas, dan menghapus data negatif dan netral pada klasifikasi 1 kelas. \\
		\hline
	\end{longtable}
\end{small}

\addtocounter{table}{-1}
\subsection{\textit{Class} FeatureExtraction}
\textit{Class }FeatureExtraction merupakan \textit{class }yang digunakan untuk mengekstraksi atau mengambil nilai dari sebuah \textit{tweet} yang akan digunakan sebagai fitur. Berikut ini adalah \textit{method} pada \textit{class }FeatureExtraction:
\begin{table}[H]
	\caption{Daftar \textit{Method} pada \textit{Class} FeatureExtraction}
	\centering
	\small
	\begin{adjustbox}{width=1\textwidth}		
	\begin{tabular}{|p{5cm} p{3.1cm} p{1cm} p{3.1cm}|}
		\hline
		\multicolumn{2}{|l}{\textbf{Variabel:}}&\multicolumn{2}{l|}{\textbf{Variabel:}}\\
		Preprocessing (Class Lib)&preprocess&Int&max\_qout\\
		SentimentExtraction (Class)&sentiments&Int&max\_cap\\
		Tagger (Class Lib)&tagger&Int&max\_excl\\
		Array (String)&interjection\_dic&Int&max\_quest\\
		\hline
	\end{tabular}
	\end{adjustbox}
\end{table}
\begin{small}
	\begin{longtable}{@{\extracolsep{\fill}}|p{0.4cm}|p{3.2cm}|p{1.4cm}|p{1.7cm}|p{1.20cm}|p{3.35cm}|}
		\hline
		\multirow{2}{*}{\textbf{No}} & \multirow{2}{*}{\textit{\textbf{Method}}} & \multicolumn{2}{c|}{\textit{\textbf{Input}}} & \multirow{2}{*}{\textit{\textbf{Output}}} & 
		\multirow{2}{*}{\textbf{Keterangan}}\\
		\cline{3-4}
		& & \textbf{Tipe} & \textbf{Variabel} & & \\
		\hline
		\endhead
		1 & unigram & \begin{itemize}[leftmargin=*,label={-}]
			\item Array\item String\item Array\end{itemize}
		& \begin{itemize}[leftmargin=*,label={-}]
			\item features\item tweet\item IDF\end{itemize}
		& Array & \textit{Method} ini digunakan untuk menghitung kemunculan kata pada teks, dan mengembalikan nilai fitur kata yang sudah dihitung menggunakan TF-IDF \\
		\hline
		2 & tagging & \begin{itemize}[leftmargin=*,label={-}]
			\item String\end{itemize}
		& \begin{itemize}[leftmargin=*,label={-}]
			\item tweet\end{itemize}
		& String & \textit{Method} ini digunakan untuk melakukan \textit{tagging} terhadap setiap kata, sebagai contoh "mementingkan/ VBT" \\
		\hline
		3 & part\_of\_speech & \begin{itemize}[leftmargin=*,label={-}]
			\item Array\item String\end{itemize}
		& \begin{itemize}[leftmargin=*,label={-}]
			\item features\item tweet\end{itemize}
		& Array & Method ini digunakan untuk menerima \textit{tweet} yang sudah diberi \textit{tag}, dan akan menghitung kemunculan \textit{tag} seperti kemunculan kata benda, kata sifat, kata kerja, kata keterangan dan kata negasi. \\
		\hline
		4 & sentiment\_score & \begin{itemize}[leftmargin=*,label={-}]
			\item Array\item String\end{itemize}
		& \begin{itemize}[leftmargin=*,label={-}]
			\item features\item tweet\end{itemize}
		& Array & \textit{Method} ini digunakan untuk menerima parameter \textit{tweet} yang sudah diberi \textit{tag}, kemudian melakukan \textit{scoring }sentimen terhadap \textit{tweet}. \\
		\hline
		5 & punctuation\_based & \begin{itemize}[leftmargin=*,label={-}]
			\item Array\item String\item String\end{itemize}
		& \begin{itemize}[leftmargin=*,label={-}]
			\item features\item tweet\item type\end{itemize}
		& Array & \textit{Method} ini digunakan untuk menghitung kemunculan tanda tanya (?), tanda seru (!), tanda petik (") dan tanda petik tunggal ('). Setiap kemunculan akan dibagi jumlah kemunculan terbanyak pada data \textit{training}. \\
		\hline
		6 & capitalization & \begin{itemize}[leftmargin=*,label={-}]
			\item Array\item String\end{itemize}
		& \begin{itemize}[leftmargin=*,label={-}]
			\item features\item tweet\end{itemize}
		& Array & \textit{Method} ini digunakan untuk meghitung kemunculan kata kapital kemudian jumlah kemunculan dibagi jumlah kemunculan terbanyak pada data \textit{training}. \\
		\hline
		7 & stopword\_removal & \begin{itemize}[leftmargin=*,label={-}]
			\item String\end{itemize}
		& \begin{itemize}[leftmargin=*,label={-}]
			\item tweet\end{itemize}
		& String & \textit{Method} ini digunakan untuk menghilangkan kata-kata yang dianggap tidak memiliki makna. \\
		\hline
		8 & Topic & \begin{itemize}[leftmargin=*,label={-}]
			\item Array\item String\item Class\end{itemize}
		& \begin{itemize}[leftmargin=*,label={-}]
			\item features\item tweet\item topic\_ mod\end{itemize}
		& Array & \textit{Method} ini digunakan untuk menerima \textit{class} topic yang sudah dilatih pada data \textit{training}, dan digunakan untuk mendapatkan topik dari \textit{tweet}. \\
		\hline
		9 & interjection\_word & \begin{itemize}[leftmargin=*,label={-}]
			\item Array\item String\end{itemize}
		& \begin{itemize}[leftmargin=*,label={-}]
			\item features\item tweet\end{itemize}
		& Array & \textit{Method} ini digunakan untuk menghitung jumlah kemunculan kata \textit{interjection }seperti, "wow", "wahh". \\
		\hline
		10 & question\_word & \begin{itemize}[leftmargin=*,label={-}]
			\item Array\item String\end{itemize}
		& \begin{itemize}[leftmargin=*,label={-}]
			\item features\item tweet\end{itemize}
		& Array & \textit{Method} ini digunakan untuk memberikan nilai fitur kata tanya sebagai 1 (\textit{true}) atau 0 (\textit{false}), jika terdapat kata tanya pada \textit{tweet}. \\
		\hline
		11 & Idf & \begin{itemize}[leftmargin=*,label={-}]
			\item Array
			\item Array
		\end{itemize}
		& \begin{itemize}[leftmargin=*,label={-}]
			\item data\item label\end{itemize}
		& Array & \textit{Method} ini digunakan untuk menghitung nilai IDF dari masukan data \textit{training}. \\
		\hline
	\end{longtable}
\end{small}

\addtocounter{table}{-1}
\subsection{\textit{Class} SentimentExtraction}
\textit{Class }SentimentExtraction merupakan \textit{class }yang digunakan untuk mendapatkan nilai sentimen positif dan negatif dari sebuah \textit{tweet}. Berikut ini adalah \textit{method} pada \textit{class }SentimentExtraction:
\begin{table}[H]
	\caption{Daftar \textit{Method} pada \textit{Class} SentimentExtraction}
	\centering
	\small
	\begin{adjustbox}{width=1\textwidth}		
	\begin{tabular}{|p{5cm} p{2.1cm} p{3cm} p{2.1cm}|}
		\hline
		\multicolumn{2}{|l}{\textbf{Variabel:}}&\multicolumn{2}{l|}{\textbf{Variabel:}}\\
		Defaultdict (Collection)&senti\_score&Negation (class)&negasi\\
		Preprocessing (Class)&preprocess& & \\
		\hline
	\end{tabular}
	\end{adjustbox}
\end{table}

\begin{table}[H]
	\centering
	\small
	\begin{adjustbox}{width=1\textwidth}	
	\begin{tabular}{|p{0.4cm}|p{3.2cm}|p{1.4cm}|p{1.7cm}|p{1.20cm}|p{3.35cm}|}
		\hline
		\multirow{2}{*}{\textbf{No}} & \multirow{2}{*}{\textit{\textbf{Method}}} & \multicolumn{2}{c|}{\textit{\textbf{Input}}} & \multirow{2}{*}{\textit{\textbf{Output}}} & 
		\multirow{2}{*}{\textbf{Keterangan}}\\
		\cline{3-4}
		& & \textbf{Tipe} & \textbf{Variabel} & & \\
		\hline
		1 & score\_sentence & \begin{itemize}[leftmargin=*,label={-}]
			\item Array\end{itemize}
		& \begin{itemize}[leftmargin=*,label={-}]
			\item tweet\end{itemize}
		& Array [][] & \textit{Method} ini akan menerima masukan array \textit{tweet} yang sudah diberi \textit{tag}, kemudian melakukan iterasi setiap kata, dan memanggil \textit{method} score\_word untuk mendapatkan nilai sentimen positif dan negatif. \\
		\hline
		2 & score\_word & \begin{itemize}[leftmargin=*,label={-}]
			\item String\item String\end{itemize}
		& \begin{itemize}[leftmargin=*,label={-}]
			\item word\item tag\end{itemize}
		& Array [][] & \textit{Method} ini akan memberi nilai sentimen positif dan negatif berdasarkan kata dan \textit{tag }dari kata. \\
		\hline
	\end{tabular}
	\end{adjustbox}
\end{table}
\subsection{\textit{Class} Negation}
\textit{Class }Negation merupakan \textit{class} yang digunakan untuk mengatasi kata negasi yang terdapat pada \textit{tweet}. Berikut ini adalah \textit{method }pada \textit{class }Negation:

\begin{table}[H]
	\caption{Daftar \textit{Method} pada \textit{Class} Negation}
	\centering
	\small
	\begin{adjustbox}{width=1\textwidth}	
	\begin{tabular}{|p{4cm} p{9cm}|}
		\hline
		\multicolumn{2}{|l|}{\textbf{Variabel:}}\\
		Preprocessing (Class)&preprocess\\
		\hline
	\end{tabular}
	\end{adjustbox}
\end{table}
\begin{table}[H]
	\centering
	\small
	\begin{adjustbox}{width=1\textwidth}	
	\begin{tabular}{|p{0.4cm}|p{3.2cm}|p{1.4cm}|p{1.7cm}|p{1.20cm}|p{3.35cm}|}
		\hline
		\multirow{2}{*}{\textbf{No}} & \multirow{2}{*}{\textit{\textbf{Method}}} & \multicolumn{2}{c|}{\textit{\textbf{Input}}} & \multirow{2}{*}{\textit{\textbf{Output}}} & 
		\multirow{2}{*}{\textbf{Keterangan}}\\
		\cline{3-4}
		& & \textbf{Tipe} & \textbf{Variabel} & & \\
		\hline
		1 & negation\_handling & \begin{itemize}[leftmargin=*,label={-}]
			\item String
			\item String
			\item Array [][]
			\item Bool\end{itemize}
		& \begin{itemize}[leftmargin=*,label={-}]
			\vspace{-0.5cm}
			\item prev
			\item pprev
			\item score
			\item change\end{itemize}
		& Array [][], Bool & \textit{Method} ini akan digunakan pada \textit{sentiment} \textit{score}, untuk melakukan pengecekan kata negasi, jika terdapat kata negasi maka nilai dari kata setelah kata negasi akan ditambah dan dikali dua. \\
		\hline
	\end{tabular}
	\end{adjustbox}
\end{table}
\subsection{\textit{Class} Topic}
\textit{Class }Topic merupakan \textit{class} yang digunakan untuk melakukan \textit{topic} \textit{modelling} yang akan dilatih dari data \textit{training}. Berikut ini adalah \textit{method }pada \textit{class }Topic:
\begin{table}[H]
	\caption{Daftar \textit{Method} pada \textit{Class} Topic}
	\centering
	\small
	\begin{adjustbox}{width=1\textwidth}	
	\begin{tabular}{|p{4cm} p{2.1cm} p{4cm} p{2.1cm}|}
		\hline
		\multicolumn{2}{|l}{\textbf{Variabel:}}&\multicolumn{2}{l|}{\textbf{Variabel:}}\\
		Int&nbtopic&String&alpha\\
		Preprocessing (Class)&preprocess&LdaModel (Class Lib)&lda \\
		Corpora (Class Lib)&dictionary& & \\
		\hline
	\end{tabular}
	\end{adjustbox}
\end{table}
\begin{table}[H]
	\centering
	\small
	\begin{adjustbox}{width=1\textwidth}	
	\begin{tabular}{|p{0.4cm}|p{3.2cm}|p{1.4cm}|p{1.7cm}|p{1.20cm}|p{3.35cm}|}
		\hline
		\multirow{2}{*}{\textbf{No}} & \multirow{2}{*}{\textit{\textbf{Method}}} & \multicolumn{2}{c|}{\textit{\textbf{Input}}} & \multirow{2}{*}{\textit{\textbf{Output}}} & 
		\multirow{2}{*}{\textbf{Keterangan}}\\
		\cline{3-4}
		& & \textbf{Tipe} & \textbf{Variabel} & & \\
		\hline
		1 & Fit & \begin{itemize}[leftmargin=*,label={-}]
			\item Array\end{itemize}
		& \begin{itemize}[leftmargin=*,label={-}]
			\vspace{-0.5cm}
			\item tweet\end{itemize}
		& Array & \textit{Method} ini akan digunakan untuk melakukan \textit{topic} \textit{modelling} dengan \textit{library} LDA dari gensim. 
		\\
		\hline
		2 & transform & \begin{itemize}[leftmargin=*,label={-}]
			\vspace{-0.5cm}
			\item String\end{itemize}
		& \begin{itemize}[leftmargin=*,label={-}]
			\vspace{-0.5cm}
			\item tweet\end{itemize}
		& Array & \textit{Method} ini akan digunakan untuk menadapatkan topik dari sebuah \textit{tweet}. \\
		\hline
	\end{tabular}
	\end{adjustbox}
\end{table}
\subsection{\textit{Class} SVM}
\textit{Class }SVM merupakan \textit{class }yang digunakan untuk melakukan \textit{training} pada data \textit{training} dan mendapatkan hasil dari klasifikasi pada data \textit{testing}. Berikut ini adalah \textit{method} pada \textit{class }SVM:
\begin{table}[H]
	\caption{Daftar \textit{Method} pada \textit{Class} SVM}
	\centering
	\small
	\begin{adjustbox}{width=1\textwidth}	
	\begin{tabular}{|p{4cm} p{2.1cm} p{4cm} p{2.1cm}|}
		\hline
		\multicolumn{2}{|l}{\textbf{Variabel:}}&\multicolumn{2}{l|}{\textbf{Variabel:}}\\
		Float&c&Array&alphas\\
		Float&tol&Array&bias \\
		Int&max\_passes&FMeasure (class)&f\_measure\\
		\hline
	\end{tabular}
	\end{adjustbox}
\end{table}
\begin{small}
	\begin{longtable}{@{\extracolsep{\fill}}|p{0.4cm}|p{3.2cm}|p{1.4cm}|p{1.9cm}|p{1.20cm}|p{3.35cm}|}
		\hline
		\multirow{2}{*}{\textbf{No}} & \multirow{2}{*}{\textit{\textbf{Method}}} & \multicolumn{2}{c|}{\textit{\textbf{Input}}} & \multirow{2}{*}{\textit{\textbf{Output}}} & 
		\multirow{2}{*}{\textbf{Keterangan}}\\
		\cline{3-4}
		& & \textbf{Tipe} & \textbf{Variabel} & & \\
		\hline
		\endhead
		1 & smo & \begin{itemize}[leftmargin=*,label={-}]
			\item Array\item Array\end{itemize}
		& \begin{itemize}[leftmargin=*,label={-}]
			\item features\item label\end{itemize}
		& Array, Float & \textit{Method} ini akan digunakan untuk mendapatkan nilai alpha dan bias, dari data \textit{training}.\textit{ } \\
		\hline
		2 & error & \begin{itemize}[leftmargin=*,label={-}]
			\item Array\item Array\item Array\item Float\item Int\end{itemize}
		& \begin{itemize}[leftmargin=*,label={-}]
			\item feature\item label\item alpha\item bias\item i\end{itemize}
		& Float & \textit{Method} ini akan digunakan untuk mendapatkan nilai \textit{error }yang akan digunakan pada method smo. \\
		\hline
		3 & classification & \begin{itemize}[leftmargin=*,label={-}]
			\item Array\item Float\item Array\item Array\item Array\item Array\item 
			Array\end{itemize}
		& \begin{itemize}[leftmargin=*,label={-}]
			\item alpha
			\item b
			\item feature\_ train
			\item feature\_ test
			\item label\_ train
			\item label\_ test
			\item model\_ type
		\end{itemize}
		& Array, Array, Array, Array & \textit{Method} ini untuk menghitung hasil prediksi menggunakan persamaan SVM, mengembalikan nilai \textit{f-measure, }akurasi dari setiap metode klasifikasi yang digunakan, dan hasil prediksi. \\
		\hline
		4 & f\_measure\_sarkasme & 
		\begin{itemize}[leftmargin=*,label={-}]
			\item Array
			\item Array
		\end{itemize}
		& \begin{itemize}[leftmargin=*,label={-}]
			\item value\_ prediction
			\item label\_ actual
		\end{itemize}
		& Float & \textit{Method} ini digunakan untuk mengubah nilai value\_prediction menjadi nilai 1 atau -1, khusus klasifikasi 1 kelas. \\
		\hline
		5 & direct\_method & \begin{itemize}[leftmargin=*,label={-}]
			\item Array\item Array\end{itemize}
		& \begin{itemize}[leftmargin=*,label={-}]
			\item value\_ prediction\item label\_ actual\end{itemize}
		& Array, Array, Array & \textit{Method} ini digunakan untuk melakukan klasifikasi dengan \textit{direct method}. Keluaran dari \textit{method} ini adalah \textit{f-measure} dan prediksi. \\
		\hline
		6 & levelled\_method & \begin{itemize}[leftmargin=*,label={-}]
			\item Array\item Array\end{itemize}
		& \begin{itemize}[leftmargin=*,label={-}]
			\item value\_ prediction\item label\_ actual\end{itemize}
		& Array, Array, Array & \textit{Method} ini digunakan untuk melakukan 		klasifikasi dengan \textit{levelled method}. Keluaran dari method ini adalah \textit{f-measure} dan prediksi. \\
		\hline
		7 & convert\_label & \begin{itemize}[leftmargin=*,label={-}]
			\item Array\item String\end{itemize}
		& \begin{itemize}[leftmargin=*,label={-}]
			\item label\item type\end{itemize}
		& Array & \textit{Method} ini digunakan untuk mengubah label dari data sesuai dengan kelas yang akan melalui proses \textit{training}. \\
		\hline
		8 & get\_data\_label & \begin{itemize}[leftmargin=*,label={-}]
			\item String\end{itemize}
		& \begin{itemize}[leftmargin=*,label={-}]
			\item prediksi\end{itemize}
		& String & \textit{Method} ini digunakan untuk mendapatkan nilai label seperti semula. Sebagai contoh, jika prediksi "sarkasme" maka akan diubah kembali menjadi 2, untuk melalui proses perhitungan akurasi. \\
		\hline
		9 & save\_alpha\_bias & \begin{itemize}[leftmargin=*,label={-}]
			\item Int\item Array\item Float\end{itemize}
		& \begin{itemize}[leftmargin=*,label={-}]
			\item n\_classify\item alpha\item bias\end{itemize}
		& - & \textit{Method} ini untuk menyimpan nilai alpha dan bias ke dalam \textit{file}. \\
		\hline
	\end{longtable}
\end{small}

\addtocounter{table}{-1}
\subsection{\textit{Class} FMeasure}
\textit{Class} FMeasure merupakan \textit{class} yang digunakan untuk mendapatkan akurasi dari klasifikasi data \textit{testing}. Berikut ini adalah \textit{method} pada \textit{class }FMeasure:
\begin{table}[H]
	\caption{Daftar \textit{Method} pada \textit{Class} FMeasure}
	\centering
	\small
	\begin{adjustbox}{width=1\textwidth}	
	\begin{tabular}{|p{4cm} p{2.1cm} p{4cm} p{2.1cm}|}
		\hline
		\multicolumn{2}{|l}{\textbf{Variabel:}}&\multicolumn{2}{l|}{\textbf{Variabel:}}\\
		Float&c&Array&alphas\\
		Float&tol&Array&bias \\
		Int&max\_passes&FMeasure (class)&f\_measure\\
		\hline
	\end{tabular}
	\end{adjustbox}
\end{table}
\begin{small}	
	\begin{longtable}{@{\extracolsep{\fill}}|p{0.4cm}|p{3.2cm}|p{1.4cm}|p{1.9cm}|p{1.20cm}|p{3.35cm}|}
		\hline
		\multirow{2}{*}{\textbf{No}} & \multirow{2}{*}{\textit{\textbf{Method}}} & \multicolumn{2}{c|}{\textit{\textbf{Input}}} & \multirow{2}{*}{\textit{\textbf{Output}}} & 
		\multirow{2}{*}{\textbf{Keterangan}}\\
		\cline{3-4}
		& & \textbf{Tipe} & \textbf{Variabel} & & \\
		\hline
		\endhead
		1 & confusion\_matrix & \begin{itemize}[leftmargin=*,label={-}]
			\item Array\item Array\end{itemize}
		& \begin{itemize}[leftmargin=*,label={-}]
			\item label\_ actual\item label\_ prediction\end{itemize}
		& Array, Array, Array & \textit{Method} ini digunakan untuk mendapatkan nilai \textit{True Positive, False Positive} dan \textit{False Negative} dari label prediksi dan label sebenarnya. \\
		\hline
		2 & f\_measure\_ all & \begin{itemize}[leftmargin=*,label={-}]
			\item Array\item Array\item Array\end{itemize}
		& \begin{itemize}[leftmargin=*,label={-}]
			\item TP\item FP\item FN\end{itemize}
		& Array & \textit{Method} ini digunakan untuk menghitung f\_measure dari setiap kelas yang ada. \\
		\hline
		3 & precision\_ recall & \begin{itemize}[leftmargin=*,label={-}]
			\item Int\item Int\end{itemize}
		& \begin{itemize}[leftmargin=*,label={-}]
			\item TP\item FP\_FN\end{itemize}
		& Float & \textit{Method} ini untuk mendapatkan nilai \textit{
			precision} dan \textit{recall} berdasarkan \textit{true positive, false positive} dan \textit{false negative }yang sudah didapatkan. \\
		\hline
		4 & accuracy & \begin{itemize}[leftmargin=*,label={-}]
			\item Int\item Int\end{itemize}
		& \begin{itemize}[leftmargin=*,label={-}]
			\item precision\item recall\end{itemize}
		& Float & \textit{Method} ini digunakan untuk mendapatkan \textit{f-measure} berdasarkan nilai \textit{precision} dan \textit{recall}. \\
		\hline
	\end{longtable}
\end{small}

\addtocounter{table}{-1}
\subsection{\textit{Class} Learning}
\textit{Class }Learning merupakan \textit{class }yang digunakan untuk melakukan pemodelan pada data \textit{training} dan klasifikasi pada data \textit{testing} dimulai dari pengambil data menggunakan \textit{class }Loader, melakukan \textit{preprocessing }menggunakan \textit{class }Preprocessing, mendapatkan \textit{feature set} menggunakan \textit{class }Feature hingga melakukan klasifikasi 
menggunakan \textit{class }SVM. Berikut ini adalah \textit{method} pada \textit{class }Learning:
\begin{table}[H]
	\caption{Daftar \textit{Method} pada \textit{Class} FMeasure}
	\centering
	\small
	\begin{adjustbox}{width=1\textwidth}	
	\begin{tabular}{|p{4cm} p{1cm} p{4cm} p{3.2cm}|}
		\hline
		\multicolumn{2}{|l}{\textbf{Variabel:}}&\multicolumn{2}{l|}{\textbf{Variabel:}}\\
		Loader (Class)&load&Preprocessing (Class)&preprocess\\
		Features (Class)&fitur&SVM (Class)&svm\_classifier \\
		\hline
	\end{tabular}
	\end{adjustbox}
\end{table}
\begin{table}[H]
	\centering
	\small
	\begin{adjustbox}{width=1\textwidth}
	\begin{tabular}{|p{0.4cm}|p{3.2cm}|p{1.4cm}|p{1.7cm}|p{1.20cm}|p{3.35cm}|}
		\hline
		\multirow{2}{*}{\textbf{No}} & \multirow{2}{*}{\textit{\textbf{Method}}} & \multicolumn{2}{c|}{\textit{\textbf{Input}}} & \multirow{2}{*}{\textit{\textbf{Output}}} & 
		\multirow{2}{*}{\textbf{Keterangan}}\\
		\cline{3-4}
		& & \textbf{Tipe} & \textbf{Variabel} & & \\
		\hline
		1 & get\_model\_type & \begin{itemize}[leftmargin=*,label={-}]
			\item Int\end{itemize}
		& \begin{itemize}[leftmargin=*,label={-}]
			\item n\_ classify\end{itemize}
		& Array & \textit{Method} ini akan digunakan untuk mengembalikan jenis 		klasifikasi yang akan dilakukan, jika n\_classify=0, maka akan dilakukan klasifikasi 4 kelas yaitu positif, negatif, netral dan sarkasme. Jika n\_classify=1, akan melakukan klasifikasi tanpa sarkasme. Jika n\_classify=2, akan melakukan klasifikasi sarkasme atau non-sarkasme. \\
		\hline
		2 & model & \begin{itemize}[leftmargin=*,label={-}]
			\item String\end{itemize}
		& \begin{itemize}[leftmargin=*,label={-}]
			\item file name\end{itemize}
		& - & \textit{Method} ini berguna untuk melakukan \textit{learning}	terhadap data \textit{training} dan melakukan klasifikasi terhadap data \textit{testing}. \\
		\hline
	\end{tabular}
	\end{adjustbox}
\end{table}
\subsection{\textit{Class} Main}
\textit{Class }Main merupakan \textit{class }yang digunakan untuk menjalankan sistem berupa \textit{website} pada \textit{localhost}. Berikut ini adalah \textit{method} pada \textit{class }Main:
\begin{table}[H]
	\caption{Daftar \textit{Method} pada \textit{Class} FMeasure}
	\centering
	\small
	\begin{adjustbox}{width=1\textwidth}
	\begin{tabular}{|p{4cm} p{9cm}|}
		\hline
		\multicolumn{2}{|l|}{\textbf{Variabel:}}\\
		Learning (Class)&learn\\
		\hline
	\end{tabular}
	\end{adjustbox}
\end{table}
\begin{table}[H]
	\centering
	\small
	\begin{adjustbox}{width=1\textwidth}
	\begin{tabular}{|p{0.4cm}|p{3.2cm}|p{1.4cm}|p{1.7cm}|p{1.20cm}|p{3.35cm}|}
		\hline
		\multirow{2}{*}{\textbf{No}} & \multirow{2}{*}{\textit{\textbf{Method}}} & \multicolumn{2}{c|}{\textit{\textbf{Input}}} & \multirow{2}{*}{\textit{\textbf{Output}}} & 
		\multirow{2}{*}{\textbf{Keterangan}}\\
		\cline{3-4}
		& & \textbf{Tipe} & \textbf{Variabel} & & \\
		\hline
		1 & classify & \begin{itemize}[leftmargin=*,label={-}]
			\item String\end{itemize}
		& \begin{itemize}[leftmargin=*,label={-}]
			\item teks\end{itemize}
		& String & \textit{Method} ini untuk mengembalikan hasil klasifikasi 
		dari teks masukan\textit{ }kepada UI. \\
		\hline
		2 & upload & - & - & Array, Array, Array, Array, Array, Array, Array & 
		\textit{Method} ini digunakan untuk menerima \textit{file} yang diisi pada \textit{form }masukan \textit{file} pada website, kemudian melakukan \textit{training} pada \textit{file} tersebut. Hasil akurasi dan data yang digunakan untuk \textit{training }dan \textit{testing} akan ditampilkan pada tampilan. \\
		\hline
		3 & home & - & - & - & \textit{Method} ini untuk melakukan \textit{redirect} ke halaman awal website. \\
		\hline
	\end{tabular}
	\end{adjustbox}
\end{table}
\section{Implementasi Perangkat Lunak}
Pada bagian ini akan dijelaskan mengenai implementasi aplikasi analisis sentimen dimulai dari pengambilan data, \textit{text preprocessing}, \textit{feature extraction}, hingga klasifikasi SVM dengan SMO.
\subsection{Implementasi Pengambilan Data}
Pada bagian ini akan dilakukan proses pengambilan data dari \textit{file} ekstensi .txt. Berikut ini adalah proses yang akan dilakukan dalam pengambilan data:
\begin{table}[H]
	\normalsize
	\begin{adjustbox}{width=1\textwidth}
	\begin{tabular}{|p{13.55cm}|}
		\hline
		\begin{enumerate}[leftmargin=*]
			\item Memilih data \textit{file} dengan ekstensi .txt yang berisi data dan labelnya yang dipisahkan dengan \textit{tab }($\backslash$t)
			\item Membuka \textit{file} dengan \textit{method} pada Python, yaitu open(\textit{filename}), kemudian memisahkan data dan label berdasarkan \textit{tab} ($\backslash$t) dengan \textit{method} pada Python reader(\textit{file}, \textit{delimiter }= "$\backslash$t").
			\item Melakukan \textit{looping }atau pengulangan pada data yang sudah dipisahkan berdasarkan \textit{tab},	dan menyimpan data beserta labelnya ke dalam variabel dengan tipe	array.
		\end{enumerate}\\
		\hline
	\end{tabular}
	\end{adjustbox}
\end{table}
\subsection{Implementasi \textit{Text Preprocessing}}
Sebelum melakukan pengambilan data beserta labelnya, setiap data akan melalui proses \textit{text preprocessing} dengan melakukan iterasi atau pengulangan pada data yang sudah disimpan pada array. \textit{Text preprocessing }yang dilakukan secara umum adalah \textit{remove 	hashtag}, URL, \textit{mention}, \textit{remove} \textit{punctuation}, \textit{tokenize}, \textit{stopword removal}, yang lainnya akan dijalankan saat penambahan fitur. Sebagai contoh fitur \textit{part of speech} hanya akan melakukan \textit{case folding} Berikut ini merupakan proses yang akan dilakukan dalam \textit{text preprocessing}:

\begin{enumerate}[leftmargin=*]
	\normalsize
	\item \textit{Case Folding}\\
	\begin{adjustbox}{width=1\width}
	\framebox[\linewidth]{\parbox{\linewidth}{\begin{enumerate}[label={}, leftmargin=5pt,rightmargin=5pt]
				\item Mengubah setiap huruf pada \textit{tweet} menjadi huruf kecil dengan \textit{method} variabel.lower().
			\end{enumerate}}}
	\end{adjustbox}
	\item \textit{Remove} \textit{Hashtag, }URL, \textit{Mention}\\
	\begin{adjustbox}{width=1\width}
	\framebox[\linewidth]{\parbox{\linewidth}{\begin{enumerate}[label={}, leftmargin=5pt,rightmargin=5pt]
				\item Menghapus semua \textit{hashtag}, URL, \textit{mention} pada \textit{tweet} dengan \textit{method} pada \textit{library }Tweet-preprocessor, yaitu clean(\textit{tweet}).
			\end{enumerate}}}
	\end{adjustbox}
	\item \textit{Remove Punctuation}\\
	\begin{adjustbox}{width=1\width}
	\framebox[\linewidth]{\parbox{\linewidth}{\begin{enumerate}[label={\arabic*}, leftmargin=15pt,rightmargin=5pt]
				\item Menginisialisasikan tanda baca yang ada ke sebuah variabel dengan tipe array dengan \textit{method }pada Python, yaitu set(string.punctuation)
				\item Menghapus semua tanda baca kecuali tanda baca tanya (?), tanda seru (!), tanda petik(", ') dan tanda pemisah (-) dengan melakukan iterasi setiap \textit{character }pada \textit{tweet}, dan menghapus kemunculan tanda baca yang terdapat pada variabel tanda baca pada tahap 1.
			\end{enumerate}}}
	\end{adjustbox}
	\item \textit{Tokenization}\\
	\begin{adjustbox}{width=1\width}
	\framebox[\linewidth]{\parbox{\linewidth}{\begin{enumerate}[label={}, leftmargin=5pt,rightmargin=5pt]
				\item Melakukan \textit{tokenize} pada \textit{tweet} dengan \textit{method} pada \textit{library }NLTK, yaitu word\_tokenize(\textit{tweet}).
			\end{enumerate}}}
	\end{adjustbox}
	\item \textit{Misuse of Word}\\
	\begin{adjustbox}{width=1\width}
	\framebox[\linewidth]{\parbox{\linewidth}{\begin{enumerate}[label={}, leftmargin=5pt,rightmargin=5pt]
				\item Menggabungkan setiap huruf yang sama dan bersebelahan, dengan melakukan iterasi setiap token yang dihasilkan pada tahap \textit{tokenization}, dengan \textit{method }itertools.groupby(token).
			\end{enumerate}}}
	\end{adjustbox}
	\item \textit{Abbreviation Word}\\
	\begin{adjustbox}{width=1\width}
	\framebox[\linewidth]{\parbox{\linewidth}{\begin{enumerate}[label={\arabic*}, leftmargin=15pt,rightmargin=5pt]
				\item Menginisialisasi kamus kata singkatan yang dibuat secara manual pada \textit{file} ekstensi .txt ke dalam variabel \textit{abbreviation\_dic }tipe array.
				\item Melakukan iterasi setiap token dan menggantinya dengan persamaannya jika terdapat pada variabel kamus kata singkatan.
			\end{enumerate}}}
	\end{adjustbox}
	\pagebreak
	\item \textit{Stopword Removal}\\
	\begin{adjustbox}{width=1\width}
	\framebox[\linewidth]{\parbox{\linewidth}{\begin{enumerate}[label={\arabic*}, leftmargin=15pt,rightmargin=5pt]
				\item Menginisialisasi kamus kata \textit{stopword} pada \textit{file} ekstensi .txt ke dalam variabel stopword tipe array.
				
				\item Melakukan	iterasi setiap token dan menghapus token tersebut jika terdapat pada	variabel stopword.
			\end{enumerate}}}
	\end{adjustbox}
	\item \textit{Stemming}\\
	\begin{adjustbox}{width=1\width}
	\framebox[\linewidth]{\parbox{\linewidth}{\begin{enumerate}[label={}, leftmargin=5pt,rightmargin=5pt]
				\item Mengubah kata menjadi bentuk kata dasarnya dengan \textit{method} pada \textit{library }sastrawi, yaitu stem(\textit{tweet}).
			\end{enumerate}}}
	\end{adjustbox}
\end{enumerate}
\subsection{Implementasi \textit{Feature Extraction}}
Setelah \textit{text preprocessing}, akan dilakukan \textit{feature extraction }untuk mendapatkan nilai fitur dari setiap teks. Berikut ini merupakan proses yang akan dilakukan dalam \textit{feature extraction}:
\begin{enumerate}[leftmargin=*]
	\normalsize
	\item \textit{Unigram}\\
	\begin{adjustbox}{width=1\width}
	\framebox[\linewidth]{\parbox{\linewidth}{\begin{enumerate}[label={\arabic*}, leftmargin=15pt,rightmargin=5pt]
				\item Melakukan \textit{case folding} terhadap \textit{tweet}, \textit{stemming}, \textit{misuse of word, tokenize}.
				\item Menghitung jumlah kemunculan kata pada \textit{tweet} dan disimpan ke	dalam variabel tipe object (\{\}).
				\item Menghitung nilai TF setiap kata dan dikalikan dengan IDF katanya. \item Hasil fitur akan disimpan dalam	variabel features tipe object (\{\})
			\end{enumerate}}}
	\end{adjustbox}

	\item \textit{Part of Speech}\\
	\begin{adjustbox}{width=1\width}	
	\framebox[\linewidth]{\parbox{\linewidth}{\begin{enumerate}[label={\arabic*}, leftmargin=15pt,rightmargin=5pt]
				\item Melakukan \textit{case folding} terhadap \textit{tweet}.
				\item Melakukan \textit{tagging }terhadap \textit{tweet} dengan \textit{method} pada \textit{library }IPosTagger, yaitu taggingStr(\textit{tweet})
				\item Hasil \textit{tweet} yang sudah diberi \textit{tag} akan dilakukan perhitungan kemunculan setiap \textit{tag}.
				\item Hasil	fitur akan disimpan dalam variabel features tipe object (\{\})
			\end{enumerate}}}	
	\end{adjustbox}
	
	\item \textit{Sentiment Score}\\
	\begin{adjustbox}{width=1\width}
	\framebox[\linewidth]{\parbox{\linewidth}{\begin{enumerate}[label={\arabic*}, leftmargin=15pt,rightmargin=5pt]
				\item Menginisialisasi variabel senti\_score dengan tipe defaultdict untuk menyimpan kata, \textit{tag} beserta nilai sentimen positif dan negatifnya.
				\item Melakukan \textit{case folding} terhadap \textit{tweet}.
				\item Melakukan \textit{tagging }terhadap \textit{tweet} dengan \textit{method} pada \textit{library }IPosTagger, yaitu taggingStr(\textit{tweet})
				\item Hasil \textit{tweet} yang sudah diberi \textit{tag} akan dicek ke dalam variabel senti\_score untuk mendapatkan nilai sentimennya. 
				\item Hasil fitur akan disimpan dalam variabel features tipe object (\{\})
			\end{enumerate}}}
	\end{adjustbox}	
	\pagebreak
		\item \textit{Punctuation Based}\\
	\begin{adjustbox}{width=1\width}	
		\framebox[\linewidth]{\parbox{\linewidth}{\begin{enumerate}[label={\arabic*}, leftmargin=15pt,rightmargin=5pt]
					\item Menghitung kemunculan tanda baca tanya (?), tanda seru (!) dan tanda petik (', ") dengan \textit{method} pada Python, yaitu 
					count(tanda\_baca).
					\item Hasil fitur akan disimpan dalam variabel features tipe object (\{\})
				\end{enumerate}}}
	\end{adjustbox}
	\item \textit{Capitalization}\\
	\begin{adjustbox}{width=1\width}
		\framebox[\linewidth]{\parbox{\linewidth}{\begin{enumerate}[label={\arabic*}, leftmargin=15pt,rightmargin=5pt]
					\item Menghitung kemunculan kata kapital, dan membagi jumlah kemunculan kapital dengan kemunculan maksimal kata kapital pada data \textit{training}.
					\item Hasil fitur akan disimpan dalam variabel features tipe object (\{\})
				\end{enumerate}}}
	\end{adjustbox}
		\item \textit{Topic}\\
	\begin{adjustbox}{width=1\width}
		\framebox[\linewidth]{\parbox{\linewidth}{\begin{enumerate}[label={\arabic*}, leftmargin=15pt,rightmargin=5pt]
					\item Melakukan \textit{training topic modelling }LDA dengan \textit{method }pada \textit{library }gensim, yaitu LdaModel(\textit{corpus}, \textit{dictionary}, jumlah\_topik, alpha). Menyimpan model dan	\textit{dictionary} kedalam \textit{file pickle} dengan \textit{method} save(\textit{filename}). Model dan \textit{dictionary }
					disimpan ke dalam variabel lda dan dictionary.
					\item Setelah mendapatkan model \textit{topic}, lakukan \textit{case\_folding}, \textit{stemming}, \textit{misuse of word}, \textit{tokenize} pada \textit{tweet}.
					\item Melakukan perhitungan kemunculan kata dengan \textit{method} pada \textit{gensim}, yaitu doc2bow(token) dan menyimpan hasilnya ke dalam variabel corpus\_sentence.
					\item Mendapatkan probabilitas \textit{tweet} terhadap topik yang ada dengan memanggil model LDA, yaitu lda$[$corpus\_sentence$]$.
					\item Hasil fitur akan disimpan dalam variabel features tipe object (\{\})
				\end{enumerate}}}
		\end{adjustbox}
		\item \textit{Interjection}\\
		\begin{adjustbox}{width=1\width}
		\framebox[\linewidth]{\parbox{\linewidth}{\begin{enumerate}[label={\arabic*}, leftmargin=15pt,rightmargin=5pt]
					\item Melakukan \textit{case folding}, \textit{misuse of word}, 
					\textit{tokenize}.
					\item Melakukan iterasi dan menghitung kemunculan kata interjeksi
					\item Hasil fitur akan disimpan dalam variabel features tipe object (\{\})
				\end{enumerate}}}
		\end{adjustbox}
		\item \textit{Question Word}\\
		\begin{adjustbox}{width=1\width}
		\framebox[\linewidth]{\parbox{\linewidth}{\begin{enumerate}[label={\arabic*}, leftmargin=15pt,rightmargin=5pt]
					\item Melakukan \textit{case folding}, \textit{misuse of word}, 
					\textit{tokenize}.
					\item Melakukan iterasi dan menghitung kemunculan kata tanya
					\item Hasil fitur akan disimpan dalam variabel features tipe object (\{\})
				\end{enumerate}}}
		\end{adjustbox}
		\item TF-IDF\\
		\begin{adjustbox}{width=1\width}
		\framebox[\linewidth]{\parbox{\linewidth}{\begin{enumerate}[label={\arabic*}, leftmargin=15pt,rightmargin=5pt]
					\item Melakukan \textit{case\_folding}, menghapus semua kata negasi pada variabel array, dan kata tanya pada variabel array. Setelah itu	\textit{stemming}, \textit{misuse of word}, \textit{tokenize}.
					\item Menghitung kemunculan kata pada sejumlah data \textit{training}.
					\item Menghitung nilai IDF yang disimpan pada variabel tipe  defaultdict.
				\end{enumerate}}}
		\end{adjustbox}
\end{enumerate}

Semua hasil fitur ekstraksi pada data \textit{training} akan disimpan ke dalam \textit{file }fitur\_training.pickle dengam \textit{method} pada \textit{library numpy}, yaitu numpy.save(\textit{filename}).
\subsection{Implementasi SVM dengan SMO}
Setelah melalui proses \textit{feature extraction}, selanjutnya melakukan proses \textit{learning} dengan \textit{Simplified Sequential Minimal Optimization }(SMO) dan klasifikasi menggunakan SVM linear. Berikut ini merupakan proses yang dilakukan dalam \textit{learning }dan klasifikasi:

\begin{enumerate}[leftmargin=*]
	\item Menggunakan hasil fitur ekstraksi data \textit{training} untuk mencari nilai alpha dan bias.\item Mengubah setiap label menjadi +1 atau -1 sesuai dengan model yang ingin dibuat, yaitu label\_train$[$'all'$]$ untuk label model positif, 	negatif, netral, dan label\_train$[$'sar'$]$ untuk label model sarkasme.
	\item Menghitung nilai alpha dan bias untuk setiap model menggunakan 
	SMO. Hasil dari perhitungan alpha dan bias akan disimpan kedalam \textit{file pickle} dengan metode numpy.save(\textit{filename}).
	\item Pada tahap testing akan melakukan tahap 1 sampai dengan 2, kemudian menghitung nilai f(x) menggunakan alpha dan bias pada tahap 3 berdasarkan masing-masing model.
	\item Pengklasifikasian pada data \textit{testing} akan dilakukan dengan 2 macam cara klasifikasi, yaitu \textit{direct method }dan \textit{levelled method}.
	\item Pengecekan pada \textit{direct method }akan melakukan perhitungan f(x) pada model positif, negatif, netral dan sarkasme. Kemudian hasil dari model positif, negatif, netral akan dicari nilai tertingginya. Jika nilai yang didapatkan adalah model positif, maka akan melakukan pengecekan model sarkasme. Jika nilai sign(f(x)) $>$=1 maka \textit{tweet} akan diklasifikasikan sebagai sarkasme, jika bukan maka 
	\textit{tweet} akan diklasifikasikan sebagai positif.
	\item Pengecekan pada \textit{levelled method }akan melakukan perhitungan f(x) pada model netral. Jika sign(f(x))$>$=1 maka akan diklasifikasikan \textit{tweet} sebagai netral, sebaliknya termasuk kelas opini. Jika termasuk kelas opini, maka akan melakukan pengecekan pada model positif dan negatif. Perhitungan nilai sign(f(x)) pada model positif dan negatif akan diambil nilai tertinggi. Jika nilai tertinggi adalah pada model positif, maka akan dilakukan pengecekan pada model 
	sarkasme. Jika sign(f(x))$>$=1 maka akan termasuk kelas sarkasme, sebaliknya kelas positif.
\end{enumerate}
\section{Pengujian}
Pada bab ini, akan dilakukan berbagai pengujian untuk menentukan kombinasi fitur terbaik, perbandingan tahap \textit{preprocessing}, menentukan parameter regularisasi SMO terbaik, dan menentukan metode klasifikasi terbaik untuk analisis sentimen.
\subsection{Pengujian Kombinasi Fitur}
Pada bagian ini akan dilakukan pengujian kombinasi fitur untuk mendapatkan kombinasi fitur\textit{ }terbaik. Kombinasi fitur yang akan dilakukan pada pengujian ini ada dua, yaitu KF1 dan KF 2. KF1 adalah kombinasi fitur 1 pada kelas positif, negatif dan netral yang terdiri dari fitur \textit{unigram, sentiment score, question word, }
dan \textit{punctuation based}. KF 2 adalah kombinasi fitur 2 pada kelas sarkasme yang terdiri dari fitur \textit{unigram, sentiment score, topic, part of speech, punctuation based, interjection word} dan \textit{capitalization}. 

\noindent Berikut ini adalah singkatan fitur yang akan digunakan pada tabel 
pengujian:
\begin{enumerate}[leftmargin=*]
	\item U adalah \textit{Unigram}
	\item SS adalah \textit{Sentiment Score}
	\item QW adalah \textit{Question Word}
	\item PB adalah \textit{Punctuation Based}
	\item T adalah \textit{Topic}
	\item POS adalah \textit{Part of Speech}
	\item IW adalah \textit{Interjection Word}
	\item C adalah \textit{Capitalization}.
\end{enumerate}
Berikut ini adalah singkatan metode klasifikasi pada tabel pengujian:
\begin{enumerate}[leftmargin=*]
	\item DM adalah \textit{Direct Method}
	\item LM adalah \textit{Levelled Method}
\end{enumerate}
Berikut ini adalah singkatan dari setiap \textit{f-measure} pada tabel 
pengujian:
\begin{enumerate}[leftmargin=*]
	\item FP\_L/FP\_D adalah \textit{f-measure }positif \textit{Levelled 
		Method }dan \textit{f-measure }positif \textit{Direct Method}.
	\item FG\_L/FG\_D adalah \textit{f-measure }negatif \textit{Levelled 
		Method }dan \textit{f-measure }negatif \textit{Direct Method}.
	\item FT\_L/FT\_D adalah \textit{f-measure }netral \textit{Levelled 
		Method }dan \textit{f-measure }netral \textit{Direct Method}.
	\item FS\_L/FS\_D adalah \textit{f-measure }sarkasme \textit{Levelled 
		Method }dan \textit{f-measure }sarkasme \textit{Direct Method}.
\end{enumerate}

Pengujian dilakukan dengan data \textit{training} 177, data \textit{testing }59, dengan parameter C=5, tol=0.001, dan max\_passes=5. Pengujian ini akan mencari kombinasi fitur terbaik untuk klasifikasi 4 kelas (positif, negatif, netral, sarkasme) dan 3 kelas (positif, negatif, netral). Berikut ini adalah tabel pengujian pada kombinasi fitur 1 pada klasifikasi 4 kelas dan 3 kelas:

\begin{small}
	\begin{longtable}{@{\extracolsep{\fill}}|p{1cm}|p{2cm}|p{2cm}|p{1.1cm}|p{1.1cm}|p{1.1cm}|p{1.1cm}|p{1.1cm}|}
		\caption{Tabel Pengujian Kombinasi Fitur 1}\\
		\hline
		\multirow{2.5}{*}{\textbf{\parbox{1cm}{No / Kelas}}} & \multirow{2.5}{*}{\parbox{2cm}{\textbf{Kombinasi fitur non-sarkasme}}} & \multirow{2.5}{*}{\parbox{2cm}{\textbf{Kombinasi fitur sarkasme}}} & 
		\multicolumn{5}{c|}{\textbf{Akurasi \textit{F-Measure}}} \\
		\cline{4-8}
		& & & \textbf{FP\_L/ FP\_D} & \textbf{FN\_L/ FN\_D} & \textbf{FT\_L/ FT\_D} & \textbf{FS\_L/ FS\_D} & \textbf{LM/ DM} \\
		\hline
		\endhead
		1/4 & KF1 & KF2 & 0.72 / 0.71 & 0.71 / 0.69 & 0.82 / 0.83 & 0.24 / 0.25 
		& 0.62 / 0.62 \\
		\hline
		1/3 & KF1 & - & 0.88 / 0.81 & 0.75 / 0.75 & 0.88 / 0.83 & - & 0.84 / 
		0.80 \\
		\hline
		2/4 & U + PB + QW & KF2 & 0.61 / 0.63 & 0.55 / 0.53 & 0.73 / 0.79 & 0.13 
		/ 0.13 & 0.51 / 0.52 \\
		\hline
		2/3 & U + PB + QW & - & 0.63 / 0.62 & 0.63 / 0.59 & 0.91 / 0.81 & - & 
		0.72 / 0.67 \\
		\hline
		3/4 & U + SS + QW & KF2 & 0.71 / 0.71 & 0.65 / 0.65 & 0.74 / 0.74 & 0.27 
		/ 0.27 & 0.59 / 0.59 \\
		\hline
		3/3 & U + SS + QW & - & 0.79 / 0.71 & 0.71 / 0.71 & 0.78 / 0.74 & - & 
		0.76 / 0.72 \\
		\hline
		4/4 & U + SS + PB & KF2 & 0.70 / 0.74 & 0.73 / 0.72 & 0.75 / 0.83 & 0.25 
		/ 0.29 & 0.61 / 0.65 \\
		\hline
		4/3 & U + SS + PB & - & 0.82 / 0.79 & 0.81 / 0.77 & 0.88 / 0.83 & - & 
		0.84 / 0.80 \\
		\hline
		5/4 & U + SS & KF2 & 0.63 / 0.63 & 0.74 / 0.79 & 0.65 / 0.72 & 0.29 / 
		0.29 & 0.58 / 0.61 \\
		\hline
		5/3 & U + SS & - & 0.67 / 0.67 & 0.74 / 0.76 & 0.69 / 0.72 & - & 0.70 / 
		0.72 \\
		\hline
		6/4 & U + PB & KF2 & 0.69 / 0.69 & 0.63 / 0.65 & 0.76 / 0.78 & 0.24 / 
		0.13 & 0.58 / 0.56 \\
		\hline
		6/3 & U + PB & - & 0.58 / 0.6 & 0.67 / 0.67 & 0.85 / 0.88 & - & 0.70 / 
		0.72 \\
		\hline
		7/4 & U + QW & KF2 & 0.72 / 0.72 & 0.53 / 0.57 & 0.74 / 0.70 & 0.14 / 
		0.14 & 0.53 / 0.53 \\
		\hline
		7/3 & U + QW & - & 0.65 / 0.60 & 0.59 / 0.61 & 0.80 / 0.76 & - & 0.68 / 
		0.66 \\
		\hline
		8/4 & SS + PB + QW & KF2 & 0.63 / 0.63 & 0.44 / 0.44 & 0.81 / 0.81 & 
		0.19 / 0.19 & 0.52 / 0.52 \\
		\hline
		8/3 & SS + PB + QW & - & 0.63 / 0.63 & 0.63 / 0.59 & 0.87 / 0.84 & - & 
		0.71 / 0.69 \\
		\hline
	\end{longtable}
\end{small}

Kolom "No/Kelas" pada pengujian tabel di atas, menunjukkan nomor urut pengujian dan klasifikasi kelasnya. Sebagai contoh, 1/4 adalah nomor urut pengujian ke-1 dengan klasifikasi 4 kelas (positif, negatif, netral, sarkasme), dan 1/3 adalah nomor urut pengujian ke-1 dengan klasifikasi 3 kelas (positif, negatif, netral). Berdasarkan pengujian kombinasi fitur di atas, kombinasi fitur terbaik yang dihasilkan pada fitur non-sarkasme klasifikasi 4 kelas dan 3 kelas adalah \textit{unigram, punctuation based }dan \textit{sentiment score}. Pengujian kombinasi fitur 1 untuk klasifikasi 3 kelas tidak akan dilakukan lagi, karena kombinasi fitur 2 tidak akan digunakan untuk klasifikasi 3 kelas. Setelah mendapatkan kombinasi fitur terbaik pada pengujian tabel 4.11, maka selanjutnya mencari kombinasi fitur terbaik pada kombinasi fitur 2\textit{ }dengan percobaan menggunakan hasil terbaik dari kombinasi fitur 1. Berikut ini adalah tabel pengujian pada kombinasi fitur 2 pada klasifikasi 4 kelas:
\begin{table}[H]
	\caption{Tabel Pengujian Kombinasi Fitur 2}
	\small
	\begin{adjustbox}{width=1\textwidth}
		\begin{tabular}{|p{1cm}|p{2cm}|p{2cm}|p{1.1cm}|p{1.1cm}|p{1.1cm}|p{1.1cm}|p{1.1cm}|}
			\hline
			\multirow{2.5}{*}{\textbf{\parbox{1cm}{No}}} & \multirow{2.5}{*}{\parbox{2cm}{\textbf{Kombinasi fitur non-sarkasme}}} & \multirow{2.5}{*}{\parbox{2cm}{\textbf{Kombinasi fitur sarkasme}}} & 
			\multicolumn{5}{c|}{\textbf{Akurasi \textit{F-Measure}}} \\
			\cline{4-8}
			& & & \textbf{FP\_L/ FP\_D} & \textbf{FN\_L/ FN\_D} & \textbf{FT\_L/ FT\_D} & \textbf{FS\_L/ FS\_D} & \textbf{LM/ DM} \\
			\hline
			1 & U + SS + PB & KF2 & 0.70 / 0.74 & 0.73 / 0.72 & 0.75 / 0.83 & 0.25 / 
			0.29 & 0.61 / 0.65 \\
			\hline
			2 & U + SS + PB & T + IW + C + POS + SS + PB & 0.69 / 0.71 & 0.73 / 0.75 
			& 0.75 / 0.83 & 0.22 / 0.25 & 0.60 / 0.64 \\
			\hline
			3 & U + SS + PB & U + IW + C + POS + SS + PB & 0.76 / 0.78 & 0.73 / 0.75 
			& 0.75 / 0.83 & 0.25 / 0.29 & 0.62 / 0.66 \\
			\hline
			4 & U + SS + PB & U + T + IW + C + SS + PB & 0.70 / 0.74 & 0.73 / 0.75 & 
			0.75 / 0.83 & 0.38 / 0.40 & 0.64 / 0.68 \\
			\hline
			5 & U + SS + PB & U + T + IW + C + PB & 0.72 / 0.73 & 0.73 / 0.75 & 0.75 
			/ 0.83 & 0.47 / 0.47 & 0.67 / 70 \\
			\hline
			6 & U + SS + PB & U + T + IW + C & 0.75 / 0.76 & 0.71 / 0.71 & 0.79 / 
			0.83 & 0.57 / 0.57 & 0.70 / 0.72 \\
			\hline
			7 & U + SS + PB & U + T + C & 0.75 / 0.76 & 0.71 / 0.71 & 0.79 / 0.83 & 
			0.57 / 0.57 & 0.70 / 0.72 \\
			\hline
			8 & U + SS + PB & U + T & 0.78 / 0.77 & 0.71 / 0.71 & 0.79 / 0.83 & 0.46 
			/ 0.5 & 0.69 / 0.70 \\
			\hline
			9 & U + SS + PB & U + C & 0.75 / 0.76 & 0.71 / 0.71 & 0.79 / 0.83 & 0.57 
			/ 0.57 & 0.70 / 0.72 \\
			\hline
			10 & U + SS + PB & U & 0.74 / 0.75 & 0.71 / 0.71 & 0.79 / 0.83 & 0.36 / 
			0.36 & 0.65 / 0.66 \\
			\hline
		\end{tabular}
	\end{adjustbox}

\end{table}
Berdasarkan pengujian di atas, kombinasi fitur 2 yang terbaik adalah \textit{unigram}, \textit{topic, }dan \textit{capitalization}, dengan metode klasifikasi terbaik adalah \textit{direct method}.
\subsubsection{Analisis \textit{Error} Fitur Non-Sarkasme}
Pada bagian ini akan dilakukan analisis \textit{error }terhadap setiap fitur yang ada pada fitur non-sarkasme, yaitu \textit{unigram}, \textit{punctuation based} dan \textit{sentiment score}.
\begin{enumerate}[leftmargin=*,nolistsep]
	\item Fitur \textit{Unigram}\\
	Pada bagian ini akan dilakukan analisis \textit{error} ketika hanya menggunakan fitur \textit{sentiment score} dan \textit{punctuation based}, tanpa fitur \textit{unigram}. Berikut ini adalah analisis \textit{error} pada fitur \textit{unigram}:
	
	\begin{table}[H]
		\caption{Analisis \textit{Error} Fitur \textit{Unigram} pada Klasifikasi Non-Sarkasme}
		\centering
		\small
		\begin{adjustbox}{width=1\textwidth}
		\begin{tabular}{|p{1.7cm}|p{3.15cm}|p{4.3cm}|p{1cm}|p{1.3cm}|}
			\hline
			\textbf{KF} & \textbf{Teks} & \textbf{Fitur} & \textbf{Label} & \textbf{Prediksi} \\
			\hline
			\multirow{6}{*}{\parbox{1.7cm}{U + PB + SS}}& \multirow{6}{*}{\parbox{3.15cm}{sarankan tema mading sekolah}} & positive\_sentiment: 0.031 
			& \multirow{6}{*}{\parbox{1cm}{Netral}} & \multirow{6}{*}{\parbox{1.3cm}{Netral}} \\
			\cline{3-3}
			& & negative\_sentiment: 0.0 & & \\
			\cline{3-3}
			& & questionM: 0.0 & & \\
			\cline{3-3}
			& & sekolah: 0.13 & & \\
			\cline{3-3}
			& & \textbf{saran: 0.48} & & \\
			\cline{3-3}
			& & \textbf{tema: 0.48} & & \\
			\hline
			\multirow{3}{*}{\parbox{1.7cm}{SS + PB}} & \multirow{3}{*}{\parbox{3.15cm}{sarankan tema mading sekolah}} & \textbf{positive\_sentiment: 
				0.031} & \multirow{3}{*}{\parbox{1cm}{Netral}} & \multirow{3}{*}{\parbox{1.3cm}{Positif}} \\
			\cline{3-3}
			& & negative\_sentiment: 0.0 & & \\
			\cline{3-3}
			& & questionM: 0.0 & & \\
			\hline
		\end{tabular}
		\end{adjustbox}
	\end{table}
	Berdasarkan hasil percobaan klasifikasi di atas, tanpa fitur \textit{unigram}, teks akan selalu diklasifikasikan berdasarkan nilai sentimen. Hal ini terjadi karena tidak ada fitur kata seperti "saran" dan "tema" yang termasuk kata netral pada data \textit{training}, sehingga klasifikasi tanpa \textit{unigram} bergantung dengan nilai sentimen dari sebuah kalimat. Sebagai contoh teks netral 
	pada data \textit{training} yang menggunakan kata "saran": "aku minta saran dong enaknya jualan apa ya modalnya gak terlalu banyak bisa online+tawarin temen sekolah. makasih sarannya$<$3".
	
	\item Fitur \textit{Sentiment Score}\\
	Pada bagian ini akan dilakukan analisis \textit{error} ketika hanya menggunakan fitur \textit{unigram }dan \textit{punctuation} \textit{based}, tanpa fitur \textit{sentiment score}. Berikut ini adalah analisis \textit{error} pada fitur \textit{sentiment score}:
	
	\begin{table}[H]
		\caption{Analisis \textit{Error} Fitur \textit{Sentiment Score} pada Klasifikasi Non-Sarkasme}
		\centering
	\small
	\begin{adjustbox}{width=1\textwidth}
	\begin{tabular}{|p{1.7cm}|p{3.15cm}|p{4.3cm}|p{1cm}|p{1.3cm}|}
		\hline
		\textbf{KF} & \textbf{Teks} & \textbf{Fitur} & \textbf{Label} & \textbf{Prediksi} \\
		\hline
		\multirow{4}{*}{\parbox{1.7cm}{U + PB + SS}}& \multirow{4}{*}{\parbox{3.15cm}{ruang sekolah}} & questionM: 0.0 
		& \multirow{4}{*}{\parbox{1cm}{Netral}} & \multirow{4}{*}{\parbox{1.3cm}{Netral}} \\
		\cline{3-3}
		& & sekolah: 0.27 & & \\
		\cline{3-3}
		& &  \textbf{positive\_sentiment: 0.034} & & \\
		\cline{3-3}
		& & \textbf{negative\_sentiment: 0.017} & & \\
		\hline
		\multirow{2}{*}{\parbox{1.7cm}{U + PB}} & \multirow{2}{*}{\parbox{3.15cm}{ ruang sekolah}} & questionM: 0.0 & \multirow{2}{*}{\parbox{1cm}{Netral}} & \multirow{2}{*}{\parbox{1.3cm}{Positif}} \\
		\cline{3-3}
		& & \textbf{sekolah: 0.27}& & \\
		\hline
	\end{tabular}
	\end{adjustbox}
	\end{table}
	Berdasarkan hasil percobaan klasifikasi di atas, tanpa fitur \textit{sentiment score} teks netral yang pendek dan tidak memiliki tanda baca tanya (?) tidak akan bisa diklasifikasikan ke kelas yang seharusnya, yaitu netral. Hal ini disebabkan oleh klasifikasi yang menjadi bergantung terhadap fitur kemunculan kata atau \textit{unigram}. Pada percobaan di atas, teks netral diprediksi sebagai teks positif, karena cukup banyak kemunculan kata "sekolah" pada data \textit{training} 
	yang termasuk pada teks positif.
	
	\item Fitur \textit{Punctuation Based}\\
	Pada bagian ini akan dilakukan analisis \textit{error} ketika hanya menggunakan fitur \textit{unigram }dan \textit{sentiment score}, tanpa fitur \textit{punctuation based}. Berikut ini adalah analisis	\textit{error} pada fitur \textit{punctuation based}:
	\begin{table}[H]
		\caption{Analisis \textit{Error} Fitur \textit{Punctuation Based} pada Klasifikasi Non-Sarkasme}
		\centering
		\small
		\begin{adjustbox}{width=1\textwidth}
		\begin{tabular}{|p{1.7cm}|p{3.15cm}|p{4.3cm}|p{1cm}|p{1.3cm}|}
			\hline
			\textbf{KF} & \textbf{Teks} & \textbf{Fitur} & \textbf{Label} & \textbf{Prediksi} \\
			\hline
			\multirow{5}{*}{\parbox{1.7cm}{U + PB + SS}}& \multirow{5}{*}{\parbox{3.15cm}{Sekolah mu berani terima tantangan ? keberanian mention ig}} & positive\_sentiment: 0.69
			& \multirow{5}{*}{\parbox{1cm}{Netral}} & \multirow{5}{*}{\parbox{1.3cm}{Netral}} \\
			\cline{3-3}
			& &negative\_sentiment: 0.49 & & \\
			\cline{3-3}
			& &  \textbf{questionM: 0.2} & & \\
			\cline{3-3}
			& & sekolah: 0.069 & & \\
			\cline{3-3}
			 & & terima: 0.16 & & \\
			\hline
			\multirow{4}{*}{\parbox{1.7cm}{U + SS}} & \multirow{4}{*}{\parbox{3.15cm}{sekolah mu berani terima tantangan ? keberanian mention ig}} & \textbf{positive\_sentiment: 0.69} & \multirow{4}{*}{\parbox{1cm}{Netral}} & \multirow{4}{*}{\parbox{1.3cm}{Positif}} \\
			\cline{3-3}
			& & negative\_sentiment: 0.49 & & \\
			\cline{3-3}
			& & sekolah: 0.069 & & \\
			\cline{3-3}
			& & terima: 0.16 & & \\
			\hline
		\end{tabular}
		\end{adjustbox}
	\end{table}
	Berdasarkan hasil percobaan klasifikasi di atas, tanpa fitur \textit{punctuation based},\textit{ }teks netral yang memiliki nilai sentimen tidak akan bisa diklasifikasikan ke kelas yang seharusnya, yaitu netral. Hal ini terjadi karena klasifikasi menjadi sangat bergantung terhadap fitur \textit{sentiment} \textit{score}. Pada percobaan di atas, nilai sentimen positif lebih tinggi dibanding negatif, oleh karena itu teks diklasifikasikan sebagai teks positif.
\end{enumerate}

\subsubsection{Analisis \textit{Error} Fitur Sarkasme}
Pada bagian ini akan dilakukan analisis \textit{error }terhadap setiap fitur yang ada pada fitur sarkasme, yaitu \textit{unigram}, \textit{topic}, dan \textit{capitalization}.
\begin{enumerate}[leftmargin=*,nolistsep]
	\item Fitur \textit{Unigram}\\
	Pada bagian ini akan dilakukan analisis \textit{error} ketika hanya menggunakan fitur \textit{topic} dan \textit{capitalization}, tanpa fitur \textit{unigram}. Berikut ini adalah analisis \textit{error} pada fitur \textit{unigram}:
	
	\begin{table}[H]
		\caption{Analisis \textit{Error} Fitur \textit{Unigram} pada Klasifikasi Sarkasme}
		\small
		\begin{adjustbox}{width=1\textwidth}
			\begin{tabular}{|p{1cm}|p{5cm}|p{2.45cm}|p{1.45cm}|p{1.45cm}|}
				\hline
				\textbf{KF} & \textbf{Teks} & \textbf{Fitur} & \textbf{Label} & \textbf{Prediksi} \\
				\hline
				\multirow{8}{*}{\parbox{1cm}{U + T + C}}& \multirow{8}{*}{\parbox{5cm}{@Budhiheriawan @richard\_errik @whytas @riri\_hesria @Ria1Kartolo @AdieRinaldi @fuadpuad makin pinter aja anggota DPR \#sarkasme}} & \parbox{2.45cm}{topic 0: 0.055} 
				& \multirow{8}{*}{\parbox{1.45cm}{Sarkasme}} & \multirow{8}{*}{\parbox{1.45cm}{Sarkasme}} \\
				\cline{3-3}
				& & \parbox{2.45cm}{topic 1: 0.059} & & \\
				\cline{3-3}
				& & \parbox{2.45cm}{topic 2: 0.036} & & \\
				\cline{3-3}
				& &\parbox{2.45cm}{\textbf{topic 3: 0.84}} & & \\
				\cline{3-3}
				& & \parbox{2.45cm}{\textbf{capitalization: 0.14}} & & \\
				\cline{3-3}
				& &\parbox{2.45cm}{\textbf{dpr: 0.26}} & & \\
				\cline{3-3}
				& &\parbox{2.80cm}{pintar: 0.74} & & \\
				\cline{3-3}
				& & \parbox{3.25cm}{\textbf{angota: 0.44}} & & \\
				\hline
				\multirow{5}{*}{\parbox{1cm}{T + C}} & \multirow{5}{*}{\parbox{5cm}{ @Budhiheriawan @richard\_errik @whytas @riri\_hesria @Ria1Kartolo @AdieRinaldi @fuadpuad makin pinter aja anggota DPR \#sarkasme }} & \parbox{2.45cm}{topic 0: 0.055} & \multirow{5}{*}{\parbox{1.45cm}{Sarkasme}} & \multirow{5}{*}{\parbox{1.45cm}{Positif}} \\
				\cline{3-3}
				& & \parbox{2.45cm}{topic 1: 0.059} & & \\
				\cline{3-3}
				& & \parbox{2.45cm}{topic 2: 0.036} & & \\
				\cline{3-3}
				& &\parbox{2.45cm}{\textbf{topic 3: 0.84}} & & \\
				\cline{3-3}
				& & \parbox{2.45cm}{\textbf{capitalization: 0.14}} & & \\
				\cline{3-3}
				\hline
			\end{tabular}
		\end{adjustbox}
	\end{table}
	Berdasarkan hasil percobaan klasifikasi di atas, tanpa fitur \textit{unigram}, klasifikasi akan salah, karena fitur yang digunakan tidak memberikan informasi yang cukup untuk mengklasifikasikan sarkasme.
	
	\item Fitur \textit{Topic}\\
	Pada bagian ini akan dilakukan analisis \textit{error} ketika hanya	menggunakan fitur \textit{unigram}, tanpa fitur \textit{topic}. Fitur \textit{capitalization} tidak digunakan untuk analisis \textit{error} karena pada pengujian fitur tabel 4.13 menunjukkan tidak ada perubahan akurasi saat menggunakan semua fitur terbaik sarkasme, yaitu \textit{unigram}, \textit{capitalization}, \textit{topic} dan 
	tanpa fitur \textit{topic}. Berikut ini adalah analisis fitur \textit{topic} pada data \textit{training }dengan label sarkasme:
	\begin{small}
	
		\begin{longtable}{@{\extracolsep{\fill}}|p{0.5cm}|p{5.5cm}|p{1.2cm}|p{1.2cm}|p{1.2cm}|p{1.2cm}|}
			\caption{Analisis Fitur \textit{Topic}} \\
			\hline
			\textbf{No} & \textbf{Teks} & \textbf{Topic 0} & \textbf{Topic 1} & \textbf{Topic 2} & \textbf{Topic 3} \\
			\hline
			\endhead
			1 & Dan besok sekolah, hell yeah ! :D \#sarkasme & \textbf{0.88} & 0.046 & 0.028 & 0.038 
			\\
			\hline
			2 & BPK terbaik. BI terbaik. DPR.....ter..? @Fahrihamzah ... ter...? Mandi dulu Ooom! & \textbf{0.93} & 
			0.025 & 0.015 & 0.020 \\
			\hline
			3 & astaga, besar sekali hati internet provider satu ini, isinya minta maaf trus \#sarkasme & \textbf{0.92} & 0.032 
			& 0.019 & 0.027 \\
			\hline
			4 & Wow, abis ditinggal sesiangan, jam 8 malam internet @innovateIND @HOMElinksBSD udah nyala lagi \#sarkasme & \textbf{
				0.93} & 0.024 & 0.015 & 0.020 \\
			\hline
			5 & Pantesan namanya dia \& papa bs hilang dari list anggota DPR kasus e-KTP Ternyata mahluk paling suci sebagai gedung DPR & 0.016 & \textbf{0.95} & 0.010 & 0.014 \\
			\hline
			6 & Oh @fadlizon betapa beruntungnya DPR RI punya sampeyan. \#sarkasme \#ILC & 0.036 & 0.038 & 0.023 & \textbf{
				0.90} \\
			\hline
			7 & Nilai-nilai gue angkanya gede-gede bgt ya wow \#sarkasme \#edisidepresi & 0.036 & 0.040 & 0.023 & 
			\textbf{0.89} \\
			\hline
			8 & hebat fadlizon ketemu sahabat lama sepaham dan seidiologi dari rusia & 0.027 & 
			0.028 & 0.017 & \textbf{0.92} \\
			\hline
			9 & jangan salahkan film horor lokal, mereka hanya menuruti KEMAUAN dan KEMALUAN pasar \#sarkasme & 0.023 & 
			\textbf{0.93} & 0.015 & 0.020 \\
			\hline
			10 & Owh internet connection-nya keren sekali,saat dibutuhkan sangat bisa diandalkan \#sarkasme \#sinis \#speechless & \textbf{0.90
			} & 0.038 & 0.023 & 0.031 \\
			\hline
			11 & Hari ini ga ketemu @gistaanindya ckck sekolah segede apa sih? \#sarkasme & \textbf{0.91} & 0.033 & 
			0.020 & 0.027 \\
			\hline
			12 & TERIMA KASIH INTERNET & \textbf{0.85} & 0.058 & 0.036 & 0.049 \\
			\hline
			13 & Salam buat Ketua Dewan-nya ya. Dah sehatkah...? & 0.043 & \textbf{0.88} & 0.028 
			& 0.038 \\
			\hline
			14 & Semoga FH dan FZ ada di DPR selamanya.. krn sepi dunia kalau gak ada mereka.. :) & 0.021 & 
			\textbf{0.94} & 0.013 & 0.018 \\
			\hline
			15 & film berkualitas "Mr Bean Kesurupan DEPE"   jadi bangga dengan negeri ini,, \#sarkasme & 
			\textbf{0.93} & 0.025 & 0.015 & 0.020 \\
			\hline
			16 & Sepertinya sudah masuk jam2nya nih,internet di sini luar biasa!!!!!!!\#sarkasme & 0.05 & \textbf{0.85} & 0.05 
			& 0.049 \\
			\hline
			17 & Lg istirahat nih di rumah habis sekolah dari pagi sampe sore...!!! & \textbf{0.92} & 
			0.032 & 0.019 & 0.027 \\
			\hline
			18 & Anggota DPR Pancasilais. Mereka amalkan sila ke-4. Musyawarah dan mufakat untuk membagi2 jarahan atas negeri ini. \#Sarkasme & 0.014 & 0.018 & \textbf{0.94} & 0.015 \\
			\hline
			19 & Yah ga bisa liat sekolah!!! & \textbf{0.88} & 0.047 & 0.028 & 
			0.039 \\
			\hline
			20 & besok sekolah. Hatiku gembiraaa! \#sarkasme & \textbf{0.88} & 0.046 & 0.028 
			& 0.038 \\
			\hline
			21 & masuk sekolah/.\ & 0.077 & 0.081 & 0.049 & \textbf{0.79} \\
			\hline
			22 & RT @fnabila: Asiiiik bentar lagi sekolah!!!!!!!! Ga sabar!!!! \#sarkasme & 0.036 
			& \textbf{0.90} & 0.023 & 0.032 \\
			\hline
			23 & Super sekali internet smartfren utk streaming...muter terussss...sampe ga keliatan... Waakakakakkakak \#sarkasme & 0.024 & \textbf{0.93} & 0.015 & 0.020 \\
			\hline
			24 & Anda lebih hebat lagi pak @Fahrihamzah tidak punya partai bisa jadi anggota dpr. & 0.036 & 0.038 & 0.023 & \textbf{
				0.90} \\
			\hline
			25 & Minggu....les :\ besok...sekolah:| oke,gw suka belajar. Sukaa bgt. \#sarkasme & 0.021 & 
			\textbf{0.94} & 0.018 & 0.013 \\
			\hline
			26 & Entar sist kalo anggarannya UDAH NAIK. & \textbf{0.80} & 0.080 & 
			0.049 & 0.066 \\
			\hline
			\multicolumn{2}{|c|}{Total}&12 & 8 & 1 & 5 \\
			\hline
		\end{longtable}
	\end{small}
		
	Berdasarkan analisis fitur \textit{topic} pada 26 data \textit{training} dengan label sarkasme, dapat disimpulkan setiap teks sarkasme cenderung termasuk pada topik 0, 1 dan 3, dari jumlah topik yang ada adalah 4. Hasil analisis fitur \textit{topic} di atas akan membantu dalam analisis \textit{error }pada fitur \textit{topic} yang akan dilakukan. Berikut ini analisis \textit{error} fitur \textit{topic}:
	\begin{table}[H]
		\caption{Analisis \textit{Error} Fitur \textit{Topic} pada Klasifikasi Sarkasme}
		\centering
		\small
		\begin{adjustbox}{width=1\textwidth}
		\begin{tabular}{|p{1cm}|p{5cm}|p{2.45cm}|p{1.45cm}|p{1.45cm}|}
			\hline
			\textbf{KF} & \textbf{Teks} & \textbf{Fitur} & \textbf{Label} & \textbf{Prediksi} \\
			\hline
			\multirow{8}{*}{\parbox{1cm}{U + T}}& \multirow{8}{*}{\parbox{5cm}{Berkaitan akses situs porno anggota dewan Badan Kehormatan DPR adakan investigasi siapa ditonton}} & \parbox{2.45cm}{topic 0: 0.044 }
			& \multirow{8}{*}{\parbox{1.45cm}{Sarkasme}} & \multirow{8}{*}{\parbox{1.45cm}{Sarkasme}} \\
			\cline{3-3}
			 & & \parbox{2.45cm}{\textbf{topic 1: 0.88}} & & \\
			 \cline{3-3}
			 & & \parbox{2.45cm}{\textbf{topic 2: 0.02}} & & \\
			 \cline{3-3}
			 & & \parbox{2.45cm}{\textbf{topic 3: 0.03}} & & \\
			 \cline{3-3}
			& & \parbox{2.45cm}{dewan: 0.16}  & & \\
			\cline{3-3}
			& & \parbox{2.45cm}{angota: 0.11} & & \\
			\cline{3-3}
			& & \parbox{2.45cm}{\textbf{tonton: 0.18}} & & \\
			\hline
			\multirow{4}{*}{\parbox{1cm}{U}} & \multirow{4}{*}{\parbox{5cm}{ Berkaitan akses situs porno anggota dewan Badan Kehormatan DPR adakan investigasi siapa ditonton}} & \parbox{2.45cm}{dpr: 0.06} & \multirow{4}{*}{\parbox{1.45cm}{Sarkasme}} & \multirow{4}{*}{\parbox{1.45cm}{Positif}} \\
			\cline{3-3}
			& & \parbox{2.45cm}{dewan: 0.16}  & & \\
			\cline{3-3}
			& & \parbox{2.45cm}{angota: 0.11} & & \\
			\cline{3-3}
			& & \parbox{2.45cm}{\textbf{tonton: 0.18}} & & \\
			\hline
		\end{tabular}
		\end{adjustbox}
	\end{table}
	Berdasarkan hasil percobaan klasifikasi di atas, tanpa fitur \textit{topic},\textit{ }klasifikasi akan salah, jika terdapat kata yang kemunculannya sering terdapat pada kelas selain kelas sebenarnya. Pada kasus penggunaan di atas kata "tonton" menjadi fitur yang paling menentukan kelas dari sebuah teks, sehingga ketika fitur \textit{topic} tidak digunakan klasifikasi sarkasme menjadi salah. Sebagai contoh teks positif pada data \textit{training} yang menggunakan kata "tonton", yaitu "Jadi support film Indonesia, tapi juga harus jeli apa yang kita tonton ... hmm, aulion bener nih.".
	
	\item Fitur \textit{Capitalization}\\
	Pada bagian ini akan dilakukan analisis \textit{error} ketika hanya menggunakan fitur \textit{unigram}, \textit{topic}, dan \textit{sentiment score}, tanpa fitur \textit{capitalization}. Berikut ini adalah analisis \textit{error} fitur \textit{capitalization}:
		\begin{table}[H]
			\caption{Analisis \textit{Error} Fitur \textit{Capitalization} pada Klasifikasi Sarkasme}
			\centering
			\small
			\begin{adjustbox}{width=1\textwidth}
				\begin{tabular}{|p{1.65cm}|p{3.15cm}|p{3.45cm}|p{1.45cm}|p{1.45cm}|}
					\hline
					\textbf{KF} & \textbf{Teks} & \textbf{Fitur} & \textbf{Label} & \textbf{Prediksi} \\
					\hline
					\multirow{9}{*}{\parbox{1.65cm}{U + T + C}}& \multirow{9}{*}{\parbox{3.15cm}{Hampir 11 jam dekat sekolah I LOVE MY SENIOR LIFE PEEPS!}} & \textbf{topic 0: 0.46} 
					& \multirow{9}{*}{\parbox{1.45cm}{Sarkasme}} & \multirow{9}{*}{\parbox{1.45cm}{Sarkasme}} \\
					\cline{3-3}
					& & topic 1: 0.060 & & \\
					\cline{3-3}
					& & topic 2: 0.036 & & \\
					\cline{3-3}
					& & topic 3: 0.043 & & \\
					\cline{3-3}
					& & \textbf{capitalization: 4.0} & & \\
					\cline{3-3}
					& & sekolah: 0.079 & & \\
					\cline{3-3}
					& & life: 0.32 & & \\
					\cline{3-3}
					& & jam: 0.23 & & \\
					\cline{3-3}
					& & cinta: 0.27 & & \\
					\hline
					\multirow{8}{*}{\parbox{1.65cm}{U + T}} & \multirow{8}{*}{\parbox{3.15cm}{Hampir 11 jam dekat sekolah I LOVE MY SENIOR LIFE PEEPS!}} & \textbf{topic 0: 0.46} & \multirow{8}{*}{\parbox{1.45cm}{Sarkasme}} & \multirow{8}{*}{\parbox{1.45cm}{Positif}} \\
					\cline{3-3}
					& & topic 1: 0.060 & & \\
					\cline{3-3}
					& & topic 2: 0.036 & & \\
					\cline{3-3}
					& & topic 3: 0.043 & & \\
					\cline{3-3}
					& & sekolah: 0.079 & & \\
					\cline{3-3}
					& & life: 0.32 & & \\
					\cline{3-3}
					& & jam: 0.23 & & \\
					\cline{3-3}
					& & cinta: 0.27 & & \\
					\hline
				\end{tabular}
			\end{adjustbox}
		\end{table}
Berdasarkan hasil percobaan klasifikasi di atas, tanpa fitur \textit{capitalization}, klasifikasi akan salah, jika terdapat sebuah teks yang memiliki nilai topik yang kecil. Hal ini terjadi karena nilai topik dari teks hanya 0.46, sedangkan kata "cinta" yang cenderung positif pada data \textit{training}, hal ini yang menyebabkan klasifikasi salah ketika tidak ada fitur \textit{capitalization}.
\end{enumerate}

\subsection{Pengujian Parameter pada SMO}
Pada bagian ini, akan dilakukan pengujian parameter SMO yaitu, C, \textit{tolerance}, dan max\_passes. Nilai parameter C yang akan digunakan untuk pengujian adalah 1, 5 dan 10. Nilai parameter max\_passes yang akan digunakan untuk pengujian adalah 5 dan 10. Nilai parameter tol yang akan digunakan untuk pengujian adalah 0.001 dan 0.0001. Berikut ini adalah tabel pengujian parameter SMO:

\begin{table}[H]
	\caption{Pengujian Parameter SMO}
	\centering
	\small
	\begin{adjustbox}{width=1\textwidth}
	\begin{tabular}{|p{1.1cm}|p{1.7cm}|p{2cm}|p{1.1cm}|p{1.1cm}|p{1.1cm}|p{1.1cm}|p{1.1cm}|p{1.1cm}|}
		\hline
		\multicolumn{3}{|c|}{\textbf{Parameter}} & \multicolumn{5}{c|}{\textbf{Akurasi} \textit{\textbf{F-Measure}}} \\
		\hline
		\textbf{C} & \textbf{Tol} & \textbf{MaxPasses} & \textbf{FP\_D} & \textbf{FN\_D} & \textbf{FT\_D} & \textbf{FS\_D} & \textbf{DM} \\
		\hline
		1 & 0.001 & 5 & 0.65 & 0.52 & 0.89 & 0.18 & 0.56 \\
		\hline
		1 & 0.001 & 10 & 0.65 & 0.52 & 0.89 & 0.18 & 0.56 \\
		\hline
		1 & 0.0001 & 5 & 0.65 & 0.52 & 0.89 & 0.18 & 0.56 \\
		\hline
		1 & 0.0001 & 10 & 0.65 & 0.52 & 0.89 & 0.18 & 0.56 \\
		\hline
		5 & 0.001 & 5 & 0.76 & 0.71 & 0.83 & 0.57 & 0.72 \\
		\hline
		5 & 0.001 & 10 & 0.76 & 0.71 & 0.83 & 0.57 & 0.72 \\
		\hline
		5 & 0.0001 & 5 & 0.76 & 0.71 & 0.83 & 0.57 & 0.72 \\
		\hline
		5 & 0.0001 & 10 & 0.76 & 0.71 & 0.83 & 0.57 & 0.72 \\
		\hline
		10 & 0.001 & 5 & 0.74 & 0.75 & 0.81 & 0.57 & 0.72 \\
		\hline
		10 & 0.001 & 10 & 0.74 & 0.75 & 0.81 & 0.57 & 0.72 \\
		\hline
		10 & 0.0001 & 5 & 0.74 & 0.75 & 0.81 & 0.57 & 0.72 \\
		\hline
		10 & 0.0001 & 10 & 0.74 & 0.75 & 0.81 & 0.57 & 0.72 \\
		\hline
	\end{tabular}
	\end{adjustbox}
\end{table}
Berdasarkan pengujian parameter di atas, parameter yang menghasilkan akurasi terbaik adalah parameter C=5 dan C=10. Parameter C=10 memberikan hasil akurasi yang lebih merata pada setiap kelasnya, sedangkan pada parameter C=5, memberikan akurasi kelas positif yang lebih tinggi, namun akurasi kelas negatif menjadi rendah. Oleh karena itu akan dipilih parameter C=10 dengan \textit{tolerance }0.001 dan \textit{max passes
} 5. Alasan pemilihan parameter tersebut adalah semakin tinggi parameter \textit{tolerance }dan \textit{max passes} tidak terjadi adanya perubahan pada akurasi, selain itu alasan pemilihan parameter \textit{tolerance }dan \textit{max passes }yang lebih rendah adalah supaya \textit{training} yang dilakukan menjadi lebih cepat.
\subsection{Pengujian Klasifikasi 4 kelas, 3 kelas dan 1 kelas}
Pada bagian ini akan dilakukan pengujian untuk membandingkan akurasi \textit{f-measure }jika melakukan klasifikasi 4 kelas, yaitu positif, negatif, netral, sarkasme, kemudian mengklasifikasikan 3 kelas, yaitu positif, negatif dan netral, dan yang terakhir mengklasifikasikan 1 kelas sarkasme atau non-sarkasme. Klasifikasi kelas akan menggunakan fitur yang dihasilkan pada pengujian kombinasi fitur yang telah 
dilakukan, yaitu \textit{unigram, punctuation based} dan \textit{sentiment score }untuk klasifikasi non-sarkasme, dan fitur \textit{unigram, capitalization, topic }dan \textit{sentiment score} untuk klasifikasi sarkasme. Berikut ini adalah tabel pengujian klasifikasi 4 kelas:

\begin{table}[H]
	\caption{Pengujian Klasifikasi 4 Kelas (Positif, Negatif, Netral, Sarkasme)}
	\centering
	\small
	\begin{adjustbox}{width=1\textwidth}
	\begin{tabular}{|p{5.3cm}|p{1.2cm}|p{1.2cm}|p{1.2cm}|p{1.2cm}|p{1.2cm}|}
		\hline
		\multirow{2}{*}{\textbf{Klasifikasi}} & \multicolumn{5}{c|}{\textit{\textbf{F-Measure}}} \\
		\cline{2-6}
		& \textbf{FP\_D} & \textbf{FN\_L} & \textbf{FT\_D} & \textbf{FS\_D} & \textbf{DM} \\
		\hline
		Klasifikasi 4 kelas & 0.74 & 0.75 & 0.81 & 0.57 & 0.72 \\
		\hline
	\end{tabular}
	\end{adjustbox}
\end{table}

\begin{table}[H]
	\caption{Pengujian Klasifikasi 3 Kelas (Positif, Negatif, Netral)}
	\centering
	\small
	\begin{adjustbox}{width=1\textwidth}
	\begin{tabular}{|p{5.6cm}|p{1.5cm}|p{1.5cm}|p{1.5cm}|p{1.5cm}|}
		\hline
		\multirow{2}{*}{\textbf{Klasifikasi}} & \multicolumn{4}{c|}{\textit{\textbf{F-Measure}}} \\
		\cline{2-5}
		& \textbf{FP\_D} & \textbf{FN\_L} & \textbf{FT\_D} & \textbf{DM} \\
		\hline
		Klasifikasi 3 kelas & 0.79 & 0.81 & 0.86 & 0.82 \\
		\hline
	\end{tabular}
	\end{adjustbox}
\end{table}

\begin{table}[h]
	\caption{Pengujian Klasifikasi 1 Kelas (Sarkasme/Non-Sarkasme)}
	\centering
	\small
	\begin{adjustbox}{width=1\textwidth}
	\begin{tabular}{|p{8cm}|p{4.8cm}|}
		\hline
		\textbf{Klasifikasi} & \textit{\textbf{F-Measure}} \\
		\hline
		Klasifikasi 1 kelas (sarkasme/non-sarkasme) & 0.75 \\
		\hline
	\end{tabular}
	\end{adjustbox}
\end{table}
Berdasarkan hasil pengujian di atas, hasil akurasi meningkat sebanyak 10\% ketika hanya mengklasifikasikan 3 kelas. Dan akurasi deteksi sarkasme meningkat sebanyak 18\% ketika hanya melakukan deteksi sarkasme atau non-sarkasme.
\newpage
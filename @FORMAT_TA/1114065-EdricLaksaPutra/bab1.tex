%-----------------------------------------------------------------------------%
\chapter{PENDAHULUAN}
%-----------------------------------------------------------------------------%

\vspace{4.5pt}

\section{Latar Belakang Masalah} \label{sec:latar_belakang}
Sistem komputer adalah interaksi dari perangkat lunak dan perangkat keras yang membentuk sebuah jaringan elektronik. Tugas dari sebuah sistem adalah menerima input, memproses data input, menyimpan data olahan, dan menampilkan output sebagai bentuk informasi. Dalam penerapannya, kita menyebut sistem aplikasi sebagai program komputer yang bertugas untuk menyelesaikan kebutuhan khusus. Terdapat beberapa tahapan umum dalam mengembangkan sistem aplikasi yaitu perencanaan, analisa, desain, pengembangan, testing, implementasi, dan pemeliharaan [1].  Tahap yang cukup penting dan akan menjadi fokus diskusi adalah desain dan pengembangan, yang dimana peran arsitektur perangkat lunak sangat berperan penting untuk menetapkan landasan dasar pengembangan aplikasi dari awal sampai selesai. Hasil dari arsitektur perangkat lunak merupakan struktur yang melandasi keberadaan komponen-komponen perangkat lunak, cara komponen untuk saling berinteraksi dan organisasi komponen dalam membentuk perangkat lunak [2]. Arsitektur yang paling sering digunakan saat ini adalah model monolitik. Arsitektur monolitik merupakan arsitektur yang mudah dimengerti dan dimodifikasi karena lebih sederhana implementasinya. Arsitektur ini menggunakan kode sumber dan teknologi yang serupa untuk menjalankan semua tugas-tugasnya. Secara garis besar keunggulan dari arsitektur monolitik dapat dirasakan apabila aplikasi ingin mudah untuk dikembangkan, mudah untuk di deploy, dan dapat selalu dipantau pertumbuhan perfomanya [5]. Namun apabila aplikasi semakin besar dan anggota tim semakin banyak, arsitektur monolitik akan menghadapi kekurangan yang semakin lama akan semakin signifikan. Kelemahan dari arsitektur monolitik dapat dirasakan dari sisi \textit{deployment}, tingkat keamanan, tingkat ketersediaan, dan manajemen aplikasi yang semakin sulit dan kualitasnya menurun.\\
Pada tahun 2015 sebuah arsitektur microservice merupakan arsitektur alternatif yang dapat mengatasi kelemahan dari monolitik. Arsitektur microservice menjadikan sebuah aplikasi lebih mandiri dan efisien. Namun kendala yang sering menjadi hambatan adalah setiap kasus memiliki masalah dan desain yang berbeda satu dengan yang lain, sehingga dibutuhkan desain yang tepat untuk setiap kasus yang diangkat. Kendala lain adalah sulitnya implementasi arsitektur baru, dibutuhkan panduan yang menjelaskan tahap migrasi dari arsitektur monolitik hingga menjadi microservice. Pada tahap terakhir, testing dibutuhkan untuk menjadi parameter keberhasilan arsitektur yang baru.\\
Penelitian ini akan mengangkat contoh kasus penerapan arsitektur microservice pada sistem informasi manajemen rumah sakit Apertura. Penelitian ini bertujuan untuk membuktikan bahwa arsitektur microservice dapat memberikan performa yang lebih baik untuk aplikasi Apertura di kemudian hari.
\section{Rumusan Masalah}
Berikut ini adalah rumusan masalah yang dibuat berdasarkan latar belakang di atas:
\begin {enumerate}[nolistsep, leftmargin=0.5cm]
\item Bagaimana melakukan migrasi dari arsitektur monolitik ke model arsitektur microservice?
\item Bagaimana perbandingan performa yang dihasilkan dari arsitektur microservice yang baru terhadap arsitektur monolitik?
\end{enumerate}

\section{Batasan Masalah}
Batasan masalah pada penelitian ini adalah penelitian hanya berfokus pada perancangan \textit{service} pada arsitektur microservice serta penerapannya secara general untuk ditinjau peluang dan keunggulannya dibandingkan arsitektur monolitik.

\section{Tujuan Penelitian}
Berdasarkan batasan masalah di atas, berikut ini adalah tujuan penelitian dari tugas akhir ini:
\begin{enumerate}[nolistsep,leftmargin=0.5cm]
\item Menentukan bagaimana tahap yang benar untuk melakukan migrasi dari arsitektur monolitik ke arsitektur microservice.
\item Membandingkan apa saja kelebihan dan kekurangan dari arsitektur microservice dibandingkan dengan arsitektur monolitik.
\end{enumerate}

\section{Kontribusi Penelitian}
Berikut ini adalah kontribusi penelitian yang diberikan pada pengembangan sistem analisis sentimen ini:
\begin{enumerate}[nolistsep,leftmargin=0.5cm]
\item Memberikan contoh panduan untuk melakukan migrasi dari arsitektur konvensional ke arsitektur microservice pada sistem informasi manajemen rumah sakit.
\item Memberikan hasil perbandingan arsitektur microservice dan monolitik.
\end{enumerate}

\section{Metode Penelitian}
Tahapan-tahapan yang akan dilakukan dalam pelaksanaan penelitian ini adalah sebagai berikut:
\begin{enumerate}[nolistsep,leftmargin=0.5cm]
\item Analisis\\
Melakukan studi literatur dan analisa melalui wawancara dengan pihak perancang aplikasi monolitik sebelumnya. Data juga dikumpulkan dari jurnal-jurnal, karya ilmiah, dan situs yang memberikan informasi yang menunjang mengenai konsep arsitektur microservice dan tahap-tahap implementasi pada aplikasi monolitik.
\item Identifikasi\\
Melakukan identifikasi mengenai target-target yang ingin dicapai setelah menerapkan arsitektur baru. Identifikasi ini berguna untuk menentukan perbandingan apa saja yang akan diperhatikan antara arsitektur microservice dan arsitektur monolitik ketika pengujian dilakukan.
\item Perancangan\\
Perancangan arsitektur microservice meliputi perancangan model dasar (proses bisnis), perancangan sistem basis data, perancangan kelas service yang baru dengan konsep microservice, juga perancangan jalur komunikasi dan pertukaran data antara kelas-kelas service yang akan dibuat.
\item Implementasi\\
Melakukan implementasi hasil perancangan dalam bentuk web service application (server side). Selanjutnya dibuat aplikasi client sederhana untuk input data dan pengujian performa dari arsitektur server yang telah dibuat.
\item Pengujian\\
Melakukan pengujian terhadap rancangan aplikasi microservice yang baru dan aplikasi monolitik dengan menggunakan data rumah sakit untuk mengetahui hasil kinerja dan perbandingan performa antara kedua jenis arsitektur.
\end{enumerate}

\section{Sistematika Penulisan}
Pada penelitian ini peneliti menyusun berdasarkan sistematika penulisan sebagai berikut: \\[0.5cm]
\noindent \textbf{BAB I \hspace{1cm} PENDAHULUAN}
\begin{addmargin}[2.35cm]{0em}
Berisi penjelasan mengenai latar belakang, rumusan masalah, batasan masalah, tujuan penelitian, manfaat penelitian, metodologi penelitian, dan sistematika penulisan laporan penelitian. 
\end{addmargin}
\noindent \textbf{BAB II \hspace{0.8cm} LANDASAN TEORI}
\begin{addmargin}[2.35cm]{0em}
Membahas tentang definisi dari arsitektur microservice, teori-teori pendukung, dan metode penerapan arsitektur baru yang akan dijadikan landasan dan dipelajari serta dirangkum dari berbagai sumber, seperti buku, karya tulis, jurnal, artikel dari situs ilmiah. Juga pembahasan sekilas mengenai arsitektur konvensional yang digunakan sebelumnya.
\end{addmargin}
\noindent \textbf{BAB III \hspace{0.7cm} ANALISIS DAN PERANCANGAN}
\begin{addmargin}[2.35cm]{0em}
Berisi analisa mengenai bagaimana konsep penerapan arsitektur microservice pada aplikasi rumah sakit dan modul-modul apa saja yang akan di migrasi menjadi modul baru untuk diimplementasikan pada tahap selanjutnya. Selanjutnya membahas perancangan aplikasi dari hasil analisa dengan menggunakan arsitektur microservice yang baru.
\end{addmargin}
\noindent \textbf{BAB IV \hspace{0.7cm} IMPLEMENTASI DAN PENGUJIAN}
\begin{addmargin}[2.35cm]{0em}
Berisi implementasi dari hasil perancangan aplikasi dalam bentuk perangkat lunak menggunakan teknologi \textit{web service}. Selanjutnya perangkat lunak diuji fungsi dan performanya dari sisi \textit{client} (pengguna).
\end{addmargin}
\noindent \textbf{BAB V \hspace{0.8cm} KESIMPULAN DAN SARAN}
\begin{addmargin}[2.35cm]{0em}
Memberikan kesimpulan berdasarkan hasil analisis, perancangan, implementasi dan pengujian, serta evaluasi yang dilakukan, juga saran-saran yang dibutuhkan untuk pengembangan lebih lanjut.
\end{addmargin}

\newpage
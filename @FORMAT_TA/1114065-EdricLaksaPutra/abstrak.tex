\chapter*{ABSTRAK}

\noindent Arsitektur perangkat lunak merupakan ringkasan yang menguraikan organisasi standar proses perangkat lunak. Arsitektur ini menguraikan pemesanan, antarmuka, kesalingbergantungan, dan hubungan lainnya di antara unsur-unsur organisasi standar proses perangkat lunak \cite{14}. Tantangan pada analisis penelitian adalah mengembangkan arsitektur \textit{microservice} untuk mengatasi masalah pada arsitektur \textit{monolithic} pada sistem yang semakin besar untuk menjaga sinergi dari sistem tetap efisien \cite{9}. Contoh kasus yang diangkat untuk penelitian ini adalah sistem rawat jalan pada Sistem Informasi Manajemen Rumah Sakit Apertura. Pengembangan arsitektur \textit{microservice} diawali dengan menentukan modul \textit{service} yang terbentuk dari hasil dekomposisi proses bisnis \cite{6}, dengan menggunakan \textit{pattern database per service} \cite{6}. Metode komunikasi antar \textit{service} menggunakan RESTful (\textit{Representational State Transfer}) API dengan format data yang dikirimkan berupa JSON (\textit{JavaScript Object Notation}) \cite{9} dan metode \textit{API Composition} untuk melakukan \textit{in-memory join} \cite{6}. Pengujian dilakukan dengan melakukan perbandingan analisa dari cara sistem bekerja untuk setiap parameter uji yang terdiri dari : tingkat performa, availabilitas, skalabilitas, dan reliabilitas sistem \cite{10}. Berdasarkan hasil analisa pengujian, arsitektur \textit{microservice} dapat menjadi solusi dalam menyelesaikan kendala dalam arsitektur \textit{monolithic} dan cocok diterapkan pada sistem Apertura untuk pengembangan sistem beberapa tahun kedepan.

\noindent\textbf{Kata Kunci}: \textit{Monolithic}, \textit{Microservice}, Arsitektur Perangkat Lunak, Rekayasa Perangkat Lunak.